\chapter{Implementation}
\label{pd-implementation}
Dieses Kapitel beschreibt die Einrichtung und Implementation von einigen der verwendeten \glslink{Library}{Libraries} sowie weiteren Komponenten.

\section{Kort Backend}
Das \kort{}-Backend existierte bereits und wurde von den Entwicklern nicht mehr abgeändert. 
Das Backend konnte von uns über die \gls{REST}-Schnittstelle angesprochen werden.
Dieses ist im Kapitel 10.2. REST-Schnittstellen der Bachelorarbeit von Jürg Hunziker und Stefan Oderbolz dokumentiert.\cite{ba-kort-2012}

\section{View Components}
\label{pd-implementation-components}
Components entsprechen den \hyperref[pd-flux-views]{Views} der \brand{Flux}-Architektur.
Component Klassen erweitern \inlinecode{React.Component}\footnote{\url{https://facebook.github.io/react/docs/component-api.html}} und implementieren mindestens die \inlinecode{render()} Methode\footnote{\url{https://facebook.github.io/react/docs/component-specs.html}}. 
In der \inlinecode{render()} Methode wird in \gls{JSX} Syntax definiert, wie die Component dargestellt werden soll.\newline
Eine Component kann andere Components wiederverwenden,
Dadurch wird sowohl ein hierarchischer Aufbau der View, als auch ein modularer Einsatz von Components  ermöglicht.\newline
Ein weiteres Konzept, das von Components realisiert wird, ist die Unterscheidung von Property und State.
Eine Property repräsentiert eine sich nicht ändernde Eigenschaft einer Component, während der State den aktuellen Zustand ausdrückt.
Properties werden vom Owner\footnote{Die Owner-Ownee-Beziehung wird unter \emph{Ownership} beschrieben: \url{https://facebook.github.io/react/docs/multiple-components.html}} gesetzt und können durch den Ownee als \inlinecode{propTypes} deklariert werden.
Wird eine State Variable neu gesetzt, führt dies zu einem erneuten Aufruf der \inlinecode{render()} Methode.\newline
Components dürfen den Zustand -- der in den Stores gehalten wird -- nie direkt manipulieren, sondern müssen dies über Actions auslösen.
In den Callbacks, welche sie bei den Stores registriert haben, können sie dann die Getter aufrufen um den neuen Zustand zu erhalten.
Components sollen nicht mehr über den Applikationszustand wissen als für ihre Ansicht nötig ist.
Die Logik sollte also so weit möglich in den Stores umgesetzt sein.\newline

\section{Libraries}
Damit die entsprechenden \glslink{Library}{Libraries} genutzt werden können, mussten sie mit dem Node-Package-Manager (npm) heruntergeladen werden. 
Mit dem \inlinecode{npm install}-Befehl wurden alle aufgelisteten Abhängigkeiten in der  \inlinecode{package.json}-Datei installiert und im Ordner \newline\inlinecode{node-modules} gespeichert.
Wenn eine Library eine Bridge zu nativem Code verwendet, muss sie zusätzlich im \brand{Android}- und \brand{iOS}-Projekt gelinkt werden. 
Dann konnten die Libraries in einer \brand{JavaScript}-Komponente importiert und verwendet werden.

Nachfolgend sind die wichtigsten Libraries aufgeführt, die von uns verwendet wurden.

\begin{table}[H]
\centering
\begin{tabular}{|p{0.25\threecelltabwidth}|p{0.15\threecelltabwidth}|p{0.60\threecelltabwidth}|}
\hline 
\textbf{Library} & \textbf{Version} & \textbf{Verwendung} \\
\hline 
flux & 2.1.1 & Nutzung des \hyperref[pd-flux-dispatcher]{Dispatchers} \\
\hline 
react & 15.0.2. & Entwicklung vom User Interface \\
\hline 
react-native & 0.26.3 & Entwicklung von native Apps mit \brand{React} \\
\hline 
react-native-google-signin & 0.6.0 & Native Authentifizierung mit Google OAuth \\
\hline 
react-native-i18n & 0.0.8 & Internationalisierung der Benutzeroberfläche \\
\hline 
react-native-mapbox-gl & 4.1.1 & Darstellung der Karte \\
\hline 
react-native-router-flux & 3.26.15 & Navigation im User Interface \\
\hline 
\end{tabular}
\caption{Abhängigkeiten im \brand{JavaScript}-Code}
\label{table-dependencies}
\end{table}

\begin{table}[H]
\centering
\begin{tabular}{|p{0.25\threecelltabwidth}|p{0.15\threecelltabwidth}|p{0.60\threecelltabwidth}|}
\hline 
\textbf{Library} & \textbf{Version} & \textbf{Verwendung} \\
\hline 
babel-eslint \newline
babel-jest \newline
babel-polyfill \newline
babel-preset-es2015 \newline
babel-preset-react \newline
babel-preset-react-native-stage-0
& 
6.0.4 \newline
12.1.0 \newline
6.9.0 \newline
6.6.0 \newline
6.5.0 \newline
1.0.1
& Compiler für die verschiedenen Libraries und \brand{ES2015} \\
\hline 
eslint \newline
eslint-config-airbnb \newline
eslint-plugin-react \newline
eslint-plugin-react-native
&
2.11.1 \newline
6.2.0 \newline
4.3.0 \newline
1.0.0
& Einhaltung der \nameref{pd-entwicklungsumgebung-cr} \\
\hline 
jest & 12.1.1 & \nameref{pd-entwicklungsumgebung-testing} \\
\hline 
\end{tabular}
\caption{Abhängigkeiten für die Entwicklung}
\label{table-dev-dependencies}
\end{table}

Die vollständige Übersicht aller Abhängigkeiten finden sich in der Datei \inlinecode{package.json}.

\subsection{Navigation}
Die Navigation ist in der Component \inlinecode{App} deklariert.
Alle Scenes werden mit einem Key deklariert und falls nötig mit Optionen für eine Tab-Ansicht erweitert. 
Die definierten Scenes werden als Kind-Komponenten einem Router übergeben. 
Jede Scene kann mit einem Funktionsaufruf 
\begin{itemize}
	\item die Properties aktualisieren.
	\item sich selber schliessen.
	\item eine andere Scene mit dem Key und mit optionalen Properties, als Parameter, aufrufen.
\end{itemize} 
Die \gls{API} der Navigations-Komponente ist auf der entsprechenden \brand{GitHub}-Seite\footnote{\url{https://github.com/aksonov/react-native-router-flux}} dokumentiert. 


\subsection{Karte}
Um die Map-Komponente von \brand{Mapbox} zu nutzen und ein Access-Token zu erhalten, muss auf der \brand{Mapbox}-Webseite\footnote{\url{https://www.mapbox.com/}} ein Benutzerkonto angelegt werden. 

Damit die Marker an der richtigen Position angezeigt werden, wurden der \brand{Mapbox}-Komponente ein Array mit Annotations als Parameter übergeben. 
Das Annotation-Array wird fortlaufend mit allen Missionen vom \gls{Backend} aktualisiert. 
Dabei werden nur Missionen in der Umgebung des Benutzers geladen. 
Eine Annotation enthält die Koordinaten und die ID einer Mission. 
Es war von der \brand{Mapbox}-\gls{API} her nicht möglich, einer Annotation direkt die Mission als Objekt mitzugeben. 
Bei einem Klick auf einen Marker wird die Funktion \inlinecode{onOpenAnnotation} aufgerufen.
Diese Funktion öffnet wiederum eine Scene, die es dem Benutzer erlaubt, die gewählte Mission zu lösen.
Die \brand{Mapbox}-\gls{API} ist auf der entsprechenden \brand{Github}-Seite\footnote{\url{https://github.com/mapbox/react-native-mapbox-gl}} dokumentiert. 


\subsection{OAuth}
Für die Authentifizierung über \brand{Google} wurde eine Open Source \gls{Library} evaluiert (Kapitel Evaluation \nameref{tb-evaluation-oauth}). 
Damit die App sich mit der \brand{Google}-\gls{API} verbinden kann, muss eine \inlinecode{google-services.json}-Datei auf der \brand{Google}-Webseite\footnote{\url{https://developers.google.com/cloud-messaging/android/client}} generiert werden. 
Diese Datei verknüpft die \brand{Google}-\brand{OAuth}-Konfiguration mit einem eindeutigen Identifier für die App (Bundle ID und Signing Key).

Nach dem Login des Benutzers erhält die App über diese \gls{Library} das Benutzer-Token von \brand{Google}.
Dieses Token wird dann von der App dem \kort{}-\gls{Backend} mitgeteilt und von dort aus bei \brand{Google} überprüft. 
Wie die Authentifizierung auf der \gls{Backend}-Seite weiter verläuft ist im Kapitel 10.4.1. der Bachelorarbeit von Jürg Hunziker und Stefan Oderbolz beschrieben.\cite{ba-kort-2012}