% Inhalt der CD
\chapter*{Inhalt der CD}
% Titel auch in Kopfzeile anzeigen
\markboth{Inhalt der CD}{Inhalt der CD}
% Kapitel in Inhaltsverzeichnis einfügen
\addcontentsline{toc}{chapter}{Inhalt der CD}

\begin{forest}
  for tree={
    font=\sffamily,
    text=white,
    text width=10cm,
    minimum height=0.75cm,
    if level=0
      {fill=root}
      {fill=folder},
    rounded corners=4pt,
    grow'=0,
    child anchor=west,
    parent anchor=south,
    anchor=west,
    calign=first,
    edge={root,rounded corners,line width=1pt},
    edge path={
      \noexpand\path [draw, \forestoption{edge}]
      (!u.south west) +(7.5pt,0) |- (.child anchor)\forestoption{edge label};
    },
    before typesetting nodes={
      if n=1
        {insert before={[,phantom]}}
        {}
    },
    fit=band,
    s sep=8pt,
    before computing xy={l=15pt},
  }
[/
  [Dokumentation/
    [Kort Reloaded - Marino Melchiori und Dominic Mülhaupt.pdf]
    [Anhänge/
      [Abstract.pdf]
      [Aufgabenstellung.pdf]
      [Eigenständigkeitserklärung.pdf]
      [Einverständniserklärung Publikation auf eprints.pdf]
      [Gamified Mobile App für die Verbesserung von OpenStreetMap - Jürg Hunziker und Stefan Oderbolz.pdf]
      [Poster.pdf]
      [Diagramme/]
      [Organisatorisches/]
      [Sitzungsprotokolle/]
    ]
  ]
  [Kort/
	[Sourcecode/
	  [kort-reloaded/
	    [android/]
	    [ios/]
	    [js/]
	  ]
	]
	[APK]
  ]
]
\end{forest}