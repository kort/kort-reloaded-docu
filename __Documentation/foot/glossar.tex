% Glossar
\newglossaryentry{NativeApp} {
	name = Native App,
	description = {Native Apps im engeren Sinn zeichnen sich dadurch aus, dass sie speziell an die Zielplattform angepasst und sehr leicht über ein herstellerspezifisches Online-Portal bezogen und installiert werden können.\cite{mobileapp}}
}

\newglossaryentry{API} {
	name = API,
	description = {Application Programming Interface, Schnittstelle für die Programmierung}
}

\newglossaryentry{OAuth} {
	name = OAuth,
	description = {OAuth ist ein offenes Protokoll, das eine standardisierte, sichere API-Autorisierung erlaubt\cite{oauth}}
}

\newglossaryentry{CRUD} {
	name = CRUD,
	description = {Das Akronym CRUD beschreibt die 4 Standardoperationen einer Datenbank: \textbf{C}reate, \textbf{R}ead, \textbf{U}pdate, \textbf{D}elete\cite{crud}}
}

\newglossaryentry{WebApp} {
	name = Web-App,
	description = {Der Begriff Web-App (von der englischen Kurzform für web application), bezeichnet im allgemeinen Sprachgebrauch Apps für mobile Endgeräte wie Smartphones und Tablet-Computer, die über einen in das Betriebssystem integrierten Browser aus dem Internet geladen und so ohne Installation auf dem mobilen Endgerät genutzt werden können\cite{webapp}}
}

\newglossaryentry{REST} {
	name = REST,
	description = {Representational State Transfer\cite{rest} ist ein Programmierparadigma, welches besagt, dass sich der Zustand einer Webapplikation als Ressource in Form einer URL beschreiben lässt. Auf eine solche Ressourcen können folgende Befehle angewendet werden: \inlinecode{GET}, \inlinecode{POST}, \inlinecode{PUT}, \inlinecode{PATCH}, \inlinecode{DELETE}, \inlinecode{HEAD} und \inlinecode{OPTIONS}.
	HTTP ist ein Protokoll welches REST implementiert}
}

\newglossaryentry{Git} {
	name = Git,
	description = {Git ist ein verteiltes Versionsverwaltungssystem für Dateien. Ursprünglich wurde es für die Entwicklung des Linux Kernels kreiert}
}

\newglossaryentry{CI} {
	name = Continuous Integration,
	description = {Unter dem Begriff \emph{Continuous Integration}\cite{cont-integration} beschreibt die Idee, dass Änderungen an einer Software schnell eingebracht werden sollen. Dazu zählt, dass diese in einem Versionsverwaltungs-Tool eingetragen und durch automatisierte Tests geprüft werden}
}

\newglossaryentry{POI} {
	name = POI,
	description = {Abkürzung für Point of Interest. Dies ist ein allgemeiner Begriff für einen Ort mit irgendeiner Bedeutung, sei es eine Schule, Kirche, Bushaltestelle oder sonst etwas von besonderem Interesse}
}

\newglossaryentry{Mapper} {
	name = Mapper,
	description = {Personen welche auf OpenStreetMap die Karten ergänzen und pflegen, nennen sich selbst \emph{Mapper}}
}

\newglossaryentry{App Store} {
	name = App Store,
	description = {Ein App Store ist eine Verkaufsplattform eines Betriebssystemherstellers für Smartphones, beispielsweise Google Play für Android, Apple App Store für iOS}
}

\newglossaryentry{Node} {
	name = Node,
	description = {Ein Node ist das grundlegendste Objekt in OpenStreetMap, es wird durch seine Attribute (genannt \glspl{Tag}) genauer bestimmt. Nodes können Teil eines \gls{Way} sein}
}

\newglossaryentry{Way} {
	name = Way,
	description = {Ein Weg oder eine Strasse wird in OpenStreetMap als \emph{Way} bezeichnet. Es handelt sich dabei um eine Serie von miteinander verbundenen Nodes}
}

\newglossaryentry{Relation} {
	name = Relation,
	description = {Eine Relation stellt eine Verbindung zwischen verschiedenen OSM-Objekten dar. Relationen werden meist dazu gebraucht, um übergeordnete Beziehungen darzustellen. Beispielsweise können alle Bushaltestellen einer Buslinie über eine Relation miteinander verknüpft sein}
}

\newglossaryentry{Tag} {
	name = Tag,
	description = {Ein Tag bezeichnet ein Attribut eines \glslink{Node}{Nodes}. Es besteht aus einem Namen und einem Wert. Ein \gls{Node} kann beliebig viele Tags beinhalten}
}

\newglossaryentry{OpenStreetMap} {
	name = OpenStreetMap,
	description = {OpenStreetMap ist ein freies Projekt, das für jeden frei nutzbare Geodaten sammelt (Open Data). Mit Hilfe dieser Daten können Weltkarten errechnet oder Spezialkarten abgeleitet werden sowie Navigation betrieben werden}
}

\newglossaryentry{Gamification} {
	name = Gamification,
	description = {Unter Gamification versteht man das Hinzufügen von Spiel-Elementen in einem nicht spieltypischen Kontext\cite{for-the-win}}
}

\newglossaryentry{MVC} {
	name = MVC,
	description = {Der englischsprachige Begriff \emph{model view controller} (MVC) ist ein Muster zur Strukturierung von Software-Entwicklung in die drei Einheiten Datenmodell (model), Präsentation (view) und Programmsteuerung (controller)\cite{patterns}}
}

\newglossaryentry{Crowdsourcing} {
	name = Crowdsourcing,
	description = {Unter Crowdsourcing versteht man die Auslagerung von  Auslagerung von traditionell internen Teilaufgaben an eine Menge von freiwilligen Usern im Internet\cite{crowdsourcing}}
}

\newglossaryentry{Routing} {
	name = Routing,
	description = {Routing beschreibt den Vorgang einen optimalen Weg zwischen zwei oder mehr Punkten zu finden. Eine so gefundene Strecke wird oft als \emph{Route} bezeichnet}
}

\newglossaryentry{Pair Programming}{
	name = Pair Programming,
	description = {Bei Pair Programming handelt es sich um eine Arbeitsweise, bei der zwei Programmierer an einem Rechner arbeiten. Dabei ist ein Programmierer damit beschäftigt, den Code zu schreiben, während der andere über die Problemstellungen nachdenkt, den geschriebenen Code kontrolliert sowie Probleme, die ihm dabei auffallen, sofort anspricht. \cite{pairprogramming}}
}

\newglossaryentry{Minimum Viable Product}{
	name = Minimum Viable Product,
	description = {Das Minimum Viable Product ist ein Produkt, welches gerade die Kernfunktionalität aufweist, die nötig ist, um es zu veröffentlichen. \cite{mvp}}
}

\newglossaryentry{Framework}{
	name = Framework,
	description = {Ein Framework ist noch kein fertiges Programm, sondern ein Rahmen, ein Gerüst, das der Programmierer verwenden kann um sein eigenes, persönliches Programm zu erstellen. \cite{framework}}
}

\newglossaryentry{GUI}{
	name = GUI,
	description = {GUI ist eine Apkürzung für graphical user interface und bedeutet Grafische  Benutzeroberfläche.\cite{gui}}
}

\newglossaryentry{Library}{
	name = Library,
	description = {Eine Library (Programmbibliothek) bezeichnet in der Programmierung eine Sammlung von Unterprogrammen, die Lösungswege anbieten. 
	Bibliotheken laufen im Unterschied zu Programmen nicht eigenständig, sondern sie enthalten Hilfsmodule, die angefordert werden können.\cite{library}}
	plural={Libraries},
  	descriptionplural = {Eine Library (Programmbibliothek) bezeichnet in der Programmierung eine Sammlung von Unterprogrammen, die Lösungswege anbieten. 
  	Bibliotheken laufen im Unterschied zu Programmen nicht eigenständig, sondern sie enthalten Hilfsmodule, die angefordert werden können.\cite{library}},
}

\newglossaryentry{DOM}{
	name = DOM,
	description = {DOM bedeutet Document Object Model und ist eine Spezifikation für den Zugriff auf HTML-Dokumente. 
	Anhand der DOM kann ein Computerprogramm den Inhalt der HTML-Elemente dynamisch verändern. 
	Für einen Webentwickler wäre das der HTML-Code.\cite{dom}}
}

\newglossaryentry{Virtual DOM}{
	name = Virtual DOM,
	description = {Die virtual DOM ist eine Abstraktion der DOM. 
	Es ist eine lokale Kopie der DOM. \cite{virtual-dom}}
}

\newglossaryentry{JSX}{
	name = JSX,
	description = {JSX ist eine \brand{JavaScript}-Syntax-Erweiterung, die ähnlich wie XML aussieht. \cite{jsx}}
}

\newglossaryentry{Backend}{
	name = Backend,
	description = Als Backend wird das auf dem Server laufenden Programm bezeichnet.
}

\newglossaryentry{Frontend}{
	name = Frontend,
	description = Als Frontend wird das beim Client laufende Programm bezeichnet.
}


\newglossaryentry{DTO}{
	name = DTO,
	description = Ein Datentransferobjekt ist ein Objekt, das nur zur Übertragung von Daten oder Informationen dient und keine logischen Komponenten enthält.
}
