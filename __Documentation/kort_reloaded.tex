% Glossar (muss am Beginn eingefügt werden damit alle Referenzen funktionieren)
\input{formatting/preamble}
\begin{document}

% ToDo Liste
% Abstract aktualisieren

% -----------------------------------------
% HEAD
% -----------------------------------------
% Titelseite
\title{Kort Reloaded – A Gamified App for Collecting OpenStreetMap Data}
\author{Marino Melchiori
		\and
		Dominic Mülhaupt}
%\date{10. März 2016}

\begin{titlepage}

% Logos
\begin{figure}[H]
\subfigure{\includegraphics[width=200px]{images/titelseite/logo-hsr}}
\hfill
\raisebox{37px}{\subfigure{\includegraphics[width=125px]{images/titelseite/logo-liip}}}
\end{figure}

\vspace{2cm}

\begin{center}
{ \Large
	% Titel
	\textbf{Kort Reloaded – A Gamified App for Collecting OpenStreetMap Data}
	\vspace{1cm}

	% Arbeitstyp / Schule
	\textbf{Bachelorarbeit}
	\vspace{1cm}

	Abteilung Informatik \\[0.2cm]
	HSR Hochschule für Technik Rapperswil
	\vspace{1cm}

	% Semester
	Frühjahrssemester 2016
}
\end{center}
\vspace{1cm}


\begin{table}[H] 
\centering 
\begin{tabular}{p{0.19\twocelltabwidth}p{0.4\twocelltabwidth}}
Autoren: & \textbf{Marino Melchiori} \newline
\textbf{Dominic Mülhaupt} \\ 
Betreuer: & \textbf{Prof. Stefan Keller} \\ 
Projektpartner: & \textbf{Jürg Hunziker} \newline
\textbf{Stefan Oderbolz} \newline
Liip AG\\ 
Experte: & \textbf{} \\ 
Datum: & \textbf{} \\ 
\end{tabular}
\end{table}

\end{titlepage}

% Seitennummerierung mit roemischen Zeichen
\pagenumbering{roman}

% Impressum
\chapter*{Impressum}
% Titel auch in Kopfzeile anzeigen
\markboth{Impressum}{Impressum}

% Impressum
\begin{table}[H] 
\centering 
\begin{tabular}{|p{0.35\twocelltabwidth}|p{0.65\twocelltabwidth}|}
\hline 
\textbf{Autoren:} & Marino Melchiori (\url{mmelchio@hsr.ch}) \newline
Dominic Mülhaupt (\url{dmuelhau@hsr.ch}) \\ 
\hline 
\textbf{Dokument erstellt:} & 10.03.2016 \\ 
\hline 
\textbf{Letzte Aktualisierung:} & 28.06.2016 \\ 
\hline 
\end{tabular}
\end{table}

Dieses Dokument wurde mit \LaTeX{} erstellt.

% Erklaerung
%\chapter*{Eigenständigkeitserklärung}
% Titel auch in Kopfzeile anzeigen
\markboth{Eigenständigkeitserklärung}{ErkläEigenständigkeitserklärungrung}

Ich erkläre hiermit,
\begin{itemize}
\item dass ich die vorliegende Arbeit selber und ohne fremde Hilfe durchgeführt habe, ausser derjenigen, welche explizit in der Aufgabenstellung erwähnt ist oder mit dem Betreuer schriftlich vereinbart wurde,
\item dass ich sämtliche verwendeten Quellen erwähnt und gemäss gängigen wissenschaftlichen Zitierregeln korrekt angegeben habe.
\end{itemize}

\vspace{3cm}

\begin{tabular}{p{0.5\twocelltabwidth}p{0.5\twocelltabwidth}}
Rapperswil-Jona, 16. Juni 2016 & \\ 
\end{tabular} 

\vspace{0.5cm}

\begin{tabular}{p{0.5\twocelltabwidth}p{0.5\twocelltabwidth}}
Marino Melchiori, Unterschrift: & Dominic Mülhaupt, Unterschrift: \\ 
\end{tabular} \\ 

% Abstract
\chapter*{Abstract}
\thispagestyle{scrheadings}
% Titel auch in Kopfzeile anzeigen
\markboth{Abstract}{Abstract}

\brand{OpenStreetMap} ist ein freies Projekt das von Benutzern aus der ganzen Welt unterstützt wird. 
Wege, Gebäude und viele andere geografische Daten werden weltweit in einer Datenbank erfasst und gepflegt. 
\brand{OpenStreetMap} kann von jedem bearbeitet werden, besteht aus einer grossen Community und setzt auf lokales Wissen der «Mapper». 
Aus diesem Grund ist es nicht ausgeschlossen, dass fehlerhafte oder unvollständige Daten enthalten
sind. 
Zur Korrektur der Daten gibt es viele verschiedene Tools, die von Experten genutzt werden können. 
Um eine breitere Masse anzusprechen, entstand 2012 die Web-App «\kort{}» im Rahmen einer Bachelorarbeit. 
Mit \kort{} kann der Benutzer Aufträge lösen, die zur Verbesserung der Daten in \brand{OpenStreetMap} beitragen. 
Auf einer Kartenansicht werden die Aufträge, welche sich im Umfeld des Benutzers befinden, dargestellt. 
Für das Eintragen einer Lösung wird man mit Punkten (sogenannten «Koins») belohnt und kann
so in der Rangliste aufsteigen. 
Da sich HTML5 weiterentwickelt hat, muss die \gls{WebApp} abgelöst werden.

In dieser Arbeit wurde \kort{} als natives Mobile App für \brand{Android} und \brand{iOS} komplett neu entwickelt. 
Das bestehende \gls{Backend} wurde dabei praktisch unverändert weitergenutzt. 
Die App basiert auf dem \brand{React Native} Framework von \brand{Facebook}. 
\brand{React Native} ist eine moderne, noch junge Technologie, welche es ermöglicht, mit \brand{JavaScript} Apps für \brand{Android} und \brand{iOS} zu entwickeln. 
Dabei wird \brand{JavaScript}-Code in Komponenten der jeweiligen Smartphone-Plattform übersetzt, was dem Benutzer die Erfahrung einer nativen App bietet.

Es konnten wichtige Erfahrungen mit \brand{React Native} gesammelt werden. 
Dabei sind eine \brand{Android}- und eine \brand{iOS}-App entstanden. 
Die \brand{Android}-App befindet sich noch in der Beta-Phase und wird nach Abschluss dieser Arbeit im \brand{Google Play} Store erwartet. 
Die \brand{iOS}-App ist noch in der Testphase und wird später im \brand{Apple} App Store veröffentlicht. 
Weitere Informationen: www.kort.ch. 


% Dank
\chapter*{Dank}
\thispagestyle{scrheadings}
% Titel auch in Kopfzeile anzeigen
\markboth{Dank}{Dank}

Für die Betreuung während des ganzen Projektes möchten wir uns besonders bei Herr Prof. Stefan
Keller bedanken.
Bei Fragen zu \brand{OpenStreetMap} konnte er uns während der gesamten Laufzeit der Arbeit unterstützen und er gab uns bei Problemen immer Inspirationen für eine Lösung. 
Ein Dank geht auch an Jürg Hunziker und Stefan Oderbolz, für die Unterstützung während des ganzen Projektes.
Sie gaben uns Tipps zum Vorgehen und haben für uns grössere und wichtige Backend-Änderungen vorgenommen. 
Herr Prof. Stefan Keller, Jürg Hunziker und Stefan Oderbolz liessen uns freien Spielraum für die Suche nach einer optimalen Lösung.

\section*{Ausgangslage}
Unsere Aufgabe war es die nicht mehr voll funktionsfähige \gls{WebApp} mit einer native Android App zu ersetzen.
Das erste Ziel war die Einbettung der \brand{OpenStreetMap}-Karte.
Für eine editierbare Karte, die auch Bilder enthält, gibt es für \brand{React-Native} eine experimentelle MapBox Library. 
%ToDo: Verweis richtig verlinken
Die Ziele sind in der Aufgabenstellung genauer definiert.
Verwenden konnten wir die ganze Infrastruktur der \kort{} Bachelorarbeit.



% Aufgabenstellung
\chapter*{Aufgabenstellung}
\thispagestyle{scrheadings}
% Titel auch in Kopfzeile anzeigen
\markboth{Aufgabenstellung}{Aufgabenstellung}

Der \kort{}-Client soll neu als native \brand{Android} App zur Verfügung stehen. Das bedingt ein kompletter Rewrite des aktuellen \brand{JavaScript}-Codes (aktuell \brand{Sencha Touch}) mit dem \gls{Framework} \brand{React Native}. Es soll ein Erfahrungsbericht zu \brand{React Native} erstellt werden.

Ziele:
\begin{itemize}
	\item Gleiche Funktionalität mit neuem Framework, damit die mobile App mit den neusten Technologien arbeitet und künftig besser wartbar ist
	\item Neue Erkenntnisse zu einem aktuellen Framework zur Realisierung von native mobile
Apps \brand{React Native}
	\item Getestete Software-Entwicklungsumgebung
\end{itemize}

\section*{Deliverables}
Mindestens...
\begin{itemize}
	\item Neuer \kort{}-Client als native \brand{Android}-App
	\item Ablösung des Validierungsmechanismus
	\item Erfahrungsbericht mit Hinweisen zu Tutorials zu \brand{React Native}
	\item Erweiterte Software-Entwicklungsumgebung
	\item Die vom Studiengang geforderten Lieferobjekte: Dokumentation, Management Summary, Abstract, Poster, Präsentation mit Stellwand, Zwischenpräsentation
\end{itemize}

Erweitert...
\begin{itemize}
	\item \kort{}-Client als native \brand{iOS}-App
	\item neue Funktion: Promotions anzeigen
	\item Kurzvideo
\end{itemize}

Die definitive Aufgabenstellung, Lieferobjekte und das Vorgehen werden am Kickoff (erste Semesterwoche) zusammen mit dem Industriepartner festgelegt. Die gemeinsam besprochene Aufgabenstellung wird ca. zwei Wochen nach Semesterbeginn aktualisiert.

\section*{Vorgaben/Rahmenbedingungen}
\begin{itemize}
	\item Die ursprüngliche \kort{}-App ist clientseitig in \brand{HTML5} und \brand{JavaScript} geschrieben.
	\item Serverseitig ist u.a. \brand{PHP} und \brand{PostgreSQL} mit der \brand{PostGIS}-Erweiterung vorhanden.
	\item Als Technologien stehen \brand{React} und \brand{React Native} im Vordergrund.
	\item Es wird genügend Zeit für die Einarbeitung in die Themengebiete einberechnet.
	\item Die Software soll Open Source sein.
	\item Die SW-Engineering-Methode und Meilensteine werden mit dem Betreuer vereinbart.
	\item Sourcecode und Software-Dokumentation sind Englisch (inkl. Installation, keine Benutzerdokumentation, höchstens eine Online-Kurzhilfe).
	\item Die Software-Benutzerschnittstelle ist mind. Deutsch und Englisch.
	\item Die Projekt-Dokumentation und -Präsentation sind auf Deutsch. 
	\item Der Source Code, die Code-Kommentare und die Versionsverwaltung sind in Englisch.
	\item Die Nutzungsrechte an der Arbeit bleiben bei den Autoren und gehen auch an die \brand{HSR} und den Betreuer über. Die Softwarelizenz ist „MIT“.
	\item Ein Video gemäss den Vorgaben des Studiengangs (kann ggf. nach dem Dokumentations-Abgabetermin abgegeben werden).
	\item Ansonsten gelten die Rahmenbedingungen, Vorgaben und Termine des Studiengangs Informatik bzw. der \brand{HSR}.
\end{itemize}

\section*{Inhalt der Dokumentation}
\begin{itemize}
	\item Die fertige Arbeit muss folgende Inhalte haben:
	\begin{enumerate}
		\item Abstract, Aufgabenstellung
		\item Technischer Bericht
		\item Projektdokumentation
		\item Anhänge (Literaturverzeichnis, Glossar, CD-Inhalt)
	\end{enumerate}
	\item Die Abgabe ist so zu gliedern, dass die obigen Inhalte klar erkenntlich und auffindbar sind.
	\item Zitate sind zu kennzeichnen, die Quelle ist anzugeben.
	\item Verwendete Dokumente und Literatur sind in einem Literaturverzeichnis aufzuführen.
	\item Dokumentation des Projektverlaufes, Planung etc.
	\item Weitere Dokumente (z.B. Kurzbeschreibung, Poster) gemäss \url{www.hsr.ch} und Absprache mit dem Betreuer.
\end{itemize}

\section*{Form der Dokumentation}
Bericht, Dokumente und Quellen der erstellten Software gemäss Vorgaben des Studiengangs Informatik der \brand{HSR} sowie Absprache mit dem Betreuer.

\section*{Bewertungsschema}
Es gelten die üblichen Regelungen zum Ablauf und zur Bewertung der Arbeit (6 Aspekte) des Studiengangs Informatik der \brand{HSR} jedoch mit besonderem Gewicht auf moderne Softwareentwicklung (Tests, Continuous Integration, einfach installierbar, funktionsfähig).

\section*{Beteiligte}
\subsection*{Diplomanden}
Marino Melchiori und Dominic Mülhaupt

\subsection*{Industriepartner}
Jürg Hunziker und Stefan Oderbolz, Liip AG Zürich

\subsection*{Betreuung HSR}
Verantwortlicher Dozent: Prof. Stefan Keller

% Inhaltsverzeichnis
\tableofcontents

% -----------------------------------------
% BODY
% -----------------------------------------
% Neue Seite beginnen (um Seitennummerierung zurückzusetzen)
\cleardoublepage

% Seitennummerierung mit arabischen Zeichen
\pagenumbering{arabic}

% Einführung
\part{Technischer Bericht}
%Einfuehrung
\chapter{Einführung}
\label{tb-einfuehrung}

\section{Problemstellung, Vision}
\kort{} ist eine \gls{WebApp}, entwickelt mit dem \brand{Sencha Touch 2} \gls{Framework}.
Diese Technologie ist jetzt aber veraltet und funktioniert auf neuen Browsern nicht mehr sinngemäss.
Zum Beispiel ist das Scrollen blockiert.
So ist es auf einem mobilen Gerät unmöglich, das Login-Feld überhaupt auszufüllen.

Aus diesem Grund entstand die Idee, die \kort{} \gls{WebApp} mit einer anderen Technologie, neu zu schreiben.
Dazu bot sich \brand{React-Native} an. 
Diese neue Technologie ermöglicht es uns, native \brand{iOS} und \brand{Android} Apps mit \brand{JavaScript} zu erstellen. 

\section{Ziele}


\section{Rahmenbedingungen, Umfeld, Definitionen, Abgrenzungen}


\section{Vorgehen, Aufbau der Arbeit}



%Stand der Technik
\chapter{Stand der Technik}
\label{tb-stand-der-technik}
Es gibt eine grosse Anzahl von Projekten, mit dem Ziel \brand{OpenStreetMap} durch \gls{Crowdsourcing} zu verbessern.
Es werden sowohl Editoren für erfahrene Benutzer, als auch Tools, die auf das finden und Beheben von Fehlern spezialisiert sind, angeboten.
Zum Beispiel gibt es JOSM\footnote{\url{https://josm.openstreetmap.de/}} (\brand{Java OpenStreetMap Editor}) als Desktop Client für \brand{Windows}, \brand{Mac OS X} und \brand{Linux}.
Sogar Geometrieobjekte sind editierbar.
Daneben gibt es auch Dienste, welche Fehlerdaten sammeln und öffentlich anbieten.
Ein bekanntes Beispiel ist \brand{KeepRight}\footnote{\url{http://www.keepright.at/}}.
Es werden über 50 Fehlertypen angeboten.
Mit diesen Werkzeugen können nun Daten korrigiert werden.
Weitere Editoren sind auf dem \brand{OpenStreetMap}-Wiki\footnote{\url{http://wiki.openstreetmap.org/wiki/Editors#Choice_of_editors}} zu finden.

Die Zielgruppe dieser Werkzeuge liegt bei Benutzern mit dem Interesse, die \brand{OpenStreetMap}-Daten zu verbessern.
Damit \gls{Crowdsourcing} genutzt werden kann sollte die Zielgruppe erweitert werden, um möglichst viele Benutzer anzusprechen.
Ein sehr gutes Beispiel dazu ist MapRoulette\footnote{\url{http://wiki.openstreetmap.org/wiki/MapRoulette}}.
Dem Benutzer werden Challenges präsentiert, die zum Beispiel ein Satellitenbild von einem angeblichen Fussballfeld zeigen. 
Nun entscheidet der Benutzer in dem er das Feld entsprechend markiert und so die Challenge löst.

\gls{Gamification} bietet sich also sehr gut dazu an.
Die Spiel-Elemente halten den Spieler motiviert und binden ihn an das Spiel.
Einige Projekte\footnote{\url{http://wiki.openstreetmap.org/wiki/Games#Gamification_of_map_contributions}}, die durch \gls{Gamification} beitragen \brand{OpenStreetMap} zu verbessern, sind bereits entstanden.

Ähnliche Projekte, die mit \brand{React-Native} geschrieben sind, gibt es bis jetzt keine.


\section{React Native}


\section{Bestehende Lösungsansätze und Normen}
Da wir die \kort{}-\gls{WebApp} neu schreiben, übernehmen wir die dort evaluierten Konzepte.

%Hier Abgrenzung? -> Fortsetzungsarbeit

\section{OSM OAuth}


\section{Kort Schnittstelle}



%Bewertung
\chapter{Bewertung}
\label{tb-bewertung}

\section{Kriterien}


\section{Schlussfolgerungen}


%Umsetzungskonzept
\chapter{Umsetzungskonzept}
\label{tb-umsetzungskonzept}
Am 8.3.2016 haben wir an der HSR ein \brand{React Native}-Meetup durchgeführt um die Projekte von den Teilnehmern kennenzulernen. 
Es ging darum, auszutauschen, was für die Software-Entwicklungsumgebung benötigt wird, wie \brand{React Native} am besten erlernt wird und welche Lösungskonzepte es für die Darstellung der Karte gibt.
Hinweise zur Entwicklungsumgebung und den verwendeten Werkzeugen sind im Kapitel \nameref{pm-projektmanagement} beschrieben.
%ToDo: React Native Erfahrungsbericht referenzieren.
Für die Darstellung der Karte mit \brand{React Native} haben wir diese Möglichkeiten ausfindig gemacht:

%Eventuell in Tabelle darstellen:
\begin{itemize}
    \item Extended React Native Map Komponente (empfohlen von Facebook): https://github.com/lelandrichardson/react-native-maps
    \begin{itemize}
    	\item Haken: Native Map API von Apple iOS und Android SDK nutzen (fest mit Apple/Google Maps verbunden)
	\end{itemize}
	
    \item React Native "Map" Komponente nutzen: http://browniefed.com/blog/2015/05/30/create-a-map-with-react-art/
    \begin{itemize}
    	\item Haken: "Pattern Fill" issue (Raster Tiles) - Es lassen sich keine Bilder, oder Missions-Icons, auf der Karte darstellen.
	\end{itemize}     

    \item MapBox GL Library mit Android/iOS SDK https://libraries.io/npm/react-native-mapbox-gl, eventuell mit Offline-Kacheln zu füttern http://osm2vectortiles.org/downloads/ (Vector Tiles)
    \item Portierung von Leaflet nach React: https://github.com/PaulLeCam/react-leaflet
    \item Raster Tiles "von Hand" anzeigen
\end{itemize}

Die Native Map APIs von \brand{Google} und \brand{Apple} kamen für uns nicht in Frage.
Wir möchten mit unserer App \brand{OpenStreetMap} Daten verbessern und setzen somit aus moralischen Aspekten auch auf die entsprechende Karte.
%Google Lizenz-Grund nicht erwähnt, da eventuell referenziert werden müsste.
Bei der \brand{React Native} Map-Komponente gab es keine Möglichkeit Bilder auf der Karte darzustellen.
Die Raster-Tiles "von Hand" anzuzeigen wäre schlicht zu aufwendig. 
Es liesse sich auch keine schöne Map designen.

Somit sprang uns als erstes die Portierung von Leaflet für \brand{React} ins Auge. 
Nach dem betrachten vom Code fiel uns aber auf, dass diese Variante nur möglich ist, wenn die Karte in einer WebView-Komponente von \brand{React Native} dargestellt wird.
Eine Lösung mit der WebView war zu Aufwendig um richtig getestet zu werden. 
Wir waren uns nicht sicher, ob dann wirklich alles sauber funktioniert.

Als letzte Möglichkeit blieb die MapBox GL Library.
Diese hat uns nach dem Testen überzeugt.
Der einzige Haken wäre das Pricing, welches bei einer sehr hohen Benutzeranzahl beachtet werden muss.


%Resultate
\chapter{Resultate}
\label{tb-resultate}



%ToDo: Screenshots

\section{Zielerreichung}
Erreichte Ziele:

\begin{itemize}
	\item Finale \brand{Android}-App mit gleicher Funktionalität, wie die \gls{WebApp}
	\item \brand{iOS}-App mit gleicher Funktionalität, wie die \gls{WebApp}
	\item Validationsmechanismus abgelöst
	\item Erfahrungsbericht zu \brand{React-Native}
	\item Internationalisierung umgesetzt
\end{itemize}

\section{Ausblick}
Nächste geplante Arbeiten:

\begin{itemize}
	\item \brand{OpenStreetMap}-Login
	\item Finale\brand{iOS}-App
	\item \brand{Apple App Store} und \brand{Google Play Store} Veröffentlichung
	\item Promotions-Funktion
\end{itemize}

Für die Zukunft gibt es bereits viele weitere Ideen. 
Eine Liste wurde im Kapitel \hyperref[pd-resultate-weiterentwicklung]{Weiterentwicklung} erstellt.

\section{Persönliche Berichte}




% Neue Seite beginnen
\cleardoublepage

% Projektdokumentation
\part{Projektdokumentation}
% Anforderungsspezifikation
\chapter{Anforderungsspezifikation}
\label{pd-anforderungsspezifikation}

\section{Anforderungen an die Arbeit} 
Die \hyperref[pm-rollen]{Autoren} hatten im Vorfeld der Arbeit wenige Kenntnisse über Webtechnologien und insbesondere gar keine Erfahrung mit \brand{JavaScript}.
Insofern war der machbare Umfang des Projekts schwer absehbar.
In Abstimmung mit dem \hyperref[pm-rollen]{Betreuer} und dem \hyperref[pm-rollen]{Projektpartner} wurde deshalb festgelegt, dass der Fokus auf dem Frontend liegt, so dass man sich nicht auch noch in die Technologien des Backends einarbeiten muss.

In der Anforderungsanalyse konnten viele Aufgaben erkannt werden, die in diesem Projekt bearbeitet werden könnten. Im Rahmen dieser Arbeit ist nur ein Bruchteil davon umsetzbar – zum einen aus zeitlichen Gründen, zum anderen, weil Anpassungen am Backend nötig wären.
All diese Anforderungen wurden in \emph{Muss}, \emph{Soll}, \emph{Kann}, \emph{zukünftige Arbeiten} und \emph{abgewiesene Arbeiten} unterteilt.

\subsection{Muss}
\begin{itemize}
	\item neuer Kort-Client als native \brand{Android}-App
	\item Ablösung des Validationsmechanismus
	\item Erfahrungsbericht mit Hinweisen zu Tutorials zu React Native
	\item die vom Studiengang geforderten Lieferobjekte: Dokumentation, Management Summary, Abstract, Poster, Präsentation mit Stellwand, Zwischenpräsentation
\end{itemize}

\subsection{Soll}
\begin{itemize}
	\item Kort-Client als native \brand{iOS}-App
\end{itemize}

\subsection{Kann}
\begin{itemize}
	\item neue Funktion: Promotions anzeigen
	\item Kurzvideo (zur Instruktion und Promotion)
\end{itemize}


\section{Use Cases}


\section{System-Sequenzdiagramme}




% Technologien
\chapter{Technologien}
\label{pd-technologien}

\section{React} 


\section{React Native}



% Design
\chapter{Design}
\label{pd-design}

% Datenmodell
\label{pd-datenmodell}
\section{Datenmodell}
Die Model Logik ist weitestgehend im \gls{Backend} implementiert, welches die Daten über eine \gls{REST}-Schnittstelle zur Verfügung stellt.
Die über die Schnittstelle zur Verfügung gestellten Informationen sind also Ausgangspunkt für die von uns verwendbaren Domain Klassen.
Es waren kleinere Anpassungen nötig, etwa um den neuen Validierungsmechanismus zu unterstützen.
Da die empfangenen Daten fast ausschliesslich als Repräsentation von Informationen dienen und keine eigene Logik implementieren, wurden sie als \glslink{DTO}{Datentransferobjekte} umgesetzt.\newline
Das folgende Datenmodell repräsentiert die \glslink{DTO}{Datentransferobjekte} und ist nicht als semantisches Modell zu verstehen.\newline
\begin{figure}[H]
 	\centering
 	\includegraphics[width=\textwidth]{images/projektdokumentation/Datenmodell.png}
 	\caption{Modellierung der DTOs}
 	\label{image-data-model}
\end{figure}


% Design
\label{pd-architektur}
\section{Architektur}
Für unsere Applikationsarchitektur haben wir das \brand{Flux}\footnote{\url{https://facebook.github.io/flux/}} Architekturpattern von \brand{Facebook} eingesetzt.\newline
Die folgenden Erklärungen wurden zu grossen Teilen aus der \brand{Flux} Dokumentation\cite{flux-docs-overview} abgeleitet.
Zu den nachfolgend beschriebenen Konzepten der \brand{Flux}-Architektur mussten für unsere Realisierung keine Anpassungen gemacht werden.
Die Beschreibungen entsprechen also unserer Implementation.\newline
\brand{Flux} ist eine Architektur, die in \gls{Frontend}-Applikationen eingesetzt wird.
Ein unidirektionaler Datenfluss ist das Grundprinzip, auf welchem \brand{Flux} aufbaut, und in welchem es sich von \gls{MVC}-Architekturen unterscheidet.
Das Pattern beschreibt folgende vier Hauptbestandteile: die Stores, die Views (oder Controller-Views), die Actions und der Dispatcher.
\begin{figure}[H]
 	\centering
 	\includegraphics[width=\textwidth]{images/projektdokumentation/flux-uebersicht-weiss.png}
 	\caption{Idee der \brand{Flux}-Architektur}
 	\label{image-flux-overview-simple}
\end{figure}
\subsection{Stores}
\label{pd-flux-stores}
Stores enthalten den Applikationszustand und die Applikationslogik.
Verglichen mit dem \gls{MVC}-Pattern entsprechen sie am ehesten dem Model.
Sie unterscheiden sich aber insofern, dass sie nicht unbedingt einzelne Modellklassen repräsentieren sollen, sondern domänenspezifische Aufgaben übernehmen.
Da \kort{} keine besonders komplexe Domäne enthält, ist diese Unterscheidung allerdings nicht von grosser Bedeutung.
Ein Beispiel für diese Einteilung nach Domäne und nicht nach Modell findet man in der Aufteilung von \inlinecode{UserStore} und \inlinecode{AuthenticationStore}\newline
Stores sind als \glslink{Singleton}{Singletons} umgesetzt.
Sie registrieren sich beim Dispatcher um Updates zu erhalten und ihren Zustand entsprechend anzupassen.
Views wiederum können sich bei den Stores registrieren um neue Informationen zu erhalten, können die Stores aber nicht direkt anweisen, sich zu aktualisieren.

\subsection{Views}
\label{pd-flux-views}
Der Applikationsanwender kommuniziert seine Absichten über die View.
Deshalb gilt die View in \brand{Flux} als Auslöser für neue Aktionen.
In \brand{React} kann zwischen \emph{Views} und \emph{Controller-Views} unterschieden werden.\newline
\emph{Controller-Views} finden sich an der Spitze der View-Hierarchie.
Sie warten auf Updates der Stores und reichen die Daten entlang der Kette ihrer untergeordneten \emph{Views} weiter.\newline
Diese untergeordneten \emph{Views} wiederum reagieren auf Zustandsänderungen indem ihre \newline\inlinecode{render()} Methode neu aufgerufen wird.
Somit bleiben \emph{View} Komponenten modular austauschbar, da sie unabhängig von ihrem Kontext eingesetzt werden können.

\subsection{Actions}
\label{pd-flux-actions}
Actions sind Helfermethoden, welche im Dispatcher ein Ereignis und somit in den Stores ein Update auslösen.
Sie werden ausschliesslich durch Views ausgelöst, da diese für den Kontrollmechanismus der Applikation zuständig sind.
Ausnahmen wären hier denkbar, waren aber nicht nötig.
Beispielsweise könnte der \inlinecode{LocationStore} eine Action auslösen wenn er eine neue Position erkannt hat oder der Server wenn ein Update an die Applikation gesendet werden soll.\newline
Ausserdem sollten Daten, wenn diese für ein Update nötig sind, bereits durch die Actions an den Dispatcher mitgeliefert werden.
Aufrufe an die \gls{REST}-Schnittstelle des \glslink{Backend}{Backends} werden also über die Actions ausgelöst.
Somit kann sichergestellt werden, dass verschiedene Stores, welche auf dieselbe Action reagieren, denselben \gls{API}-Aufruf mehrmals ausführen.
In Abbildung \ref{image-flux-overview-detailed} ist dies ersichtlich.

\subsection{Dispatcher}
\label{pd-flux-dispatcher}
Der Dispatcher ist der zentrale Knotenpunkt, durch den der gesamte Datenfluss der Applikation koordiniert wird.
Seine einzige Aufgabe ist die Verteilung der Actions an die Stores.
Er enthält also keine intelligente Logik, sondern ist grundsätzlich ein Register von Callbacks.\newline
\begin{figure}[H]
 	\centering
 	\includegraphics[width=\textwidth]{images/projektdokumentation/flux-diagram.png}
 	\caption{Vollständiges \brand{Flux}-Diagramm}
 	\label{image-flux-overview-detailed}
\end{figure}


\section{Klassenkonzepte}


\section{Sequenzdiagramme}


\section{UI-Design}


% Entwicklungsumgebung
\chapter{Entwicklungsumgebung}
\label{pd-entwicklungsumgebung}
\section{IDE}
Die Funktionalitäten und Features der App wurden alle mit \brand{Atom} implementiert. 
Für das Debugging waren die native \glslink{IDE}{IDEs} \brand{Android Studio} und \brand{Xcode} aber besser geeignet. 
Das Debuggen in der \brand{Atom}-\gls{IDE} oder im \brand{Google Chrome} Browser war fehlerbehaftet. 
Die native \glslink{IDE}{IDEs} wurden des Weiteren verwendet um projektspezifische Konfigurationen -- etwa das Einbinden von Libraries oder Build-Konfigurationen -- vorzunehmen.
Ansonsten wurden die native \glslink{IDE}{IDEs} nur für das Einfügen von statischen Bildern mit passenden Auflösungen für entsprechende Displays genutzt.

\section{Continuous Integration}
Als Versionsverwaltungssystem wurde \brand{git}\footnote{\url{https://git-scm.com/}} zusammen mit dem Online-Dienst \brand{GitHub}\footnote{\url{https://github.com}} verwendet: \url{https://github.com/kort/kort-reloaded}.
Dabei haben wir folgendes Branching Model eingesetzt: Der master Branch wird nur für Releases verwendet.
Für die Entwicklung wurde der develop Branch genutzt, wobei jedes Feature und jeder Fix eines Bugs einen eigenen Branch erhielt, welcher nach Fertigstellung wieder in den develop Branch gemerged wurde.\newline
\newline
Für die \gls{CI} (CI) nutzen wir den freien Dienst von \brand{Travis-CI}\footnote{\url{https://travis-ci.org/}}.
Dieses Setup lässt sich bequem in Verbindung mit \brand{GitHub} nutzen. 
Dabei wird bei jeder Neuerung auf dem master und develop Branch über \brand{Travis-CI} ein neuer Build erstellt.
Bei jedem Build werden die Tests durchlaufen und der Code auf die Einhaltung der \nameref{pd-entwicklungsumgebung-cr} überprüft.\newline
Die Konfigurationsdatei für \brand{Travis-CI} (\inlinecode{.travis.yml}) befindet sich im \brand{GitHub}-Repository von \kort{}.

\section{Projektmanagement-Tool}
Als Projektmanagement-Tool wurde \brand{Redmine} verwendet. 
Weiterführende Links zum Redmine-Projekt, das für die Ticketerfassung verwendet wurde, sind im Kapitel \hyperref[pm-projektmanagement]{Projektmanagement} dokumentiert.

\section{Testing}
\label{pd-entwicklungsumgebung-testing}
Fürs Testing wurden \glslink{Unit Test}{Unit Tests}, \glslink{Integration Test}{Integration Tests} und \glslink{Funktionaler Test}{Funktionalen Tests} evaluiert.
Letztlich konnten -- zum Zeitpunkt der Abgabe -- lediglich die Unit Tests umgesetzt werden.
Integration Tests wären für die \inlinecode{data} Klassen wünschenswert gewesen, konnten aber aus Zeitgründen nicht umgesetzt werden.
Automated UI Tests sind mit \brand{React Native} möglich\footnote{Einen guten Überblick bietet diese Seite: \url{http://testdroid.com/tech/testing-react-native-apps-on-android-and-ios}}, sind aber sehr aufwendig einzurichten.
Für die Unit Tests wurde \brand{Jest}\footnote{\url{https://facebook.github.io/jest/}} eingesetzt. 
\brand{Jest} ist ein \brand{JavaScript} Testing-\gls{Framework} und wird zum Beispiel von \brand{Facebook} zum Testen von \brand{React}-Applikationen verwendet.\newline
Eine besondere Eigenschaft von \brand{Jest} ist, dass standardmässig für alle Module automatisch \glslink{Mock}{Mocks} bereitgestellt werden.
Somit wird verhindert, dass aus Versehen das Verhalten anderer Module getestet wird.
Leider gab es in diesem Zusammenhang öfters Probleme beim Mocken vom \brand{Node} Modulen.

\section{Code-Richtlinien}
\label{pd-entwicklungsumgebung-cr}
Um Code-Richtlinien festzulegen und deren Einhaltung zu prüfen, wurde \brand{ESLint}\footnote{\url{http://eslint.org/}} eingesetzt.
Dadurch wird die die Lesbarkeit und Wartbarkeit des Codes erhöht.
Die Konfiguration von \brand{ESLint} stammt von \brand{Airbnb}\footnote{\url{https://github.com/airbnb/javascript/tree/master/packages/eslint-config-airbnb}}.
Darüber hinaus wurden Plugins für \brand{React}\footnote{\url{https://github.com/yannickcr/eslint-plugin-react}} und \brand{React Native}\footnote{\url{https://github.com/Intellicode/eslint-plugin-react-native}} eingesetzt.\newline
Die Konfiguration findet sich in der Datei \inlinecode{.eslintrc.json}.

% Implementation
\chapter{Implementation}
\label{pd-implementation}
% Hinweis Einrichtung

\section{Kort Backend}
% Verweis auf BA


\section{Libraries}
Damit die entsprechenden \glslink{Library}{Libraries} genutzt werden können, mussten sie mit dem Node-Package-Manager (npm) heruntergeladen werden. 
Mit dem \inlinecode{npm install}-Befehl wurden alle aufgelisteten Abhängigkeiten in der  \inlinecode{package.json}-Datei installiert und im Ordner \inlinecode{node-modules} gespeichert.
Das Linken der \glslink{Library}{Libraries} fand dann in der MainActivity und in \brand{Xcode} statt. 
Nun konnten die \glslink{Library}{Libraries} in einer \brand{JavaScript}-Komponente importiert und verwendet werden. 


\section{Navigation}
Wie die Navigations-Komponente verwendet wird, ist auf der \brand{Github}-Seite erklärt. 

% Aufbau
% Deklaration der Scenes - beim Kompilieren
% Reducer
% Actions


\section{Karte}
% Mapbox
% Annotations vom Store
% Parameter
% Token
% Location-tracking Einstellung


\section{OAuth}
% Google signin
% Google Konfigurationsdatei generieren mit Debug Keystore
% Google Developer Account für Web-Client-ID und iOS-Client-ID
% Secret config


\section{Internationalisierung}
% Properties und Json Umwandlung hier?



% Weiterentwicklung
\chapter{Weiterentwicklung}
\label{pd-weiterentwicklung}

\kort{} hat ein grosses Potenzial um weiterentwickelt zu werden.
Dieses umfasst vor allem zwei Bereiche:

Wie kann \kort{} besser dazu beitragen, \brand{OpenStreetMap} Daten zu verbessern?

Und wie kann der Benutzer mithilfe von Konzepten der Gamification weiter motiviert werden, zur Datenpflegung beizutragen?

Wir haben uns Gedanken dazu gemacht und hier zusammengefasst, wie \kort{} weiter optimiert werden könnte.
Das Unterkapitel \hyperref[pd-weiterentwicklung-vorgehen]{Vorgehen} enthält Arbeiten, die beim derzeitigen Stand noch verbessert werden müssen. 
Weiterhin sind Ideen aufgelistet, die keine grösseren Änderungen am Backend mit sich bringen.

\section{Vorgehen}
\label{pd-weiterentwicklung-vorgehen}
Ein sehr wichtiges Feature, das bisher noch nicht umsetzbar war, ist das \brand{OpenStreetMap}-Login.
Neben den Änderungen am Backend, damit dieses Feature überhaupt realisierbar ist, gibt es folgende Schritte zu erledigen:

\begin{enumerate}
	\item App bei \brand{OpenStreetMap} mit Callback-URL registrieren
	\begin{itemize}
		\item Als Callback-URL könnte zum Beispiel 'http://www.kort.ch/' verwendet werden.
	\end{itemize}
	\item Request-Token-URL aufrufen, um das OAuth-Token und das Token-Secret zu erhalten
	\item Authorize-URL mit dem OAuth-Token aufrufen, damit sich der Benutzer auf der \brand{OpenStreetMap}-Seite einloggen kann
	\item WebView-Komponente öffnen, um auf die Callback-URL zu warten 
	\item Wenn sich der Benutzer eingeloggt hat, wird er zur Callback-Seite, mit dem OAuth-Token und dem OAuth-Verifier in der URL, umgeleitet.
	\item WebView schliessen
	\item Request Access-Token-URL
	\item dem Backend das Token zur Überprüfung senden
\end{enumerate}

Dieses Vorgehen wurde noch nicht getestet und dient nur als Idee zur möglichen Umsetzung.
Die korrekten Request-URLs können aus der Dokumentation im \brand{OpenStreetMap}-Wiki\footnote{\url{http://wiki.openstreetmap.org/wiki/OAuth}} entnommen werden.

Als nächstes müsste für den \gls{Gamification}-Ansatz das Design weiter verbessert werden. 
Dafür wäre zuerst eine Änderung am Backend geplant, um die Validationen endgültig abzuschaffen, indem sie als normale Missionen gezählt werden. 
Dann wäre es auch wieder möglich korrekte Badges für den Missions-Counter anzuzeigen. 

Durch den Einsatz von Farben oder einem Hintergrundbild beim Login-Screen würde das Design einen Spieler besser ansprechen.  
Passend dazu könnten die verwendeten Bilder, Icons und Marker einheitlich gestaltet und erneuert werden. 

Meldungen, die dem Benutzer die Anzahl gewonnener Koins anzeigen, erscheinen momentan im Vollbildmodus. 
Das liegt an einem Fehler der eingesetzten Navigations-Komponente, die keine transparenten Hintergründe zulässt. 
In Zukunft könnte das aber noch behoben werden. 
Dann wäre es eleganter, Meldungen in einem Fenster zu zeigen.

Um die Funktionalität der \gls{WebApp} komplett umgesetzt zu haben fehlt nur noch die zweite Highscore-Ansicht und die Karte, die beim Lösen einer Mission angezeigt wird.

Gerne hätten wir noch Tests der Komponenten durchgeführt, doch die Priorität dafür war zu tief und die Zeit zu knapp. 
Dazu haben wir für die Zukunft die Enzyme\footnote{\url{http://airbnb.io/enzyme/}}-Testing-API (für Komponenten-Tests) und das Mocha\footnote{\url{https://github.com/mochajs/mocha}}-\gls{Framework} (um die Tests laufen zu lassen) evaluiert.

\section{Realistische Arbeiten}
\label{pd-weiterentwicklung-realistisch}
Punkte, die \kort{} attraktiver machen würden:

\begin{itemize}
	\item Benutzerlogin mit weiteren \gls{OAuth}-Diensten erweitern (z.B. \brand{Twitter}, \brand{GitHub})
	\item \gls{Gamification}
	\begin{itemize}
		\item von gesammelten \emph{Koins} abhängige Levels einführen (z.B. bestimmte Fehlertypen erst ab fortgeschrittenem Level anzeigen, Avatars, Levelbezeichnungen)
		\item verschiedene Schwierigkeitsstufen
		\item Einbindung in \brand{Game Center} der jeweiligen Plattform
		\item weitere Badges einführen (viele Ideen finden sich hier: \url{https://wiki.openstreetmap.org/wiki/Badges}, zum Beispiel auch Badges für Spielertypen)
		\item verschiedene Highscores anzeigen (zum Beispiel zeitlich oder Regional begrenzt, nach Fehlertypen kategorisiert, schnellste Aufsteiger)
		\item zusätzliche Berechtigungen für erfahrene Benutzer (für den langfristigen Erfolg)
	\end{itemize}
	\item neue realistische Fehlertypen:\footnote{\url{https://github.com/kort/kort/issues/81}}
	\begin{itemize}
		\item Hausnummern einfügen
		\item Stockwerk-Anzahl einfügen
		\item Einbahnstrassen erfassen
		\item Öffnungszeiten von öffentlichen Gebäuden festhalten
	\end{itemize}
	\item weniger realistische Fehlertypen:
		\begin{itemize}
			\item Kreisel erfassen
			\item Bushaltestellen von \brand{DIDOK}\footnote{\url{https://didok.osm.ch/}} erfassen
		\end{itemize}
	\item Erkennen von Benutzern, die nicht sorgfältig validieren\footnote{\url{https://github.com/kort/kort/issues/7}}
	\item ausführliche Statistiken für individuelle Benutzer\footnote{\url{https://github.com/kort/kort/issues/71}}
	\item Aufträge aus \brand{wheelmap}\footnote{\url{http://wheelmap.org}}	einfügen
	\item Offline-Fähigkeit (offline Maps für \brand{React Native} wären erforderlich)
	\item wenn Aufträge zum Beispiel drei Mal nicht gelöst werden können, soll eine \brand{OpenStreetMap}-Notiz generiert werden
\end{itemize}

\subsubsection{Unrealistische Arbeiten}
Punkte, die für \kort{} als weniger geeignet empfunden wurden:

\begin{itemize}
	\item Erweitern der Verifizierung mit der Möglichkeit, ein Foto als Beweis hochzuladen
	\begin{itemize}
	  \item \emph{Begründung:} Aspekte des Datenschutzes bergen ein gewisses Risiko. Benutzer müssten für das Hochladen von Bildern zusätzliche Bedingungen akzeptieren.
	\end{itemize}
	\item standortunabhängige Aufgaben lösen (Gefahr von Couch Mapping)
	\begin{itemize}
	  \item \emph{Begründung:} Es ist ein Anliegen der \brand{OpenStreetMap} Community, dass die Mapper vor Ort sein sollen um Aufträge zu lösen. 
	\end{itemize}
\end{itemize}


% Installation
\chapter{Installation}
\label{pd-installation}

In diesem Kapitel werden zwei Varianten beschrieben, wie die App getestet werden kann. 


\section{Installation mit Source Code}
Um die App anhand des Source Codes zu installieren, muss die \brand{React Native} Entwicklungsumgebung aufgesetzt sein. 
Wie diese eingerichtet wird, ist in der Getting-Started-Anleitung, der Onlinedokumentation von \brand{React Native}\footnote{\url{https://facebook.github.io/react-native/docs/getting-started.html}}, erklärt.
Diese Anleitung erklärt Schritt für Schritt den Ablauf von der Einrichtung der Entwicklungsumgebung auf einem \brand{Mac}-, \brand{Linux}- oder \brand{Windows}-Rechner (für \brand{iOS} und \brand{Android}) bis zum Starten der App.

Hier muss noch beachtet werden, dass die SecretConfig.js-Datei mit folgenden Werten ergänzt werden sollte.
Der Dateipfad lautet: \inlinecode{/kort-reloaded/js/constants/SecretConfig.js}\newline

\begin{itemize}
	\item Mapbox Access Token
	\begin{itemize}
		\item Dieses Token kann auf der Mapbox-Webseite\footnote{\url{https://www.mapbox.com/help/create-api-access-token/}} erstellt werden.
	\end{itemize}
	
	\item Google Client ID 
	\begin{itemize}
		\item Wie die Google Client ID erhalten wird, ist in der Anleitung des \brand{GitHub}-Projekts der Google-Signin-Library erklärt.\footnote{\url{https://www.mapbox.com/help/create-api-access-token/}}
		\item Wichtig ist, dass anhand dieser Anleitung auch die google-services.json-Datei neu generiert und in das Projekt eingefügt werden muss.
	\end{itemize}
\end{itemize}

\section{Installation mit APK-Datei}
Damit die \brand{Android}-App von der bereitgestellten APK-Datei auf der CD installiert werden kann, muss das Smartphone mit dem Computer per USB-Kabel verbunden sein und erkannt werden. 

\begin{enumerate}
	\item Die APK-Datei in einen Ordner auf dem Gerätespeicher kopieren.
	\item Die Smartphone Verbindung mit dem Computer trennen.
	\item Die APK-Datei auf dem Smartphone auffinden und anklicken.
	\item Dem App-Herausgeber vertrauen und Installation abschliessen.
\end{enumerate}


% Neue Seite beginnen
\cleardoublepage

% Projektmanagement
\part{Projektmanagement}
%Projektmanagement
%Projektmanagement
%Projektmanagement
\input{body/projektmanagement/projektmanagement.tex}

%Projektmonitoring
\input{body/projektmanagement/projektmonitoring.tex}

%Projektmonitoring
\chapter{Projektmonitoring}
\label{pm-projektmonitoring}

\section{Zeitanalyse}
Für eine Bachelorarbeit werden 12 ECTS-Punkte vergeben, wobei ein Punkt einem Aufwand von 30 Stunden entspricht.
In einem Team von zwei Entwicklern, entspricht dies 360 Stunden Aufwand pro Person. 
Leider haben wir diese Vorgabe (siehe Tabelle \ref{pm-arbeitsaufwand}) überschritten. 

\begin{table}[H]
\centering
\begin{tabular}{|l|r|}
\hline 
\multicolumn{1}{|c|}{\textbf{Person}} & \multicolumn{1}{|c|}{\textbf{Aufwand}} \\
\hline 
Marino Melchiori & 461.5 h \\
\hline 
Dominic Mülhaupt & 467 h \\  
\hline 
\end{tabular}
\caption{Arbeitsaufwand pro Person}
\label{pm-arbeitsaufwand}
\end{table}

Wenn die gleich folgenden Tabellen \ref{pm-arbeitsaufwand-aktivität-mm} und \ref{pm-arbeitsaufwand-aktivität-dm} verglichen werden, ist die Arbeitsaufteilung sehr gut erkennbar. 
Dies ist in den Unterschieden des Aufwandes, in den Aktivitäten Analyse, Implementation, Dokumentation und Testing, erkennbar. 
Bei neuen Erfahrungen haben sich die Entwickler ausgetauscht und waren so immer auf dem aktuellen Stand. 

\begin{table}[H]
\centering
\label{pm-arbeitsaufwand-aktivität-mm}
\begin{tabular}{|l|r|}
\hline
\multicolumn{2}{|l|}{\textbf{Marino Melchiori}} \\
\hline
\multicolumn{1}{|c|}{\textbf{Aktivität}} & \multicolumn{1}{|c|}{\textbf{Aufwand}} \\
\hline
Analyse \& Design & 106.00 h \\
\hline
Implementation & 125.00 h \\
\hline
Dokumentation & 141.00 h \\
\hline
Requirements & 30.75 h \\
\hline
Deployment & 16.25 h \\
\hline
Allgemein & 16.75 h \\
\hline
Projektmanagement & 25.75 h \\
\hline
\end{tabular}
\caption{Aufwand pro Aktivität --- Marino Melchiori}
\end{table}

\begin{table}[H]
\centering
\label{pm-arbeitsaufwand-aktivität-dm}
\begin{tabular}{|l|r|}
\hline
\multicolumn{2}{|l|}{\textbf{Dominic Mülhaupt}} \\
\hline
\multicolumn{1}{|c|}{\textbf{Aktivität}} & \multicolumn{1}{|c|}{\textbf{Aufwand}} \\
\hline
Analyse \& Design & 49.50 h \\
\hline
Implementation & 191.75 h \\
\hline
Dokumentation & 66.25 h \\
\hline
Requirements & 26.00 h \\
\hline
Deployment & 7.00 h \\
\hline
Allgemein & 52.50 h \\
\hline
Projektmanagement & 52.00 h \\
\hline
Testing & 22.00 h \\
\hline
\end{tabular}
\caption{Aufwand pro Aktivität --- Dominic Mülhaupt}
\end{table}


\subsection{Soll-Ist-Zeitvergleich}
Die Tabelle zum \ref{pm-arbeitsaufwand-kategorie-ges} zeigt, wie das Projekt kategorisiert wurde. 
In der \textit{Kort Projekt}-Kategorie ging es vor allem um die Implementation der Architektur und die Verwendung und Einrichtung von \brand{React Native}. 
Die Kategorien \textit{Login}, \textit{Missionen}, \textit{Profil} und \textit{Highscore} enthalten die Implementation der \gls{GUI}. 
\textit{Map} und \textit{Internationalisierung} sind Kategorien, bei denen es um die Installation der entsprechend verwendeten \glslink{Library}{Libraries} ging. 
In der Kategorie \textit{Technologien} wurde die Einarbeitungszeit gebucht. 
Die \textit{Qualitätssicherung} beinhaltet das Testing und die Kategorie \textit{Allgemein} die Evaluation von verwendeten Konzepten und \glslink{Library}{Libraries}.

Die Schätzung fand jeweils bei der Erstellung eines Tickets in einer Kategorie statt. 

Die Differenz der \textit{Kort Projekt}-Kategorie lässt sich durch die häufigen Build-Fehler erklären. 
Das Suchen nach Lösungen im Internet war sehr Zeitaufwendig. 

Bei der Kategorie \textit{Missionen} wurden 15 Stunden zu viel geschätzt. 
In diesem Fall setzten wir vermehrt auf \gls{Pair Programming}, da es sich dort um eine der ersten umgesetzten Funktionen handelte.
Somit konnte kostbare Zeit, die für das Refactoring geplant war, eingespart werden. 
Später folgende Features, der Kategorien \textit{Login}, \textit{Profil} und \textit{Highscore}, wurden viel besser geschätzt.

Schlussendlich wurden 21 Stunden zu viel geschätzt, was bei einem Arbeitstag von 8 Stunden etwa 2.5 Arbeitstagen entspricht. 


\begin{table}[H]
\centering
\label{pm-arbeitsaufwand-kategorie-ges}
\begin{tabular}{|l|r|r|r|}
\hline
\multicolumn{4}{|l|}{\textbf{Soll-Ist-Vergleich vom Gesamtaufwand der Kategorien}} \\
\hline
\multicolumn{1}{|c|}{\textbf{Kategorie}} & \multicolumn{1}{|c|}{\textbf{Soll-Aufwand}} & \multicolumn{1}{|c|}{\textbf{Ist-Aufwand}} & \multicolumn{1}{|c|}{\textbf{Differenz}}\\
\hline
Kort Projekt & 71.75 h & 88.25 h & +16.50 h \\
\hline
Login & 91.50 h & 92.50 h & +1.00 h \\
\hline
Map & 47.75	h & 45.50 h & -2.25 h \\
\hline
Missionen & 59.75 h & 44.75 h & -15.00 h \\
\hline
Highscore & 11.50 h & 12.70 h & +1,20 h \\
\hline
Profil & 18.00 h & 16.25 h & -1.75 h \\
\hline
Technologien & 182.50 h & 187.55 h & +0.05 h \\
\hline
Gamification und Design & 7.25 h & 4.75 h & -2.50 h \\
\hline
Qualitätssicherung & 36.50 h & 30.00 h & -6.50 h \\
\hline
Internationalisierung & 4.50 h & 4.25 h & -0.25 h \\
\hline
Dokumentation & 204.00 h & 198.25 h & -5.75 h \\
\hline
Allgemein & 152.75 h & 144.50 h & -8.25 h \\
\hline
Meeting & 61.75 h & 59.25 h & -2.50 h \\
\hline
\textbf{Total} & \textbf{975.00 h} & \textbf{926.50 h} & \textbf{-21.00 h} \\
\hline
\end{tabular}
\caption{Soll-Ist-Vergleich --- Gesamtaufwand pro Kategorie}
\end{table}

\section{Code-Statistik}

% Anzahl Zeilen pro Package (Actions, Stores, etc...)
Zur groben Einschätzung, der Grösse vom Projekt wurde in der Tabelle \ref{pm-cloc} die Gesamtanzahl der Dateien und der Code-Zeilen (ohne Code-Kommentare) gezählt.

\begin{table}[H]
\centering
\begin{tabular}{|l|l|l|}
\hline 
\textbf{Sprache} & \textbf{Dateien} & \textbf{Zeilen} \\ 
\hline 
\brand{JavaScript} & ... & ... \\
\hline 
\end{tabular}
\caption{Dateien und Codezeilen}
\label{pm-cloc}
\end{table}

Die Tabelle \ref{pm-package-cloc} zeigt die Anzahl an Code-Zeilen in einem Package. 
Die Packages \textit{Actions}, \textit{Data}, \textit{Dispatcher}, \textit{DTO} und \textit{Stores} sind von der verwendeten Architektur gegeben. 
\textit{Constants} beinhaltet Konstanten, IDs und sonstige fixe Parameter. 
Im Package \textit{Components} befinden sich die \gls{GUI}-Komponenten. 

\begin{table}[H]
\centering
\begin{tabular}{|l|l|}
\hline 
\textbf{Package} & \textbf{Zeilen} \\ 
\hline 
\brand{Actions} & ... \\
\hline 
\brand{Components} & ... \\
\hline 
\brand{Constants} & ... \\
\hline 
\brand{Data} & ... \\
\hline 
\brand{Dispatcher} & ... \\
\hline 
\brand{DTOs ohne Tests} & ... \\
\hline 
\brand{DTO Tests} & ... \\
\hline 
\brand{Stores ohne Tests} & ... \\
\hline 
\brand{Store Tests} & ... \\
\hline 
\end{tabular}
\caption{Codezeilen pro Package}
\label{pm-package-cloc}
\end{table}


%Projektmonitoring
\chapter{Projektmonitoring}
\label{pm-projektmonitoring}

\section{Zeitanalyse}
Für eine Bachelorarbeit werden 12 ECTS-Punkte vergeben, wobei ein Punkt einem Aufwand von 30 Stunden entspricht.
In einem Team von zwei Entwicklern, entspricht dies 360 Stunden Aufwand pro Person. 
Leider haben wir diese Vorgabe (siehe Tabelle \ref{pm-arbeitsaufwand}) überschritten. 

\begin{table}[H]
\centering
\begin{tabular}{|l|r|}
\hline 
\multicolumn{1}{|c|}{\textbf{Person}} & \multicolumn{1}{|c|}{\textbf{Aufwand}} \\
\hline 
Marino Melchiori & 461.5 h \\
\hline 
Dominic Mülhaupt & 467 h \\  
\hline 
\end{tabular}
\caption{Arbeitsaufwand pro Person}
\label{pm-arbeitsaufwand}
\end{table}

Wenn die gleich folgenden Tabellen \ref{pm-arbeitsaufwand-aktivität-mm} und \ref{pm-arbeitsaufwand-aktivität-dm} verglichen werden, ist die Arbeitsaufteilung sehr gut erkennbar. 
Dies ist in den Unterschieden des Aufwandes, in den Aktivitäten Analyse, Implementation, Dokumentation und Testing, erkennbar. 
Bei neuen Erfahrungen haben sich die Entwickler ausgetauscht und waren so immer auf dem aktuellen Stand. 

\begin{table}[H]
\centering
\label{pm-arbeitsaufwand-aktivität-mm}
\begin{tabular}{|l|r|}
\hline
\multicolumn{2}{|l|}{\textbf{Marino Melchiori}} \\
\hline
\multicolumn{1}{|c|}{\textbf{Aktivität}} & \multicolumn{1}{|c|}{\textbf{Aufwand}} \\
\hline
Analyse \& Design & 106.00 h \\
\hline
Implementation & 125.00 h \\
\hline
Dokumentation & 141.00 h \\
\hline
Requirements & 30.75 h \\
\hline
Deployment & 16.25 h \\
\hline
Allgemein & 16.75 h \\
\hline
Projektmanagement & 25.75 h \\
\hline
\end{tabular}
\caption{Aufwand pro Aktivität --- Marino Melchiori}
\end{table}

\begin{table}[H]
\centering
\label{pm-arbeitsaufwand-aktivität-dm}
\begin{tabular}{|l|r|}
\hline
\multicolumn{2}{|l|}{\textbf{Dominic Mülhaupt}} \\
\hline
\multicolumn{1}{|c|}{\textbf{Aktivität}} & \multicolumn{1}{|c|}{\textbf{Aufwand}} \\
\hline
Analyse \& Design & 49.50 h \\
\hline
Implementation & 191.75 h \\
\hline
Dokumentation & 66.25 h \\
\hline
Requirements & 26.00 h \\
\hline
Deployment & 7.00 h \\
\hline
Allgemein & 52.50 h \\
\hline
Projektmanagement & 52.00 h \\
\hline
Testing & 22.00 h \\
\hline
\end{tabular}
\caption{Aufwand pro Aktivität --- Dominic Mülhaupt}
\end{table}


\subsection{Soll-Ist-Zeitvergleich}
Die Tabelle zum \ref{pm-arbeitsaufwand-kategorie-ges} zeigt, wie das Projekt kategorisiert wurde. 
In der \textit{Kort Projekt}-Kategorie ging es vor allem um die Implementation der Architektur und die Verwendung und Einrichtung von \brand{React Native}. 
Die Kategorien \textit{Login}, \textit{Missionen}, \textit{Profil} und \textit{Highscore} enthalten die Implementation der \gls{GUI}. 
\textit{Map} und \textit{Internationalisierung} sind Kategorien, bei denen es um die Installation der entsprechend verwendeten \glslink{Library}{Libraries} ging. 
In der Kategorie \textit{Technologien} wurde die Einarbeitungszeit gebucht. 
Die \textit{Qualitätssicherung} beinhaltet das Testing und die Kategorie \textit{Allgemein} die Evaluation von verwendeten Konzepten und \glslink{Library}{Libraries}.

Die Schätzung fand jeweils bei der Erstellung eines Tickets in einer Kategorie statt. 

Die Differenz der \textit{Kort Projekt}-Kategorie lässt sich durch die häufigen Build-Fehler erklären. 
Das Suchen nach Lösungen im Internet war sehr Zeitaufwendig. 

Bei der Kategorie \textit{Missionen} wurden 15 Stunden zu viel geschätzt. 
In diesem Fall setzten wir vermehrt auf \gls{Pair Programming}, da es sich dort um eine der ersten umgesetzten Funktionen handelte.
Somit konnte kostbare Zeit, die für das Refactoring geplant war, eingespart werden. 
Später folgende Features, der Kategorien \textit{Login}, \textit{Profil} und \textit{Highscore}, wurden viel besser geschätzt.

Schlussendlich wurden 21 Stunden zu viel geschätzt, was bei einem Arbeitstag von 8 Stunden etwa 2.5 Arbeitstagen entspricht. 


\begin{table}[H]
\centering
\label{pm-arbeitsaufwand-kategorie-ges}
\begin{tabular}{|l|r|r|r|}
\hline
\multicolumn{4}{|l|}{\textbf{Soll-Ist-Vergleich vom Gesamtaufwand der Kategorien}} \\
\hline
\multicolumn{1}{|c|}{\textbf{Kategorie}} & \multicolumn{1}{|c|}{\textbf{Soll-Aufwand}} & \multicolumn{1}{|c|}{\textbf{Ist-Aufwand}} & \multicolumn{1}{|c|}{\textbf{Differenz}}\\
\hline
Kort Projekt & 71.75 h & 88.25 h & +16.50 h \\
\hline
Login & 91.50 h & 92.50 h & +1.00 h \\
\hline
Map & 47.75	h & 45.50 h & -2.25 h \\
\hline
Missionen & 59.75 h & 44.75 h & -15.00 h \\
\hline
Highscore & 11.50 h & 12.70 h & +1,20 h \\
\hline
Profil & 18.00 h & 16.25 h & -1.75 h \\
\hline
Technologien & 182.50 h & 187.55 h & +0.05 h \\
\hline
Gamification und Design & 7.25 h & 4.75 h & -2.50 h \\
\hline
Qualitätssicherung & 36.50 h & 30.00 h & -6.50 h \\
\hline
Internationalisierung & 4.50 h & 4.25 h & -0.25 h \\
\hline
Dokumentation & 204.00 h & 198.25 h & -5.75 h \\
\hline
Allgemein & 152.75 h & 144.50 h & -8.25 h \\
\hline
Meeting & 61.75 h & 59.25 h & -2.50 h \\
\hline
\textbf{Total} & \textbf{975.00 h} & \textbf{926.50 h} & \textbf{-21.00 h} \\
\hline
\end{tabular}
\caption{Soll-Ist-Vergleich --- Gesamtaufwand pro Kategorie}
\end{table}

\section{Code-Statistik}

% Anzahl Zeilen pro Package (Actions, Stores, etc...)
Zur groben Einschätzung, der Grösse vom Projekt wurde in der Tabelle \ref{pm-cloc} die Gesamtanzahl der Dateien und der Code-Zeilen (ohne Code-Kommentare) gezählt.

\begin{table}[H]
\centering
\begin{tabular}{|l|l|l|}
\hline 
\textbf{Sprache} & \textbf{Dateien} & \textbf{Zeilen} \\ 
\hline 
\brand{JavaScript} & ... & ... \\
\hline 
\end{tabular}
\caption{Dateien und Codezeilen}
\label{pm-cloc}
\end{table}

Die Tabelle \ref{pm-package-cloc} zeigt die Anzahl an Code-Zeilen in einem Package. 
Die Packages \textit{Actions}, \textit{Data}, \textit{Dispatcher}, \textit{DTO} und \textit{Stores} sind von der verwendeten Architektur gegeben. 
\textit{Constants} beinhaltet Konstanten, IDs und sonstige fixe Parameter. 
Im Package \textit{Components} befinden sich die \gls{GUI}-Komponenten. 

\begin{table}[H]
\centering
\begin{tabular}{|l|l|}
\hline 
\textbf{Package} & \textbf{Zeilen} \\ 
\hline 
\brand{Actions} & ... \\
\hline 
\brand{Components} & ... \\
\hline 
\brand{Constants} & ... \\
\hline 
\brand{Data} & ... \\
\hline 
\brand{Dispatcher} & ... \\
\hline 
\brand{DTOs ohne Tests} & ... \\
\hline 
\brand{DTO Tests} & ... \\
\hline 
\brand{Stores ohne Tests} & ... \\
\hline 
\brand{Store Tests} & ... \\
\hline 
\end{tabular}
\caption{Codezeilen pro Package}
\label{pm-package-cloc}
\end{table}

%
% -----------------------------------------
% FOOT
% -----------------------------------------
% Neue Seite beginnen
\cleardoublepage

% Anhänge
\part{Anhänge}

% Inhalt der CD
% Inhalt der CD
\chapter*{Inhalt der CD}
% Titel auch in Kopfzeile anzeigen
\markboth{Inhalt der CD}{Inhalt der CD}
% Kapitel in Inhaltsverzeichnis einfügen
\addcontentsline{toc}{chapter}{Inhalt der CD}

\begin{forest}
  for tree={
    font=\sffamily,
    text=white,
    text width=10cm,
    minimum height=0.75cm,
    if level=0
      {fill=root}
      {fill=folder},
    rounded corners=4pt,
    grow'=0,
    child anchor=west,
    parent anchor=south,
    anchor=west,
    calign=first,
    edge={root,rounded corners,line width=1pt},
    edge path={
      \noexpand\path [draw, \forestoption{edge}]
      (!u.south west) +(7.5pt,0) |- (.child anchor)\forestoption{edge label};
    },
    before typesetting nodes={
      if n=1
        {insert before={[,phantom]}}
        {}
    },
    fit=band,
    s sep=8pt,
    before computing xy={l=15pt},
  }
[/
  [Dokumentation/
    [Kort Reloaded - Marino Melchiori und Dominic Mülhaupt.pdf]
    [Anhänge/
      [Abstract.pdf]
      [Aufgabenstellung.pdf]
      [Eigenständigkeitserklärung.pdf]
      [Einverständniserklärung Publikation auf eprints.pdf]
      [Gamified Mobile App für die Verbesserung von OpenStreetMap - Jürg Hunziker und Stefan Oderbolz.pdf]
      [Poster.pdf]
      [Diagramme/]
      [Organisatorisches/]
      [Sitzungsprotokolle/]
    ]
  ]
  [Kort/
	[Sourcecode/
	  [kort-reloaded/
	    [android/]
	    [ios/]
	    [js/]
	  ]
	]
	[APK]
  ]
]
\end{forest}
<<<<<<< HEAD
\label{cd-inhalt}
=======
>>>>>>> 466b2e37f8880c41847521c43ca3d4b506cbc142

% Glossar
\printglossary[style=altlist, title=Glossar, toctitle=Glossar]
\label{glossar}

% Literaturverzeichnis
\bibliographystyle{plainurl}
\bibliography{foot/literatur}

% Abbildungsverzeichnis
\listoffigures

% Tabellenverzeichnis
\listoftables

\end{document}