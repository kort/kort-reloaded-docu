\chapter*{Dank}
\thispagestyle{scrheadings}
% Titel auch in Kopfzeile anzeigen
\markboth{Dank}{Dank}

Für die Betreuung während des ganzen Projektes möchten wir uns besonders bei Herr Prof. Stefan
Keller bedanken.
Bei Fragen zu \brand{OpenStreetMap} konnte er uns während der gesamten Laufzeit der Arbeit unterstützen und er gab uns bei Problemen immer Inspirationen für eine Lösung. 
Ein Dank geht auch an Jürg Hunziker und Stefan Oderbolz, für die Unterstützung während des ganzen Projektes.
Sie gaben uns Tipps zum Vorgehen und haben für uns grössere und wichtige Backend-Änderungen vorgenommen. 
Herr Prof. Stefan Keller, Jürg Hunziker und Stefan Oderbolz liessen uns freien Spielraum für die Suche nach einer optimalen Lösung.

\section*{Ausgangslage}
Das \brand{OpenStreetMap}-Projekt beinhaltet eine sehr grosse Menge an Daten, welche frei zugänglich sind.
Für die Pflege dieser Daten ist es daher naheliegend, auf unterstützende Software zurückzugreifen.
Zu diesem Zweck gibt es eine Reihe von Applikationen, welche sich grob in zwei Kategorien einteilen lassen:
Editoren und Tools zur Qualitätssicherung.

Mit den Editoren lässt sich direkt oder indirekt die \brand{OpenStreetMap}-Karte verändern und ergänzen.
Die Qualitätssicherungstools haben sich zum Ziel gesetzt, fehlende oder falsche Daten aufzuspüren.
Diese werden dann entweder automatisch korrigiert oder übersichtlich dargestellt, um eine manuelle Korrektur zu ermöglichen.

Einige Tools wie \brand{KeepRight}\footnote{\url{http://keepright.ipax.at/}} oder \brand{Osmose}\footnote{\url{http://osmose.openstreetmap.fr/map/}} berechnen aus den Karten-Rohdaten die vorhandenen Fehler.
Dazu werden einige Heuristiken verwendet oder einfache Plausibilitätsprüfungen durchgeführt.
Typische Fehler aus diesen Quellen sind \glspl{POI} ohne Namen oder Strassen ohne definierte Geschwindigkeitslimiten.

Zur Behebung dieser Fehler ist die cross-platform \gls{WebApp} \kort{} in Form einer Bachelorarbeit von Jürg Hunziker und Stefan Oderbolz, im Herbstsemester 2012/13, entwickelt  worden.

Unsere Aufgabe war es die nicht mehr voll funktionsfähige \gls{WebApp} mit einer native Android App zu ersetzen.
Das erste Ziel war die Einbettung der \brand{OpenStreetMap}-Karte.
Für eine editierbare Karte, die auch Bilder enthält, gibt es für \brand{React-Native} eine experimentelle MapBox Library. 
%ToDo: Verweis richtig verlinken
Die Ziele sind in der Aufgabenstellung genauer definiert.
Verwenden konnten wir die ganze Infrastruktur der \kort{} Bachelorarbeit.

