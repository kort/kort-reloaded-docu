\chapter*{Aufgabenstellung}
\thispagestyle{scrheadings}
% Titel auch in Kopfzeile anzeigen
\markboth{Aufgabenstellung}{Aufgabenstellung}

Der \kort{}-Client soll neu als native \brand{Android} App zur Verfügung stehen. Das bedingt ein kompletter Rewrite des aktuellen \brand{JavaScript}-Codes (aktuell \brand{Sencha Touch}) mit dem \gls{Framework} \brand{React Native}. Es soll ein Erfahrungsbericht zu \brand{React Native} erstellt werden.

Ziele:
\begin{itemize}
	\item Gleiche Funktionalität mit neuem Framework, damit die mobile App mit den neusten Technologien arbeitet und künftig besser wartbar ist
	\item Neue Erkenntnisse zu einem aktuellen Framework zur Realisierung von native mobile
Apps \brand{React Native}
	\item Getestete Software-Entwicklungsumgebung
\end{itemize}

\section*{Deliverables}
Mindestens...
\begin{itemize}
	\item Neuer \kort{}-Client als native \brand{Android}-App
	\item Ablösung des Validationsmechanismus'
	\item Erfahrungsbericht mit Hinweisen zu Tutorials zu \brand{React Native}
	\item Erweiterte Software-Entwicklungsumgebung
	\item Die vom Studiengang geforderten Lieferobjekte: Dokumentation, Management Summary, Abstract, Poster, Präsentation mit Stellwand, Zwischenpräsentation
\end{itemize}

Erweitert...
\begin{itemize}
	\item \kort{}-Client als native \brand{iOS}-App
	\item neue Funktion: Promotions anzeigen
	\item Kurzvideo
\end{itemize}

Die definitive Aufgabenstellung, Lieferobjekte und das Vorgehen werden am Kickoff (erste Semesterwoche) zusammen mit dem Industriepartner festgelegt. Die gemeinsam besprochene Aufgabenstellung wird ca. zwei Wochen nach Semesterbeginn aktualisiert.

\section*{Vorgaben/Rahmenbedingungen}
\begin{itemize}
	\item Die ursprüngliche \kort{}-App ist clientseitig in \brand{HTML5} und \brand{JavaScript} geschrieben.
	\item Serverseitig ist u.a. \brand{PHP} und \brand{PostgreSQL} mit der \brand{PostGIS}-Erweiterung vorhanden.
	\item Als Technologien stehen \brand{React} und \brand{React Native} im Vordergrund.
	\item Es wird genügend Zeit für die Einarbeitung in die Themengebiete einberechnet.
	\item Die Software soll Open Source sein.
	\item Die SW-Engineering-Methode und Meilensteine werden mit dem Betreuer vereinbart.
	\item Sourcecode und Software-Dokumentation sind Englisch (inkl. Installation, keine Benutzerdokumentation, höchstens eine Online-Kurzhilfe).
	\item Die Software-Benutzerschnittstelle ist mind. Deutsch und Englisch.
	\item Die Projekt-Dokumentation und -Präsentation sind auf Deutsch. 
	\item Der Source Code, die Code-Kommentare und die Versionsverwaltung sind in Englisch.
	\item Die Nutzungsrechte an der Arbeit bleiben bei den Autoren und gehen auch an die \brand{HSR} und den Betreuer über. Die Softwarelizenz ist „MIT“.
	\item Ein Video gemäss den Vorgaben des Studiengangs (kann ggf. nach dem Dokumentations-Abgabetermin abgegeben werden).
	\item Ansonsten gelten die Rahmenbedingungen, Vorgaben und Termine des Studiengangs Informatik bzw. der \brand{HSR}.
\end{itemize}

\section*{Inhalt der Dokumentation}
\begin{itemize}
	\item Die fertige Arbeit muss folgende Inhalte haben:
	\begin{enumerate}
		\item Abstract, Aufgabenstellung
		\item Technischer Bericht
		\item Projektdokumentation
		\item Anhänge (Literaturverzeichnis, Glossar, CD-Inhalt)
	\end{enumerate}
	\item Die Abgabe ist so zu gliedern, dass die obigen Inhalte klar erkenntlich und auffindbar sind.
	\item Zitate sind zu kennzeichnen, die Quelle ist anzugeben.
	\item Verwendete Dokumente und Literatur sind in einem Literaturverzeichnis aufzuführen.
	\item Dokumentation des Projektverlaufes, Planung etc.
	\item Weitere Dokumente (z.B. Kurzbeschreibung, Poster) gemäss \url{www.hsr.ch} und Absprache mit dem Betreuer.
\end{itemize}

\section*{Form der Dokumentation}
Bericht, Dokumente und Quellen der erstellten Software gemäss Vorgaben des Studiengangs Informatik der \brand{HSR} sowie Absprache mit dem Betreuer.

\section*{Bewertungsschema}
Es gelten die üblichen Regelungen zum Ablauf und zur Bewertung der Arbeit (6 Aspekte) des Studiengangs Informatik der \brand{HSR} jedoch mit besonderem Gewicht auf moderne Softwareentwicklung (Tests, Continuous Integration, einfach installierbar, funktionsfähig).

\section*{Beteiligte}
\subsection*{Diplomanden}
Marino Melchiori und Dominic Mülhaupt

\subsection*{Industriepartner}
Jürg Hunziker und Stefan Oderbolz, Liip AG Zürich

\subsection*{Betreuung HSR}
Verantwortlicher Dozent: Prof. Stefan Keller