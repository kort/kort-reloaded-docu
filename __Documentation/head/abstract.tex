\chapter*{Abstract}
\thispagestyle{scrheadings}
% Titel auch in Kopfzeile anzeigen
\markboth{Abstract}{Abstract}

\brand{OpenStreetMap} ist ein freies Projekt und wird von Mappern aus der ganzen Welt erstellt. 
Wege, Gebäude und viele andere geographische Daten werden weltweit in der Datenbank erfasst und gepflegt. 
 \brand{OpenStreetMap} setzt auf lokales Wissen der Autoren, ist Community Driven und kann für jeden Zweck verwendet werden. 
 Aus diesem Grund ist es nicht ausgeschlossen, dass fehlerhafte oder unvollständige Daten enthalten sind.
 
 Zur Korrektur der Daten gibt es viele verschiedene Tools, die von Experten genutzt werden können.
 Um eine breitere Masse anzusprechen entstand 2012 die \gls{WebApp} \kort{} als Bachelorarbeit von Jürg Hunziker und Stefan Oderbolz.
 Die \kort{}-\gls{WebApp} sammelt mit dem \brand{KeepRight}-Dienst Korrekturaufträge für die Nutzer in einer \brand{PostgreSQL} Datenbank.
 In einem Umkreis von fünf Kilometern erhält der Nutzer diese Aufträge auf der Kartenansicht und kann sie lösen.
 
 Da die \gls{WebApp} nur noch im Firefox funktioniert, wurde \kort{} in dieser Bachelorarbeit als native \brand{Android}-App neu entwickelt. 
 Die App ist in \brand{JavaScript} geschrieben und basiert auf dem \brand{React Native} Framework von \brand{Facebook}.
 Das bestehende \kort{}-Backend konnte grundsätzlich  wiederverwendet werden.
 Kommuniziert wird weiterhin mit den vorhandenen \gls{REST}-Schnittstellen.
 Um das Backend effektiv zu nutzen wurde die \brand{flux}-Architektur eingesetzt.
 
 Die vielen Spiel-Elemente, aus dem \gls{Gamification}-Ansatz der \kort{}-\gls{WebApp}, wurden beibehalten.
 