\chapter*{Abstract}
\thispagestyle{scrheadings}
% Titel auch in Kopfzeile anzeigen
\markboth{Abstract}{Abstract}

\brand{OpenStreetMap} ist ein freies Projekt das von Benutzern aus der ganzen Welt unterstützt wird. 
Wege, Gebäude und viele andere geografische Daten werden weltweit in einer Datenbank erfasst und gepflegt. 
\brand{OpenStreetMap} kann von jedem bearbeitet werden, besteht aus einer grossen Community und setzt auf lokales Wissen der «Mapper». 
Aus diesem Grund ist es nicht ausgeschlossen, dass fehlerhafte oder unvollständige Daten enthalten
sind. 
Zur Korrektur der Daten gibt es viele verschiedene Tools, die von Experten genutzt werden können. 
Um eine breitere Masse anzusprechen, entstand 2012 die Web-App «\kort{}» im Rahmen einer Bachelorarbeit. 
Mit \kort{} kann der Benutzer Aufträge lösen, die zur Verbesserung der Daten in \brand{OpenStreetMap} beitragen. 
Auf einer Kartenansicht werden die Aufträge, welche sich im Umfeld des Benutzers befinden, dargestellt. 
Für das Eintragen einer Lösung wird man mit Punkten (sogenannten «Koins») belohnt und kann
so in der Rangliste aufsteigen. 
Da sich HTML5 weiterentwickelt hat, muss die \gls{WebApp} abgelöst werden.

In dieser Arbeit wurde \kort{} als natives Mobile App für \brand{Android} und \brand{iOS} komplett neu entwickelt. 
Das bestehende \gls{Backend} wurde dabei praktisch unverändert weitergenutzt. 
Die App basiert auf dem \brand{React Native} Framework von \brand{Facebook}. 
\brand{React Native} ist eine moderne, noch junge Technologie, welche es ermöglicht, mit \brand{JavaScript} Apps für \brand{Android} und \brand{iOS} zu entwickeln. 
Dabei wird \brand{JavaScript}-Code in Komponenten der jeweiligen Smartphone-Plattform übersetzt, was dem Benutzer die Erfahrung einer nativen App bietet.

Es konnten wichtige Erfahrungen mit \brand{React Native} gesammelt werden. 
Dabei sind eine \brand{Android}- und eine \brand{iOS}-App entstanden. 
Die \brand{Android}-App befindet sich noch in der Beta-Phase und wird nach Abschluss dieser Arbeit im \brand{Google Play} Store erwartet. 
Die \brand{iOS}-App ist noch in der Testphase und wird später im \brand{Apple} App Store veröffentlicht. 
Weitere Informationen: www.kort.ch. 
