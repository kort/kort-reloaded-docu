\chapter*{Abstract}
\thispagestyle{scrheadings}
% Titel auch in Kopfzeile anzeigen
\markboth{Abstract}{Abstract}

\brand{OpenStreetMap} ist ein freies Projekt und wird von Mappern aus der ganzen Welt unterstützt.
 Wege, Gebäude und viele andere geographische Daten werden weltweit in der Datenbank erfasst und gepflegt. 
 OpenStreetMap kann von jedem bearbeitet werden, besteht aus einer grossen Community und setzt auf lokales Wissen der Autoren. 
 Aus diesem Grund ist es nicht ausgeschlossen, dass fehlerhafte oder unvollständige Daten enthalten sind.
                    
Zur Korrektur der Daten gibt es viele verschiedene Tools, die von Experten genutzt werden können.
 Um eine breitere Masse anzusprechen, entstand 2012 die \gls{WebApp} \kort{} im Rahmen einer Bachelorarbeit. 
 Mit \kort{} kann der Benutzer Aufträge lösen, welche zur Verbesserung der Daten in \brand{OpenStreetMap} beitragen. 
 Auf einer Kartenansicht werden die Aufträge, welche sich im Umfeld des Benutzers befinden, dargestellt. 
 Für das Eintragen einer Lösung wird man mit Punkten (sogenannte Koins) belohnt und kann so in der Rangliste aufsteigen.
 
Die ursprüngliche \gls{WebApp} wurde nicht weiter gepflegt und ist heute nicht mehr benutzbar. 
Allerdings konnte die bestehende Backend-Lösung weiterverwendet werden. 
\kort{} wurde in dieser Bachelorarbeit als native App für \brand{Android} und \brand{iOS} neu entwickelt. 
Die App ist in \brand{JavaScript} geschrieben und basiert auf dem \brand{React Native} Framework von \brand{Facebook}. 
Dabei wird \brand{JavaScript}-Code in \brand{Android}- und \brand{iOS}-Komponenten übersetzt und bietet dem Benutzer die Erfahrung einer native App.

Die entwickelte App wird in Kürze im \brand{Apple} App Store und im \brand{Google Play} Store veröffentlicht.


 