\chapter{Technologien}
\label{pd-technologien}

\section{React} 
\brand{React}\footnote{\url{https://facebook.github.io/react/}} \footnote{\url{https://github.com/facebook/react}} ist eine Open Source \brand{JavaScript} \gls{Library} und dient für die Implementation der View vom \gls{MVC}-Pattern. 
Die View besteht aus wiederverwendbaren Komponenten, die wiederum Komponenten beinhalten.
\brand{React} wird von \brand{Facebook}, \brand{Instagram} und von der Community entwickelt und gewartet. 
\cite{react}

% One-way data flow


Anstatt der \gls{DOM} nutzt \brand{React} die sogenannte \gls{Virtual DOM}.
Wie der Begriff schon sagt, wird mit einer Abstraktion der echten \gls{DOM} --- also mit einer virtuellen DOM --- kommuniziert.
Die komplette \gls{DOM}, also die Repräsentation der View vom HTML-Code, ist im lokalen Speicher abgelegt.\cite{virtual-dom} 
In der render()-Methode wird eine Beschreibung der DOM zurückgeliefert, die \brand{React} mit der lokalen Kopie der \gls{DOM} vergleicht.
Mit einem sehr effizienten Diffing-Algorithmus berechnet \brand{React} den Unterschied zwischen diesen  Versionen der \gls{DOM} und errechnet den schnellsten Weg um den Browser zu aktualisieren. 
\cite{react-virtual-dom}

% JSX


\section{React Native}
\brand{React Native}\footnote{\url{https://facebook.github.io/react-native/}} \footnote{\url{https://github.com/facebook/react-native}} 


