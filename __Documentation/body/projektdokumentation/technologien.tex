\chapter{Technologien}
\label{pd-technologien}


\section{React} 
\brand{React}\footnote{\url{https://facebook.github.io/react/}, \url{https://github.com/facebook/react}} wurde im März 2013 veröffentlicht\cite{react-release}, ist eine Open Source \brand{JavaScript} \gls{Library} und dient für die Implementation der View vom \gls{MVC}-Pattern. 
Die View besteht aus wiederverwendbaren Komponenten, die wiederum Komponenten beinhalten.
\brand{React} wird von \brand{Facebook}, \brand{Instagram} und von der Community entwickelt und gewartet.\cite{react}

Für \brand{React} wird \gls{JSX}, welches eine HTML ähnliche Syntax nutzt, zur Erstellung der Komponenten empfohlen. 
So lassen sich Komponenten-Bäume direkt mit \brand{JavaScript} erstellen. 
Anders formuliert können \brand{JavaScript}-Objekte mit einer HTML-Syntax erzeugt werden. 
Eine Hauptkomponente gibt seine Daten per Props an die Kind-Komponenten weiter (one-way-dataflow).\cite{react-data-flow}
\gls{JSX} wird nicht zwingend benötigt.\cite{jsx-syntax}

Anstatt der \gls{DOM} nutzt \brand{React} die sogenannte \gls{Virtual DOM}.
Wie der Begriff schon sagt, wird mit einer Abstraktion der echten \gls{DOM} -- also mit einer virtuellen DOM -- kommuniziert.
Die komplette \gls{DOM}, also die Repräsentation der View vom HTML-Code, ist im lokalen Speicher abgelegt.\cite{virtual-dom}
In der \inlinecode{render()}-Methode jeder \brand{React}-Klasse wird eine Beschreibung der DOM zurückgeliefert, die \brand{React} mit der lokalen Kopie der \gls{DOM} vergleicht.
Mit einem sehr effizienten Diffing-Algorithmus berechnet \brand{React} den Unterschied zwischen diesen  Versionen der \gls{DOM} und errechnet den schnellsten Weg um den Browser zu aktualisieren.\cite{react-virtual-dom}


\section{React Native}
Der Ansatz von \brand{React Native}\footnote{\url{https://facebook.github.io/react-native/}, \url{https://github.com/facebook/react-native}} ist \textit{learn once -- write anywhere}, das heisst, lerne eine Technologie und nutze sie für alle unterstützten Plattformen.\cite{react-native}

Am 26. März 2015 wurde \brand{React Native} erstmals für \brand{iOS} veröffentlicht.
Im Oktober 2015 kam \brand{Android} dazu.\cite{react-native-release}
Seit dem Release gibt es alle zwei Wochen eine neue Version. 
Durch diese häufigen Änderungen konnten sich noch keine Best Practices etablieren. 
Auch die meisten Open Source Projekte verfolgen eigene Implementationsansätze.

Eine Desktop Unterstützung für \brand{OSX} ist ebenfalls in Entwicklung\footnote{\url{https://github.com/ptmt/react-native-desktop}}. 
Und am 13. April 2016, an der \brand{Facebook} Developer Konferenz, kündigten \brand{Microsoft} und \brand{Facebook} den Support für die \brand{Universal Windows Platform} (\brand{UWP}) an.\cite{react-native-windows}


\subsection{Technische Details}
\brand{React Native} nutzt einen \brand{JavaScript}-Layer, beziehungsweise \brand{JavaScriptCore} als Engine, um den Code auszuführen.\cite{react-native-javascriptcore} 
Die native Funktionen werden auf die \brand{JavaScript}-Objekte oder Funktionen gemappt. 
Das Endprodukt ist also keine \gls{WebApp} für den Browser und wird auch nicht in native Code kompiliert. 
Ausserdem wird der \brand{JavaScript}-Code auf einem separaten Thread ausgeführt und nicht auf dem UI-Thread. 
Dadurch wirken zum Beispiel die Animationen sehr flüssig.\cite{react-native-javascript-thread}


\subsubsection{Native Module}
Damit ein \brand{iOS} Native-Modul in \brand{React Native} verwendet werden kann, muss das konkrete Modul das \newline\inlinecode{RCTBridgeModule}-Protokoll\footnote{RCT ist eine Abkürzung für ReaCT} implementieren und das Makro \newline\inlinecode{RCT\_EXPORT\_MODULE()} enthalten. 
Dem Makro kann auf der \brand{JavaScript}-Seite ein optionaler Namen als Parameter mitgegeben werden. 
Falls dieser Parameter fehlt, wird die Komponente auf \brand{JavaScript}-Seite nach dem \brand{Objective-C}-Klassennamen benannt. 
Das Gleiche Vorgehen gilt für ein \brand{Swift}-Modul. 
Genauere Hinweise sind in der \brand{React Native}-Dokumentation unter \textit{Native Modules} beschrieben.\cite{react-native-module-ios}

Für \brand{Android}-Module muss die gewünschte Klasse, die in \brand{React Native} verwendet werden möchte, von der Klasse \inlinecode{ReactContextBaseJavaModule} erben. 
Das erfordert dann die Implementation der \inlinecode{getName()}-Methode. 
Diese Methode soll dann den Namen, für die Verwendung der Komponente in \brand{React Native}, zurückgeben. 
Eine genaue Anleitung dazu ist ebenfalls in der \brand{React Native}-Dokumentation zu finden.\cite{react-native-module-android}

Mit diesem Feature lässt sich bereits vorhandenen native Code wiederverwenden. 


\subsection{Projektstruktur}



\subsection{Plattformunabhängigkeit}
Eine Stärke von \brand{React Native} ist die Plattformunabhängigkeit. 
Wenn keine spezifische \brand{Android}- oder \brand{iOS}-Komponenten verwendet werden, kann der Code für beide Plattformen genutzt werden.


\subsection{Community}
Die Community wirkt sehr zerstreut, denn viele Informationen sind in den Issues vom \brand{React Native} GitHub-Repository\footnote{\url{https://github.com/facebook/react-native/issues}} versteckt.
Ausserdem sind aktuell, am 15.06.2016, 733 offene Issues vorhanden.

\begin{itemize}
	\item Vorhanden ist eine öffentliche, aktive und hilfsbereite \brand{Facebook}-Gruppe\footnote{\url{https://www.facebook.com/groups/react.native.community/}}, mit derzeit ca. 3 500 Mitgliedern.
	\item Es gibt eine \brand{Stack-Overflow}-Kategorie\footnote{\url{http://stackoverflow.com/questions/tagged/react-native}} - leider nur mit wenigen Antworten und Lösungen.
	\item JS.coach\footnote{\url{https://js.coach/}} - listet viele Open Source Projekte auf.
	\item Übersicht über aktuelle Artikel und Blogposts: 
	\begin{itemize}
		\item reactnative.com\footnote{\url{http://www.reactnative.com/}}
		\item \brand{React Native} Newsletter\footnote{\url{http://reactnative.cc/}}
	\end{itemize}
	\item Es ist ebenfalls ein aktiver Subreddit\footnote{\url{https://www.reddit.com/r/reactnative}} vorhanden.
\end{itemize}


\subsection{Zusammenfassung}


% Gründe für React Native:
% Navigator-Komponente (url zu docs), wiederverwendbare open source komponenten, langfristig,
% immer mehr Apps im Showcase, das führt zu neuen Pull-requests und Beiträgen