\chapter{Technologien}
\label{pd-technologien}
In diesem Kapitel sind Informationen zur Funktionsweise der Technologien \brand{React} und \brand{React Native} dokumentiert. 
Das Unterkapitel zu \brand{React Native} beschreibt ebenfalls unsere Erfahrungen und Schlüsse.


\section{React} 
\brand{React}\footnote{\url{https://facebook.github.io/react/}, \url{https://github.com/facebook/react}} wurde im März 2013 veröffentlicht\cite{react-release}, ist eine Open Source \brand{JavaScript} \gls{Library} und dient für die Implementation der View des \gls{MVC}-Patterns. 
Die View besteht aus wiederverwendbaren Komponenten, die wiederum Komponenten beinhalten.
\brand{React} wird von \brand{Facebook}, \brand{Instagram} und von der Community entwickelt und gewartet.\cite{react}

Für \brand{React} wird \gls{JSX}, welches eine HTML ähnliche Syntax nutzt, zur Erstellung der Komponenten empfohlen. 
So lassen sich Komponenten-Bäume direkt mit \brand{JavaScript} erstellen. 
Anders formuliert können \brand{JavaScript}-Objekte mit einer HTML-Syntax erzeugt werden. 
Eine Hauptkomponente gibt seine Daten per Props an die Kind-Komponenten weiter (one-way-dataflow).\cite{react-data-flow}
\gls{JSX} wird nicht zwingend benötigt.\cite{jsx-syntax}

Anstatt der \gls{DOM} nutzt \brand{React} die sogenannte \gls{Virtual DOM}.
Wie der Begriff schon sagt, wird mit einer Abstraktion der echten \gls{DOM} -- also mit einer virtuellen DOM -- kommuniziert.
Das komplette \gls{DOM} (also die Repräsentation der View vom HTML-Code) ist im lokalen Speicher abgelegt.\cite{virtual-dom}
In der \inlinecode{render()}-Methode jeder \brand{React}-Klasse wird eine Beschreibung des DOM zurückgeliefert, die \brand{React} mit der lokalen Kopie der \gls{DOM} vergleicht.
Mit einem sehr effizienten Diffing-Algorithmus berechnet \brand{React} den Unterschied zwischen diesen  Versionen des \gls{DOM} und errechnet den schnellsten Weg um den Browser zu aktualisieren.\cite{react-virtual-dom}


\section{React Native}
Der Ansatz von \brand{React Native}\footnote{\url{https://facebook.github.io/react-native/}, \url{https://github.com/facebook/react-native}} ist \textit{learn once -- write anywhere}, das heisst, lerne eine Technologie und nutze sie für alle unterstützten Plattformen.\cite{react-native}
Dieses Kredo bezieht sich darauf, dass \brand{JavaScript} sehr weit verbreitet ist und man mit \brand{React Native} nicht extra eine neue Programmiersprache erlernen muss um Apps für \brand{Android} und \brand{iOS} zu schreiben.

Am 26. März 2015 wurde \brand{React Native} erstmals für \brand{iOS} veröffentlicht.
Im Oktober 2015 kam \brand{Android} dazu.\cite{react-native-release}
Seit dem Release gibt es alle zwei Wochen eine neue Version. 
Durch diese häufigen Änderungen konnten sich noch keine Best Practices etablieren. 
Auch die meisten Open Source Projekte verfolgen eigene Implementationsansätze.

Eine Desktop Unterstützung für \brand{OSX} ist ebenfalls in Entwicklung\footnote{\url{https://github.com/ptmt/react-native-desktop}}. 
Und am 13. April 2016 an der \brand{Facebook} Developer Konferenz kündigten \brand{Microsoft} und \brand{Facebook} den Support für die \brand{Universal Windows Platform} (\brand{UWP}) an.\cite{react-native-windows}


\subsection{Layout}
Alle \gls{GUI}-Komponenten befinden sich in sogenannten Containern. 
Ein Container wird durch eine \inlinecode{View}-Komponente definiert. 
Das Layout und die Gestaltung der Container und Komponenten wird mit \brand{Flexbox}\footnote{\url{https://facebook.github.io/react-native/docs/flexbox.html}. Dabei handelt es sich um einen Polyfill für die durch das \brand{W3C} spezifizierte \brand{Flexible Box}: \url{https://www.w3.org/TR/2016/CR-css-flexbox-1-20160526/}} geregelt. 

\subsection{Technische Details}
\brand{React Native} nutzt einen \brand{JavaScript}-Layer, beziehungsweise \brand{JavaScriptCore} als Engine, um den Code auszuführen.\cite{react-native-javascriptcore} 
Die native Funktionen werden auf die \brand{JavaScript}-Objekte oder Funktionen gemappt. 
Das Endprodukt ist also keine \gls{WebApp} für den Browser und wird auch nicht in native Code kompiliert. 
Ausserdem wird der \brand{JavaScript}-Code auf einem separaten Thread ausgeführt und nicht auf dem UI-Thread. 
Dadurch wirken zum Beispiel die Animationen sehr flüssig.\cite{react-native-javascript-thread}


\subsubsection{Native Module}
Damit ein \brand{iOS} Native-Modul in \brand{React Native} verwendet werden kann, muss das Modul das \inlinecode{RCTBridgeModule}-Protokoll\footnote{RCT ist eine Abkürzung für ReaCT} implementieren und das Makro \inlinecode{RCT\_EXPORT\_MODULE()} enthalten. 
Das Protokoll dient nur dazu, das Modul in einem Array zu speichern, damit es später von der Bridge gefunden werden kann. 
Wenn die \brand{JavaScript}-Seite der Bridge initialisiert ist, kann sie auf diese Daten zugreifen. 
Dem Makro kann auf der \brand{JavaScript}-Seite ein optionaler Name als Parameter mitgegeben werden. 
Falls dieser Parameter fehlt, wird die Komponente auf \brand{JavaScript}-Seite nach dem \brand{Objective-C}-Klassennamen benannt. 
Ein Ähnliches Vorgehen gilt für \brand{Swift}- und \brand{Android}-Module. 
Genauere Hinweise sind in der \brand{React Native}-Dokumentation unter \textit{Native Modules} beschrieben.\cite{react-native-module-ios}\cite{react-native-module-android}
Mit diesem Feature lässt sich bereits vorhandener native Code wiederverwenden. 
% Für \brand{Android}-Module muss die gewünschte Klasse, die in \brand{React Native} verwendet werden möchte, von der Klasse \inlinecode{ReactContextBaseJavaModule} erben. 
% Das erfordert dann die Implementation der \inlinecode{getName()}-Methode. 
% Diese Methode soll dann den Namen, für die Verwendung der Komponente in \brand{React Native}, zurückgeben. 


\subsection{Setup}
Die Entwicklungsumgebung lässt sich am schnellsten und einfachsten auf \brand{OS X} einrichten. 
\brand{Windows} und \brand{Linux} sind mittlerweile ebenfalls geeignet, was zu Beginn dieser Bachelorarbeit nicht der Fall war. 

Mit dem \brand{React Native}-CLI (Command Line Interface) können neue Projekte initialisiert werden und Projekte auf einem Emulator oder einem Gerät getestet werden. 
Das initialisierte Projekt enthält wiederum ein \brand{Android}- und ein \brand{iOS}-Projekt.
Diese Ordner enthalten den vorgegebenen Aufbau für Projekte der jeweiligen Plattformen.
Ausserdem befinden sich die beiden Dateien \inlinecode{index.android.js} und \inlinecode{index.ios.js} im generierten Ordner.
Diese werden beim Starten der App ausgeführt und verwenden weitere \brand{JavaScript} Module, welche sich bei uns im \inlinecode{js} Ordner befinden.

\subsection{Cross Platform}
Auch wenn es nicht der ursprüngliche Gedanke von \brand{React Native} war, ein Cross Platform Framework anzubieten -- \brand{React Native} hat ursprünglich nur \brand{iOS} unterstützt --, sind die Konzepte dafür sehr ausgereift.

Zum einen gibt es \brand{JavaScript} Module, die die native Module der beiden unterstützten Plattformen unter der selben API anbieten (\inlinecode{AsyncStorage}\footnote{\url{https://facebook.github.io/react-native/docs/asyncstorage.html}} oder \inlinecode{ListView}\footnote{\url{https://facebook.github.io/react-native/docs/listview.html}} sind Beispiele dafür).\newline
Zum anderen -- wenn ein plattformspezifischeres Verhalten erforderlich ist -- kann es sein, dass für die verschiedenen native Module auch unterschiedliche \brand{JavaScript} Module angeboten werden (als Beispiel dafür dient die Lade-Anzeige mit \inlinecode{ProgressBarAndroid}\footnote{\url{https://facebook.github.io/react-native/docs/progressbarandroid.html}} und \inlinecode{ActivityIndicatorIOS}\footnote{\url{https://facebook.github.io/react-native/docs/activityindicatorios.html}}).
In diesem Fall gibt es zwei Möglichkeiten, die Module im Client Code zu verwenden.
Die eine Möglichkeit ist, in einem \inlinecode{if}-Statement zu überprüfen, auf welcher Plattform die App läuft (\inlinecode{if (Platform.OS === 'ios')}) und das passende Modul einzusetzen.\footnote{Es gibt auch noch weitere Varianten: \url{https://facebook.github.io/react-native/docs/platform-specific-code.html}}.
Die andere Möglichkeit ist, dass der Client Code auf zwei verschiedene Dateien aufgeteilt wird, die durch die Endung \inlinecode{.android.js} respektive \inlinecode{.ios.js} unterschieden werden.

Diese Konzepte bauen auf dem modularen Ansatz von \brand{React} auf und bieten die Freiheit, auf elegante Weise eine plattformunabhängige App zu entwickeln.

\subsection{Community}
Die Community ist auf mehrere Portale verstreut, wobei viele wichtige Informationen in den Issues des \brand{React Native} GitHub-Repositorys\footnote{\url{https://github.com/facebook/react-native/issues}} verborgen sind.
Ausserdem sind aktuell (am 16.06.2016) 741 offene Issues vorhanden -- am 29.04.2016 waren es noch rund 500.

\begin{itemize}
	\item Vorhanden ist eine öffentliche, aktive und hilfsbereite \brand{Facebook}-Gruppe\footnote{\url{https://www.facebook.com/groups/react.native.community/}} mit derzeit ca. 3 500 Mitgliedern.
	\item Es gibt eine \brand{Stack-Overflow}-Kategorie\footnote{\url{http://stackoverflow.com/questions/tagged/react-native}} -- leider mit nur wenigen Antworten und Lösungen.
	\item JS.coach\footnote{\url{https://js.coach/}} - listet viele Open Source Projekte auf.
	\item Übersicht über aktuelle Artikel und Blogposts: 
	\begin{itemize}
		\item reactnative.com\footnote{\url{http://www.reactnative.com/}}
		\item \brand{React Native} Newsletter\footnote{\url{http://reactnative.cc/}}
	\end{itemize}
	\item Es ist ebenfalls ein aktiver Subreddit\footnote{\url{https://www.reddit.com/r/reactnative}} vorhanden.
\end{itemize}


\subsection{Erfahrungen}
Eine Stärke von \brand{React Native} ist die Plattformunabhängigkeit. 
Wenn keine spezifischen \brand{Android}- oder \brand{iOS}-Komponenten verwendet werden, kann der Code für beide Plattformen genutzt werden.

Die grössten Hürden von \brand{React Native} sind das Erlernen von \brand{React} und \brand{Flexbox}. 
Denn \brand{Flexbox} für \brand{React Native} unterscheidet sich in vielen Details vom \brand{Flexbox} für Webseiten. 
%Ein wichtiges Detail ist, dass bei \brand{Flexbox} für \brand{React Native} Komponenten nur übereinander (im Z-Index) gerendert werden können, wenn die Anordnung und Reihenfolge der Komponenten in den Containern stimmt. 
Einerseits ist das Grundkonzept für das Layout bei vielen verschachtelten Komponenten schwer zu verstehen. 
Selbst Die Umsetzung von ganz simplen Views ist am Anfang schwer und frustrierend. 
Andererseits wirkt es, nachdem viele Stunden in das Lernen investiert wurden, doch konsistent und praktisch. 
Dazu kommt, dass durch \brand{Flexbox} die definierten Styles gut wiederverwendbar sind. 
Das Gleiche gilt für jegliche Komponenten, die auch gut durch Inspiration aus anderen Open Source Projekten erstellt werden können.
Durch die wachsende Community gibt es immer mehr solcher Projekte, Pull-Requests und Beiträge an \brand{React Native} selber. 
Das macht \brand{React Native} langfristig gesehen zu einer immer besseren und wichtigeren Technologie. 

Die Umstellung von der native Entwicklung zu \brand{React Native} ist Anfangs auch schwer. 
Vor allem wenn Fehler auf \brand{React Native}-Seite existieren, die bei der native Entwicklung nicht durch unschöne Workarounds übersteuert werden müssten. 
Mit der Zeit zeigen sich aber die grossen Vorteile von \brand{React}-\brand{JavaScript}. 
Jede einzelne Komponente wird allein durch ihren State kontrolliert. 
Je nach State kann die Darstellung der Component durch die eigene \inlinecode{render()}-Methode kontrolliert und verändert werden. 

Ein angenehmes Feature für die Entwicklung ist das Live- und das Hot-Reload Feature. 
Nur beim ersten Ausführen der App muss ein Build erstellt werden.
Dank dem gleichzeitigen Starten des \gls{Packager}s kann neu geschriebener \brand{JavaScript}-Code direkt in  die auf dem Emulator oder dem Smartphone laufende App geladen werden.

Der App-Showcase der offiziellen \brand{React Native}-Dokumentation wächst stetig\footnote{\url{https://facebook.github.io/react-native/showcase.html}}.
Wir gehen davon aus, dass \brand{React Native} zusammen mit \brand{Xamarin}\footnote{\url{https://www.xamarin.com/}} in Zukunft eine wichtige Rolle in der mobilen Cross-Platform-Entwicklung einnehmen wird.