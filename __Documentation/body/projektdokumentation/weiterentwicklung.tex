\chapter{Weiterentwicklung}
\label{pd-weiterentwicklung}

\kort{} hat ein grosses Potenzial um weiterentwickelt zu werden.
Dieses umfasst vor allem zwei Bereiche:

Wie kann \kort{} besser dazu beitragen, \brand{OSM} Daten zu verbessern?

Und wie kann der Benutzer mithilfe von Konzepten der Gamification weiter motiviert werden, zur Datenpflegung beizutragen?

Wir haben uns Gedanken dazu gemacht und hier zusammengefasst, wie \kort{} weiter optimiert werden könnte.
Das Unterkapitel \hyperref[pd-weiterentwicklung-vorgehen]{Vorgehen} enthält Arbeiten, die beim derzeitigen Stand noch verbessert werden müssen. 
Weiterhin sind Ideen aufgelistet, die keine grösseren Änderungen am Backend mit sich bringen.

\section{Vorgehen}
\label{pd-weiterentwicklung-vorgehen}
Ein sehr wichtiges Feature, das bisher noch nicht umsetzbar war, ist das \brand{OSM}-Login.
Neben den Änderungen am Backend, damit dieses Feature überhaupt realisierbar ist, gibt es folgende Schritte zu erledigen:

\begin{enumerate}
	\item App bei \brand{OSM} mit Callback-URL registrieren
	\begin{itemize}
		\item Als Callback-URL könnte zum Beispiel 'http://www.kort.ch/' verwendet werden.
	\end{itemize}
	\item Request-Token-URL aufrufen, um das OAuth-Token und das Token-Secret zu erhalten
	\item Authorize-URL mit dem OAuth-Token aufrufen, damit sich der Benutzer auf der \brand{OSM}-Seite einloggen kann
	\item WebView-Komponente öffnen, um auf die Callback-URL zu warten 
	\item Wenn sich der Benutzer eingeloggt hat, wird er zur Callback-Seite, mit dem OAuth-Token und dem OAuth-Verifier in der URL, umgeleitet.
	\item WebView schliessen
	\item Request Access-Token-URL
	\item dem Backend das Token zur Überprüfung senden
\end{enumerate}

Dieses Vorgehen wurde noch nicht getestet und dient nur als Idee zur möglichen Umsetzung.
Die korrekten Request-URLs können aus der Dokumentation im \brand{OSM}-Wiki\footnote{\url{http://wiki.openstreetmap.org/wiki/OAuth}} entnommen werden.

Als nächstes müsste für den \gls{Gamification}-Ansatz das Design weiter verbessert werden. 
Dafür wäre zuerst eine Änderung am Backend geplant, um die Validationen endgültig abzuschaffen, indem sie als normale Missionen gezählt werden. 
Dann wäre es auch wieder möglich korrekte Badges für den Missionen-Counter anzuzeigen. 

Durch den Einsatz von Farben oder einem Hintergrundbild beim Login-Screen würde das Design einen Spieler besser ansprechen. 
Passend dazu könnten die verwendeten Bilder, Icons und Marker einheitlich gestaltet und erneuert werden. 

Meldungen, die dem Benutzer die Anzahl gewonnener Koins anzeigen, erscheinen momentan im Vollbildmodus. 
Das liegt an einem Fehler der eingesetzten Navigations-Komponente, die keine transparenten Hintergründe zulässt. 
In Zukunft könnte das aber noch behoben werden. 
Dann wäre es eleganter, Meldungen in einem Fenster zu zeigen.

\subsubsection{GUI-Arbeiten}
\begin{itemize}
	\item Map-Marker ersetzen
	\item Textfeld, um den Benutzernamen anzupassen, anbieten
	\item 
\end{itemize}

Um die Funktionalität der \gls{WebApp} komplett umgesetzt zu haben fehlt nur noch die zweite Highscore-Ansicht und die Karte, die beim Lösen einer Mission angezeigt wird.

Gerne hätten wir noch Tests der Komponenten durchgeführt, doch die Priorität dafür war zu tief und die Zeit zu knapp. 
Dazu haben wir für die Zukunft die Enzyme\footnote{\url{http://airbnb.io/enzyme/}}-Testing-API (für Komponenten-Tests) und das Mocha\footnote{\url{https://github.com/mochajs/mocha}}-\gls{Framework} (um die Tests laufen zu lassen) evaluiert.

\section{Realistische Arbeiten}
\label{pd-weiterentwicklung-realistisch}
Punkte, die \kort{} attraktiver machen würden:

\begin{itemize}
	\item Benutzerlogin mit weiteren \gls{OAuth}-Diensten erweitern (z.B. \brand{Twitter}, \brand{GitHub})
	\item \gls{Gamification}
	\begin{itemize}
		\item von gesammelten \emph{Koins} abhängige Levels einführen (z.B. bestimmte Fehlertypen erst ab fortgeschrittenem Level anzeigen, Avatars, Levelbezeichnungen)
		\item verschiedene Schwierigkeitsstufen
		\item Einbindung in \brand{Game Center} der jeweiligen Plattform
		\item weitere Badges einführen (viele Ideen finden sich hier: \url{https://wiki.openstreetmap.org/wiki/Badges}, zum Beispiel auch Badges für Spielertypen)
		\item verschiedene Highscores anzeigen (zum Beispiel zeitlich oder Regional begrenzt, nach Fehlertypen kategorisiert, schnellste Aufsteiger)
		\item zusätzliche Berechtigungen für erfahrene Benutzer (für den langfristigen Erfolg)
	\end{itemize}
	\item neue realistische Fehlertypen:\footnote{\url{https://github.com/kort/kort/issues/81}}
	\begin{itemize}
		\item Hausnummern einfügen
		\item Stockwerk-Anzahl einfügen
		\item Einbahnstrassen erfassen
		\item Öffnungszeiten von öffentlichen Gebäuden festhalten
	\end{itemize}
	\item weniger realistische Fehlertypen:
		\begin{itemize}
			\item Kreisel erfassen
			\item Bushaltestellen von \brand{DIDOK}\footnote{\url{https://didok.osm.ch/}} erfassen
		\end{itemize}
	\item Erkennen von Benutzern, die nicht sorgfältig validieren\footnote{\url{https://github.com/kort/kort/issues/7}}
	\item ausführliche Statistiken für individuelle Benutzer\footnote{\url{https://github.com/kort/kort/issues/71}}
	\item Aufträge aus \brand{wheelmap}\footnote{\url{http://wheelmap.org}}	einfügen
	\item Offline-Fähigkeit (offline Maps für \brand{React Native} wären erforderlich)
	\item wenn Aufträge zum Beispiel drei Mal nicht gelöst werden können, soll eine \brand{OSM}-Notiz generiert werden
\end{itemize}

\subsubsection{Unrealistische Arbeiten}
Punkte, die für \kort{} als weniger geeignet empfunden wurden:

\begin{itemize}
	\item Erweitern der Verifizierung mit der Möglichkeit, ein Foto als Beweis hochzuladen
	\begin{itemize}
	  \item \emph{Begründung:} Aspekte des Datenschutzes bergen ein gewisses Risiko. Benutzer müssten für das Hochladen von Bildern zusätzliche Bedingungen akzeptieren.
	\end{itemize}
	\item standortunabhängige Aufgaben lösen (Gefahr von Couch Mapping)
	\begin{itemize}
	  \item \emph{Begründung:} Es ist ein Anliegen der \brand{OSM} Community, dass die Mapper vor Ort sein sollen um Aufträge zu lösen. 
	\end{itemize}
\end{itemize}
