\chapter{Anforderungsspezifikation}
\label{pd-anforderungsspezifikation}

\section{Anforderungen an die Arbeit} 
Die \hyperref[pm-team]{Autoren} hatten im Vorfeld der Arbeit wenige Kenntnisse über Webtechnologien und insbesondere gar keine Erfahrung mit \brand{JavaScript}.
Insofern war der machbare Umfang des Projekts schwer absehbar.
In Abstimmung mit dem \hyperref[pm-team]{Betreuer} und dem \hyperref[pm-team]{Projektpartner} wurde deshalb festgelegt, dass der Fokus auf dem Frontend liegt, so dass man sich nicht auch noch in die Technologien des Backends einarbeiten muss.

In der Anforderungsanalyse sind viele Aufgaben erkannt worden, die in diesem Projekt bearbeitet werden könnten. 
Im Rahmen dieser Arbeit ist nur ein Bruchteil davon umsetzbar.
Zum Einen aus zeitlichen Gründen und zum Anderen, weil Anpassungen am Backend nötig wären.
All diese Anforderungen sind in \emph{Muss}, \emph{Soll}, \emph{Kann}, \emph{zukünftige Arbeiten} und \emph{abgewiesene Arbeiten} unterteilt.

\subsection{Muss}
\begin{itemize}
	\item neuer Kort-Client als native \brand{Android}-App
	\item Ablösung des Validationsmechanismus
	\item Erfahrungsbericht mit Hinweisen zu Tutorials zu React Native
	\item die vom Studiengang geforderten Lieferobjekte: Dokumentation, Management Summary, Abstract, Poster, Präsentation mit Stellwand, Zwischenpräsentation, Schlusspräsentation
\end{itemize}

\subsection{Soll}
\begin{itemize}
	\item Kort-Client als native \brand{iOS}-App
\end{itemize}

\subsection{Kann}
\begin{itemize}
	\item neue Funktion: Promotions anzeigen
	\item Kurzvideo (zur Instruktion und Promotion)
\end{itemize}


\section{Use Cases}


\section{System-Sequenzdiagramme}


