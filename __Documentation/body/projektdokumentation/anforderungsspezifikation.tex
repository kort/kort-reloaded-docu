\chapter{Anforderungsspezifikation}
\label{pd-anforderungsspezifikation}

%Enthält folgende mögliche Unterkapitel:
% Rahmenbedingungen (wenn nicht schon oben abgehandelt)
% Anwendungsfalldiagramm
% Hauptanwendungsfall
% Funktionale Anforderungen
% Nicht-Funktionale Anforderungen
% Weitere: Aktivitätsdiagramme, Fallbeispiele, Szenarien, Prototypen...

\section{Anforderungen an die Arbeit} 
Aufgrund der Probleme der aktuellen \kort{}-\gls{WebApp} wird eine neue native App mit \brand{React-Native} erstellt. 
Da wir uns neu in JavaScript und React.js einarbeiten mussten, entschieden wir den Fokus nur auf \brand{Android} zu setzen.
Die "News", ein Tab in der derzeitigen \kort{}-\gls{WebApp}, fallen weg.

%TODO: Entscheid zur min/max Android SDK-Version / API-Level

\subsection{Prototyp}
Die Anforderungen an den Prototyp sind: 

\begin{itemize}
	\item Karten Ansicht
	\item Missionen anzeigen (mit Symbolen)
	\item Eigenes Profilseite
	\item Badges
	\item Highscore
	\item About
\end{itemize}

%II 2.2 Use Cases (Success Scenario / Success Diagramm)
\section{Use Cases}


\section{System-Sequenzdiagramme}


\section{Weitere Funktionen}


\section{Nicht-funktionale Anforderungen}


\section{Detailspezifikation}

