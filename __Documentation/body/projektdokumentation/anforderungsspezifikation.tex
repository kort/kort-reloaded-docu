\chapter{Anforderungsspezifikation}
\label{pd-anforderungsspezifikation}

%Enthält folgende mögliche Unterkapitel:
% Rahmenbedingungen (wenn nicht schon oben abgehandelt)
% Anwendungsfalldiagramm
% Hauptanwendungsfall
% Funktionale Anforderungen
% Nicht-Funktionale Anforderungen
% Weitere: Aktivitätsdiagramme, Fallbeispiele, Szenarien, Prototypen...

\section{Anforderungen an die Arbeit} 
Die \hyperref[pm-rollen]{Autoren} hatten im Vorfeld der Arbeit wenige Kenntnisse über Webtechnologien und gar keine Erfahrung damit – insbesondere mit \brand{JavaScript}.
Insofern war der machbare Umfang des Projekts schwer absehbar.
In Abstimmung mit dem \hyperref[pm-rollen]{Betreuer} und dem \hyperref[pm-rollen]{Projektpartner} wurde deshalb festgelegt, dass der Fokus auf dem Frontend liegt, so dass man sich nicht auch noch in die Technologien des Backends einarbeiten muss.

In der Anforderungsanalyse konnten viele Aufgaben erkannt werden, mit welchen man sich beschäftigen könnte. Im Rahmen dieser Arbeit ist nur ein Bruchteil davon umsetzbar – zum einen aus zeitlichen Gründen, zum anderen, weil Anpassungen am Backend nötig wären.
All diese Anforderungen wurden in \emph{Muss}, \emph{Soll}, \emph{Kann}, \emph{zukünftige Arbeiten} und \emph{abgewiesene Arbeiten} unterteilt.

\subsection{Muss}
\begin{itemize}
	\item Native App für \brand{Android} mit derselben Funktionalität wie der \gls{WebApp}\footnote{\url{play.kort.ch}}
\end{itemize}

\subsection{Soll}
\begin{itemize}
	\item Native App für \brand{iOS}
\end{itemize}

\subsection{Kann}
\begin{itemize}
	\item Social Media einbinden
	\item auf \brand{MapRoulette}\footnote{\url{http://maproulette.org}} oder de \brand{Vespucci OSM Editor}\footnote{\url{https://play.google.com/store/apps/details?id=de.blau.android}} verweisen wenn keine Aufgaben mehr zur Verfügung stehen
	\item Instruktions- und Promotionsvideo über die App
\end{itemize}

\subsection{Zukünftige Arbeiten}
\begin{itemize}
	\item Benutzerlogin mit weiteren \brand{OAuth} Diensten erweitern (z.B. \brand{Twitter}, \brand{GitHub})
	\item \gls{Gamification}
	\begin{itemize}
		\item zeitlich begrenzte Promotionen (z.B. bei Events), bei denen man zusätzliche Punkte sammeln kann; sollen über Datenbank erstellt werden können
		\item von gesammelten \emph{Koins} abhängige Levels einführen (z.B. bestimmte Fehlertypen erst ab fortgeschrittenem Level anzeigen, Avatars, Levelbezeichnungen)
		\item Einbindung in \brand{Game Center} der jeweiligen Plattform
		\item weitere Badges einführen (Ideen finden sich hier: \url{https://wiki.openstreetmap.org/wiki/Badges})
		\item verschiedene Highscores anzeigen (z.B. zeitlich oder regional begrenzte, nach Fehlertypen kategorisiert, schnellste Aufsteiger)
	\end{itemize}
	\item neue Fehlertypen\footnote{\url{https://github.com/kort/kort/issues/81}}
	\begin{itemize}
		\item realistisch
		\begin{itemize}
			\item Hausnummern
			\item Stockwerke
			\item Einbahnstrassen erfassen
			\item Öffnungszeiten von öffentlichen Gebäuden
		\end{itemize}
		\item weniger realistisch
		\begin{itemize}
			\item Kreisel erfassen
			\item Bushaltestellen von \brand{DIDOK}\footnote{\url{https://didok.osm.ch/}}
		\end{itemize}
	\end{itemize}
	\item Erkennen von Benutzern, die nicht sorgfältig validieren\footnote{\url{https://github.com/kort/kort/issues/7}}
	\item ausführliche Statistiken für individuelle Benutzer\footnote{\url{https://github.com/kort/kort/issues/71}}
	\item Funktionalität um als Benutzer neue Fehlertypen erfassen zu können
	\item Ersetzen der bisherigen Validierung durch anderen Prüfmechanismus (z.B. nur noch Aufträge anzeigen – wenn dreimal die gleiche Antwort angegeben wurde gilt der Auftrag als verifiziert)
	\item \brand{KeepRight} durch \brand{Osmose} ersetzen
	\item Aufträge aus \brand{wheelmap}\footnote{\url{http://wheelmap.org}}	 einfügen
	\item Offline-Fähigkeit (offline Maps für \brand{React Native} wären erforderlich)
	\item einführende Hilfestellung für Neuanwender
	\item wenn Aufträge nicht gelöst werden können (z.B. drei Mal) eine \brand{OSM}-Notiz generieren oder manuell eintragen
\end{itemize}

\subsection{Abgewiesene Arbeiten}
\begin{itemize}
	\item Erweitern der Verifizierung mit der Möglichkeit, ein Foto als Beweis hochzuladen
	\item Übertragungssicherheit zwischen Webserver und Datenbank durch Verwendung von HTTPS erhöhen
	\item standortunabhängig Aufgaben lösen (Gefahr von Couch Crowdsourcing)
\end{itemize}

%TODO: Entscheid zur min/max Android SDK-Version / API-Level

%II 2.2 Use Cases (Success Scenario / Success Diagramm)
\section{Use Cases}


\section{System-Sequenzdiagramme}


\section{Detailspezifikation}

