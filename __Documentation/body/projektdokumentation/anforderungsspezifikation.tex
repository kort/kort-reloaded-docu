\chapter{Anforderungsspezifikation}
\label{pd-anforderungsspezifikation}

%Enthält folgende mögliche Unterkapitel:
% Rahmenbedingungen (wenn nicht schon oben abgehandelt)
% Anwendungsfalldiagramm
% Hauptanwendungsfall
% Funktionale Anforderungen
% Nicht-Funktionale Anforderungen
% Weitere: Aktivitätsdiagramme, Fallbeispiele, Szenarien, Prototypen...

\section{Anforderungen an die Arbeit} 
Die \hyperref[pm-rollen]{Autoren} hatten im Vorfeld der Arbeit wenige Kenntnisse über Webtechnologien und gar keine Erfahrung damit – insbesondere mit \brand{JavaScript}.
Insofern war der machbare Umfang des Projekts schwer absehbar.
In Abstimmung mit dem \hyperref[pm-rollen]{Betreuer} und dem \hyperref[pm-rollen]{Projektpartner} wurde deshalb festgelegt, dass der Fokus auf dem Frontend liegt, so dass man sich nicht auch noch in die Technologien des Backends einarbeiten muss.

In der Anforderungsanalyse konnten viele Aufgaben erkannt werden, mit welchen man sich beschäftigen könnte. Im Rahmen dieser Arbeit ist nur ein Bruchteil davon umsetzbar – zum einen aus zeitlichen Gründen, zum anderen, weil Anpassungen am Backend nötig wären.
All diese Anforderungen wurden in \emph{Muss}, \emph{Soll}, \emph{Kann}, \emph{zukünftige Arbeiten} und \emph{abgewiesene Arbeiten} unterteilt.

\subsection{Muss}
\begin{itemize}
	\item Native App für \brand{Android} mit derselben Funktionalität wie der \gls{WebApp}\footnote{\url{play.kort.ch}}
	\item Validationen werden ersetzt, Aufträge sollen nur noch in Form von Missionen dargestellt werden
\end{itemize}

\subsection{Soll}
\begin{itemize}
	\item Native App für \brand{iOS}
\end{itemize}

\subsection{Kann}
\begin{itemize}
	\item zeitlich begrenzte Promotionen (z.B. bei Events), bei denen man zusätzliche Punkte sammeln kann; sollen über Datenbank erstellt werden können
	\item Instruktions- und Promotionsvideo über die App
	\item Social Media einbinden
	\item auf \brand{MapRoulette}\footnote{\url{http://maproulette.org}} oder de \brand{Vespucci OSM Editor}\footnote{\url{https://play.google.com/store/apps/details?id=de.blau.android}} verweisen wenn keine Aufgaben mehr zur Verfügung stehen
\end{itemize}

%TODO: Entscheid zur min/max Android SDK-Version / API-Level

%II 2.2 Use Cases (Success Scenario / Success Diagramm)
\section{Use Cases}


\section{System-Sequenzdiagramme}


\section{Detailspezifikation}

