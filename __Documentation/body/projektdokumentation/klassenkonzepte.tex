\section{Klassenkonzepte}
\label{pd-klassenkonzepte}
Die Klassen können folgendermassen strukturiert werden: Actions, Components, Data, Dispatcher, DTO, Stores.\newline
\newline
\textbf{\hyperref[pd-flux-actions]{Actions}} werden von den Components benötigt, um Aktionen in den Stores auszulösen, Daten vom \gls{Backend} zu laden oder Daten an das \gls{Backend} zu senden.\newline
\newline
\textbf{Components} entsprechen den \hyperref[pd-flux-views]{Views} der \brand{Flux}-Architektur.
Component Klassen erweitern \inlinecode{React.Component}\footnote{\url{https://facebook.github.io/react/docs/component-api.html}} und implementieren mindestens die \inlinecode{render()} Methode\footnote{\url{https://facebook.github.io/react/docs/component-specs.html}}. 
In der \inlinecode{render()} Methode wird in \gls{JSX} Syntax definiert, wie die Component dargestellt werden soll.\newline
Eine Component kann andere Components wiederverwenden,
Dadurch wird sowohl ein hierarchischer Aufbau der View, als auch ein modularer Einsatz von Components  ermöglicht.\newline
Ein weiteres Konzept, das von Components realisiert wird, ist die Unterscheidung von Property und State.
Eine Property repräsentiert eine sich nicht ändernde Eigenschaft einer Component, während der State den aktuellen Zustand ausdrückt.
Properties werden vom Owner\footnote{Die Owner-Ownee-Beziehung wird unter \emph{Ownership} beschrieben: \url{https://facebook.github.io/react/docs/multiple-components.html}} gesetzt und können durch den Ownee als \inlinecode{propTypes} deklariert werden.
Wird eine State Variable neu gesetzt, führt dies zu einem erneuten Aufruf der \inlinecode{render()} Methode.\newline
Components dürfen den Zustand -- der in den Stores gehalten wird -- nie direkt manipulieren, sondern müssen dies über Actions auslösen.
In den Callbacks, welche sie bei den Stores registriert haben, können sie dann die Getter aufrufen um den neuen Zustand zu erhalten.
Components sollen nicht mehr über den Applikationszustand wissen als für ihre Ansicht nötig ist.
Die Logik sollte also so weit möglich in den Stores umgesetzt sein.\newline
\newline
