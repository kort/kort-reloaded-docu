\chapter{Resultate und Weiterentwicklung}
\label{pd-resultate}

\section{Resultate}

\section{Möglichkeiten der Weiterentwicklung}
\kort{} hat ein grosses Potenzial um weiterentwickelt zu werden.
Dieses umfasst vor allem zwei Bereiche:
Wie kann \kort{} besser dazu beitragen, \brand{OpenStreetMap} Daten zu verbessern?
Und wie kann der Benutzer mithilfe von Konzepten der Gamification weiter motiviert werden, zur Datenpflegung beizutragen?\newline
Wir haben uns Gedanken dazu gemacht und hier zusammengefasst, womit \kort{} weiter optimiert werden könnte:
\begin{itemize}
  \item \kort{} als Web App (umgesetzt mit React)
	\item Benutzerlogin mit weiteren \brand{OAuth} Diensten erweitern (z.B. \brand{Twitter}, \brand{GitHub})
	\item \gls{Gamification}
	\begin{itemize}
		\item von gesammelten \emph{Koins} abhängige Levels einführen (z.B. bestimmte Fehlertypen erst ab fortgeschrittenem Level anzeigen, Avatars, Levelbezeichnungen)
		\item Einbindung in \brand{Game Center} der jeweiligen Plattform
		\item weitere Badges einführen (Ideen finden sich hier: \url{https://wiki.openstreetmap.org/wiki/Badges})
		\item verschiedene Highscores anzeigen (z.B. zeitlich oder regional begrenzte, nach Fehlertypen kategorisiert, schnellste Aufsteiger)
	\end{itemize}
	\item neue Fehlertypen\footnote{\url{https://github.com/kort/kort/issues/81}}
	\begin{itemize}
		\item realistisch
		\begin{itemize}
			\item Hausnummern
			\item Stockwerke
			\item Einbahnstrassen erfassen
			\item Öffnungszeiten von öffentlichen Gebäuden
		\end{itemize}
		\item weniger realistisch
		\begin{itemize}
			\item Kreisel erfassen
			\item Bushaltestellen von \brand{DIDOK}\footnote{\url{https://didok.osm.ch/}}
		\end{itemize}
	\end{itemize}
	\item Erkennen von Benutzern, die nicht sorgfältig validieren\footnote{\url{https://github.com/kort/kort/issues/7}}
	\item ausführliche Statistiken für individuelle Benutzer\footnote{\url{https://github.com/kort/kort/issues/71}}
	\item Funktionalität um als Benutzer neue Fehlertypen erfassen zu können
	\item Ersetzen der bisherigen Validierung durch anderen Prüfmechanismus (z.B. nur noch Aufträge anzeigen – wenn dreimal die gleiche Antwort angegeben wurde gilt der Auftrag als verifiziert)
	\item \brand{KeepRight} durch \brand{Osmose} ersetzen
	\item Aufträge aus \brand{wheelmap}\footnote{\url{http://wheelmap.org}}	 einfügen
	\item Offline-Fähigkeit (offline Maps für \brand{React Native} wären erforderlich)
	\item einführende Hilfestellung für Neuanwender
	\item wenn Aufträge nicht gelöst werden können (z.B. drei Mal) eine \brand{OSM}-Notiz generieren oder manuell eintragen
	\item Übertragungssicherheit zwischen Webserver und Datenbank durch Verwendung von HTTPS erhöhen
\end{itemize}

\subsection{Unrealistische Arbeiten}
\begin{itemize}
	\item Erweitern der Verifizierung mit der Möglichkeit, ein Foto als Beweis hochzuladen
	\begin{itemize}
	  \item \emph{Begründung:} Aspekte des Datenschutzes bergen ein gewisses Risiko. Benutzer müssten fürs Hochladen von Bildern zusätzliche Bedingungen akzeptieren.
	\end{itemize}
	\item standortunabhängig Aufgaben lösen (Gefahr von Couch Crowdsourcing)
	\begin{itemize}
	  \item \emph{Begründung:} Es ist ein Anliegen der \brand{OpenStreetMap} Community, dass die Mapper vor Ort sein sollen um Aufträge zu lösen. 
	\end{itemize}
\end{itemize}
