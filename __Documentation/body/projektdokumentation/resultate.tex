\chapter{Resultate und Weiterentwicklung}
\label{pd-resultate}

\section{Resultate}

\section{Weiterentwicklung}

\kort{} hat ein grosses Potenzial um weiterentwickelt zu werden.
Dieses umfasst vor allem zwei Bereiche:
Wie kann \kort{} besser dazu beitragen, \brand{OpenStreetMap} Daten zu verbessern?
Und wie kann der Benutzer mithilfe von Konzepten der Gamification weiter motiviert werden, zur Datenpflegung beizutragen?\newline
Wir haben uns Gedanken dazu gemacht und hier zusammengefasst, wie \kort{} weiter optimiert werden könnte:
\begin{itemize}
	\item Benutzerlogin mit weiteren OAuth-Diensten erweitern (z.B. \brand{Twitter}, \brand{GitHub})
	\item \gls{Gamification}
	\begin{itemize}
		\item von gesammelten \emph{Koins} abhängige Levels einführen (z.B. bestimmte Fehlertypen erst ab fortgeschrittenem Level anzeigen, Avatars, Levelbezeichnungen)
		\item Einbindung in \brand{Game Center} der jeweiligen Plattform
		\item weitere Badges einführen (viele Ideen finden sich hier: \url{https://wiki.openstreetmap.org/wiki/Badges})
		\item verschiedene Highscores anzeigen (z.B. zeitlich oder regional begrenzte, nach Fehlertypen kategorisiert, schnellste Aufsteiger)
	\end{itemize}
	\item neue realistische Fehlertypen:\footnote{\url{https://github.com/kort/kort/issues/81}}
	\begin{itemize}
		\item Hausnummern einfügen
		\item Stockwer-Anzahl einfügen
		\item Einbahnstrassen erfassen
		\item Öffnungszeiten von öffentlichen Gebäuden festhalten
	\end{itemize}
	\item weniger realistische Fehlertypen:
		\begin{itemize}
			\item Kreisel erfassen
			\item Bushaltestellen von \brand{DIDOK}\footnote{\url{https://didok.osm.ch/}} erfassen
		\end{itemize}
	\item Erkennen von Benutzern, die nicht sorgfältig validieren\footnote{\url{https://github.com/kort/kort/issues/7}}
	\item ausführliche Statistiken für individuelle Benutzer\footnote{\url{https://github.com/kort/kort/issues/71}}
	\item Aufträge aus \brand{wheelmap}\footnote{\url{http://wheelmap.org}}	einfügen
	\item Offline-Fähigkeit (offline Maps für \brand{React Native} wären erforderlich)
	\item wenn Aufträge nicht gelöst werden können (z.B. drei Mal) soll eine \brand{OpenStreetMap}-Notiz generiert werden
\end{itemize}

\subsection{Unrealistische Arbeiten}
\begin{itemize}
	\item Erweitern der Verifizierung mit der Möglichkeit, ein Foto als Beweis hochzuladen
	\begin{itemize}
	  \item \emph{Begründung:} Aspekte des Datenschutzes bergen ein gewisses Risiko. Benutzer müssten für das Hochladen von Bildern zusätzliche Bedingungen akzeptieren.
	\end{itemize}
	\item standortunabhängige Aufgaben lösen (Gefahr von Couch Mapping)
	\begin{itemize}
	  \item \emph{Begründung:} Es ist ein Anliegen der \brand{OpenStreetMap} Community, dass die Mapper vor Ort sein sollen um Aufträge zu lösen. 
	\end{itemize}
\end{itemize}
