\chapter{Implementation}
\label{pd-implementation}

\section{Manuelle Testverfahren}
Komponenten, die einzeln und manuell getestet werden, sind:
\begin{itemize}
	\item Login-Möglichkeiten
	\item Badge Verleihung
	\item Nur erlaubte Aufträge und Überprüfungen (fixer Umkreis)
	\item Standort Aktualisierung (Aktualisierung von Aufträgen und Überprüfungen)
	\item Spezielles Verhalten, wie
	\begin{itemize}
		\item Ortung ist ausgeschaltet
		\item WLAN ist ausgeschaltet
	\end{itemize}
\end{itemize}

\section{Internationalisierung}
\kort{} steht in mehreren Sprachen zur Verfügung.
Zur Übersetzung wurde das Plugin \brand{react-native-i18n}\footnote{\url{https://github.com/AlexanderZaytsev/react-native-i18n}} eingesetzt.
Das Plugin stellt einen \brand{React Native} Port der Library \brand{I18n.js}\footnote{\url{https://github.com/fnando/i18n-js}}, welche Übersetzungen mit \brand{JavaScript} ermöglicht.\newline
\brand{react-native-i18n} ermöglicht es, die Spracheinstellungen des verwendeten Smartphones auszulesen und über die ausgelagerten Übersetzungen den jeweiligen Text zu laden.