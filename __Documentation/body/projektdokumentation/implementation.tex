\chapter{Implementation}
\label{pd-implementation}
Dieses Kapitel beschreibt, wie die folgenden Punkte eingerichtet und implementiert wurden. 


\section{Kort Backend}
Das \kort{}-Backend existierte bereits und wurde von den Entwicklern nicht mehr abgeändert. 
Es ist im Kapitel 10.2. \gls{REST}-Schnittstellen der Bachelorarbeit von Jürg Hunziker und Stefan Oderbolz dokumentiert.\cite{ba-kort-2012}


\section{Libraries}
Damit die entsprechenden \glslink{Library}{Libraries} genutzt werden können, mussten sie mit dem Node-Package-Manager (npm) heruntergeladen werden. 
Mit dem \inlinecode{npm install}-Befehl wurden alle aufgelisteten Abhängigkeiten in der  \inlinecode{package.json}-Datei installiert und im Ordner \inlinecode{node-modules} gespeichert.
Das Linken der \glslink{Library}{Libraries} fand dann im \brand{Android}- und \brand{iOS}-Projekt statt. 
Dann konnten die \glslink{Library}{Libraries} in einer \brand{JavaScript}-Komponente importiert und verwendet werden. 


\section{Navigation}
Die Implementation der Navigation wurde in der Einstiegs-Komponente der App umgesetzt. 
Alle Scenes werden mit einem Key deklariert und falls nötig mit Optionen für eine Tab-Ansicht erweitert. 
Die definierten Scenes werden als Kind-Komponenten einem Router übergeben. 
Jede Scene kann mit einem Funktionsaufruf ...
\begin{itemize}
	\item ... die Properties aktualisieren
	\item ... sich selber schliessen
	\item ... eine andere Scene mit dem Key und mit optionalen Properties, als Parameter, aufrufen
\end{itemize} 
Die \gls{API} der Navigations-Komponente ist auf der entsprechenden \brand{Github}-Seite\footnote{\url{https://github.com/aksonov/react-native-router-flux}} dokumentiert. 


\section{Karte}
Um die Map-Komponente von \brand{Mapbox} zu nutzen und ein Access-Token zu erhalten, muss auf der \brand{Mapbox}-Webseite\footnote{\url{https://www.mapbox.com/}} ein Benutzerkonto angelegt werden. 

Damit die Marker an richtiger Position angezeigt werden, wurden der \brand{Mapbox}-Komponente ein Array mit Annotations als Parameter übergeben. 
Das Annotation-Array wird fortlaufend mit allen Missionen vom \gls{Backend}, in der Umgebung des Benutzers, aktualisiert. 
Eine Annotation enthält die Koordinaten und die ID einer Mission. 
Es war von der \brand{Mapbox}-\gls{API} her nicht möglich, einer Annotation direkt die Mission als Objekt mitzugeben. 
Bei einem Klick auf einen Marker wird die Funktion \inlinecode{onOpenAnnotation} aufgerufen.
Diese Funktion öffnet wiederum eine Scene, die es dem Benutzer erlaubt, die gewählte Mission zu lösen.
Die \brand{Mapbox}-\gls{API} ist auf der entsprechenden \brand{Github}-Seite\footnote{\url{https://github.com/mapbox/react-native-mapbox-gl}} dokumentiert. 


\section{OAuth}
Für die Authentifizierung über \brand{Google} wurde eine Open Source \gls{Library} evaluiert (Kapitel Evaluation \nameref{tb-evaluation-oauth}). 
Damit die App sich mit der \brand{Google}-\gls{API} verbinden kann, muss eine \inlinecode{google-services.json}-Datei auf der \brand{Google}-Webseite\footnote{\url{https://developers.google.com/cloud-messaging/android/client}} generiert werden. 
Diese Datei enthält die Konfiguration vom \brand{Google}-Developer Konto. 

Über diese \gls{Library} enthält die App, nach dem Login des Benutzers, das Benutzer-Token von \brand{Google}.
Dieses Token wird dann von der App dem \kort{}-\gls{Backend} mitgeteilt und von dort aus bei \brand{Google} überprüft. 
Wie die Authentifizierung auf der \gls{Backend}-Seite weiter verläuft ist im Kapitel 10.4.1. der Bachelorarbeit von Jürg Hunziker und Stefan Oderbolz beschrieben.\cite{ba-kort-2012}


\section{Internationalisierung}
% Properties und Json Umwandlung hier?

