\chapter{Implementation}
\label{pd-implementation}


\section{Kort Backend}


\section{Daten}


\section{Stores}


\section{Components}


\section{Libraries}
Damit die entsprechenden \glslink{Library}{Libraries} genutzt werden können, mussten sie mit dem Node-Package-Manager (npm) heruntergeladen werden. 
Mit dem \inlinecode{npm install}-Befehl wurden alle aufgelisteten Abhängigkeiten in der  \inlinecode{package.json}-Datei installiert und im Ordner \inlinecode{node-modules} gespeichert.
Das Linken der \glslink{Library}{Libraries} fand dann in der MainActivity und in \brand{Xcode} statt. 
Nun konnten die \glslink{Library}{Libraries} in einer \brand{JavaScript}-Komponente importiert und verwendet werden. 


\section{Navigation}
Wie die Navigations-Komponente verwendet wird, ist auf der \brand{Github}-Seite erklärt. 

% Aufbau
% Deklaration der Scenes - beim Kompilieren
% Reducer
% Actions


\section{Karte}
% Mapbox
% Annotations vom Store
% Parameter
% Token
% Location-tracking Einstellung


\section{OAuth}
% Google signin
% Google Konfigurationsdatei generieren mit Debug Keystore
% Google Developer Account für Web-Client-ID und iOS-Client-ID
% Secret config


\section{Internationalisierung}


