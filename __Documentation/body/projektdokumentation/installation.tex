\chapter{Installation}
\label{pd-installation}

In diesem Kapitel werden zwei Varianten beschrieben, wie die App getestet werden kann. 


\section{Installation mit Source Code}
Um die App anhand des Source Codes zu installieren, muss die \brand{React Native} Entwicklungsumgebung aufgesetzt sein. 
Wie diese eingerichtet wird ist in der Getting-Started-Anleitung, der online Dokumentation von \brand{React Native}\footnote{\url{https://facebook.github.io/react-native/docs/getting-started.html}}, erklärt.
Diese Anleitung erklärt Schritt für Schritt den Ablauf von der Einrichtung der Entwicklungsumgebung auf einem \brand{Mac}-, \brand{Linux}- oder \brand{Windows}-Rechner (für \brand{iOS} und \brand{Android}) bis zum Starten der App.

Hier muss noch beachtet werden, dass die SecretConfig.js-Datei \newline\inlinecode{(/kort-reloaded/js/constants/SecretConfig.js)}\newline mit folgenden Werten ergänzt werden sollte:

\begin{itemize}
	\item Mapbox Access Token
	\begin{itemize}
		\item Dieses Token kann auf der Mapbox-Webseite\footnote{\url{https://www.mapbox.com/help/create-api-access-token/}} erstellt werden.
	\end{itemize}
	
	\item Google Client ID 
	\begin{itemize}
		\item Wie die Google Client ID erhalten wird, ist in der Anleitung vom Github-Projekt der Google-Signin-Library erklärt.\footnote{\url{https://www.mapbox.com/help/create-api-access-token/}}
		\item Wichtig ist, dass anhand dieser Anleitung auch die google-services.json-Datei neu generiert und in das Projekt eingefügt werden muss.
	\end{itemize}
\end{itemize}

\section{Installation mit APK-Datei}
Damit die \brand{Android}-App, von der bereitgestellten APK-Datei auf der CD, installiert werden kann, muss das Smartphone mit dem Computer per USB-Kabel verbunden sein und erkannt werden. 

\begin{enumerate}
	\item APK-Datei in einen Ordner auf dem Gerätespeicher kopieren
	\item Smartphone Verbindung mit dem Computer trennen
	\item APK-Datei auf dem Smartphone auffinden und anklicken
	\item App-Herausgeber vertrauen und Installation abschliessen
\end{enumerate}
