\chapter{Entwicklungsumgebung}
\label{pd-entwicklungsumgebung}


\section{IDE}
Die Funktionalitäten und Features der App wurden alle mit \brand{Atom} implementiert. 
Für das Debugging waren die native \glslink{IDE}{IDEs}, \brand{Android Studio} und \brand{Xcode}, aber besser geeignet. 
Das Debuggen in er \brand{Atom}-\gls{IDE} oder im \brand{Google Chrome} Browser war oft fehlerhaft. 
Ansonsten wurden die native \glslink{IDE}{IDEs} nur für das Einfügen von statischen Bildern, mit passenden Auflösungen für entsprechende Displays, genutzt. 


\section{Continuous Integration}
Für die Continuous Integration (\gls{CI}) nutzen wir den freien Dienst von \brand{Travis-CI}\footnote{\url{https://travis-ci.org/}}.
Dieses Setup lässt sich bequem in Verbindung mit \brand{Github} nutzen. 
Dabei wird jeder Neuerung auf dem master- und develop-Branch über \brand{Travis-CI} ein neuer Build erstellt. 
Die Konfigurationsdatei für \brand{Travis-CI} befindet sich im \brand{Github}-Repository von \kort{}\footnote{\url{https://github.com/kort/kort-reloaded}}. 
Darin sind auch die Tools, um die \nameref{pd-entwicklungsumgebung-cr} zu überprüfen, konfiguriert. 


\section{Projektmanagement-Tool}
Als Projektmanagement-Tool wurde \brand{Redmine} verwendet. 
Weiterführende Links zum Redmine-Projekt, das für die Ticketerfassung verwendet wurde, sind im Kapitel \hyperref[pm-projektmanagement]{Projektmanagement} dokumentiert.


\section{Testing}
Für die Unit-Tests wurde \brand{Jest}\footnote{\url{https://facebook.github.io/jest/}} eingesetzt. 
\brand{Jest} ist ein \brand{JavaScript} Testing-\gls{Framework} und wird zum Beispiel von \brand{Facebook} zum Testen von \brand{React}-Applikationen verwendet. 


\section{Code-Richtlinien}
\label{pd-entwicklungsumgebung-cr}

Um die Style-Guidelines von \brand{JavaScript} einzuhalten, wurde \brand{ES-Lint}\footnote{\url{https://facebook.github.io/jest/}} eingesetzt. 
Dieses Tool prüft den ganzen Code auf Style-Einheitlichkeit und stellt sicher, dass ein aktueller Code-Standard eingehalten wird. 
Das erhöht die Lesbarkeit und Wartbarkeit vom Code. 
Die Konfiguration von \brand{ES-Lint} stammt von \brand{airbnb}\footnote{\url{...}} und konnte als Open Source Plugin, mit \brand{npm}, in das Projekt eingebunden werden. 

