\chapter{Entwicklungsumgebung}
\label{pd-entwicklungsumgebung}
\section{IDE}
Die Funktionalitäten und Features der App wurden alle mit \brand{Atom} implementiert. 
Für das Debugging waren die native \glslink{IDE}{IDEs}, \brand{Android Studio} und \brand{Xcode}, aber besser geeignet. 
Das Debuggen in der \brand{Atom}-\gls{IDE} oder im \brand{Google Chrome} Browser war oft fehlerhaft. 
Ansonsten wurden die native \glslink{IDE}{IDEs} nur für das Einfügen von statischen Bildern, mit passenden Auflösungen für entsprechende Displays, genutzt. 


\section{Continuous Integration}
Als Versionsverwaltungssystem wurde \brand{git}\footnote{\url{https://git-scm.com/}} zusammen mit dem Online-Dienst \brand{GitHub}\footnote{\url{https://github.com}} verwendet: \url{https://github.com/kort/kort-reloaded}.
Dabei haben wir folgendes Branching Model eingesetzt: Der master Branch wird nur für Releases verwendet.
Für die Entwicklung wurde der develop Branch genutzt, wobei jedes Feature und jeder Fix eines Bugs einen eigenen Branch erhielt, welcher nach Fertigstellung wieder in den develop Branch gemerged wurde.\newline
\newline
Für die \gls{CI} (CI) nutzen wir den freien Dienst von \brand{Travis-CI}\footnote{\url{https://travis-ci.org/}}.
Dieses Setup lässt sich bequem in Verbindung mit \brand{GitHub} nutzen. 
Dabei wird bei jeder Neuerung auf dem master und develop Branch über \brand{Travis-CI} ein neuer Build erstellt.
Bei jedem Build werden die Tests durchlaufen und der Code auf die Einhaltung der \nameref{pd-entwicklungsumgebung-cr} überprüft.\newline
Die Konfigurationsdatei für \brand{Travis-CI} (\inlinecode{.travis.yml}) befindet sich im \brand{GitHub}-Repository von \kort{}.

\section{Projektmanagement-Tool}
Als Projektmanagement-Tool wurde \brand{Redmine} verwendet. 
Weiterführende Links zum Redmine-Projekt, das für die Ticketerfassung verwendet wurde, sind im Kapitel \hyperref[pm-projektmanagement]{Projektmanagement} dokumentiert.


\section{Testing}
\label{pd-entwicklungsumgebung-testing}
Fürs Testing wurden \glslink{Unit Test}{Unit Tests}, \glslink{Integration Test}{Integration Tests} und \glslink{Funktionaler Test}{Funktionalen Tests} evaluiert.
Letztlich konnten -- zum Zeitpunkt der Abgabe -- lediglich die Unit Tests umgesetzt werden.
Integration Tests wären für die \inlinecode{data} Klassen wünschenswert gewesen, konnten aber aus Zeitgründen nicht umgesetzt werden.
Automated UI Tests sind mit \brand{React Native} möglich\footnote{Einen guten Überblick bietet diese Seite: \url{http://testdroid.com/tech/testing-react-native-apps-on-android-and-ios}}, sind aber sehr aufwendig einzurichten.
Für die Unit Tests wurde \brand{Jest}\footnote{\url{https://facebook.github.io/jest/}} eingesetzt. 
\brand{Jest} ist ein \brand{JavaScript} Testing-\gls{Framework} und wird zum Beispiel von \brand{Facebook} zum Testen von \brand{React}-Applikationen verwendet.\newline
Eine besondere Eigenschaft von \brand{Jest} ist, dass standardmässig für alle Module automatisch \glslink{Mock}{Mocks} bereitgestellt werden.
Somit wird verhindert, dass aus Versehen das Verhalten anderer Module getestet wird.

\section{Code-Richtlinien}
\label{pd-entwicklungsumgebung-cr}
Um Code-Richtlinien festzulegen und deren Einhaltung zu prüfen, wurde \brand{ESLint}\footnote{\url{http://eslint.org/}} eingesetzt.
Dadurch wird die die Lesbarkeit und Wartbarkeit vom Code erhöht.
Die Konfiguration von \brand{ESLint} stammt von \brand{Airbnb}\footnote{\url{https://github.com/airbnb/javascript/tree/master/packages/eslint-config-airbnb}}.
Darüber hinaus wurden Plugins für \brand{React}\footnote{\url{https://github.com/yannickcr/eslint-plugin-react}} und \brand{React Native}\footnote{\url{https://github.com/Intellicode/eslint-plugin-react-native}} eingesetzt.\newline
Die Konfiguration findet sich in der Datei \inlinecode{.eslintrc.json}.