\chapter{Evaluation}
\label{tb-evaluation}
An einem \brand{React Native} Meetup am 8. März 2016 -- an der HSR, konnten erste Entscheidungen zur Software-Entwicklungsumgebung geklärt werden.
Es wurden auch Lösungskonzepte für die Darstellung der Karte besprochen.

Unter den vielen \brand{JavaScript}-Editoren haben wir uns für \brand{Atom} entschieden.
\brand{Facebook} hat speziell für \brand{React}, \brand{React Native} das \brand{Atom}-Package \brand{Nuclide}\footnote{\url{https://nuclide.io/}} veröffentlicht. 
\brand{Atom} ist Open Source und bietet viele weitere Community-Packages.

Weitere Hinweise zur Entwicklungsumgebung und den verwendeten Werkzeugen sind im Kapitel \nameref{pm-projektmanagement} beschrieben.


\section{Architektur}
\label{tb-evaluation-architektur}
Bei der Evaluation, welches Architektur-Pattern wir anwenden würden, konnten wir uns auf \gls{Frontend}-Architekturen beschränken.
Dabei kamen für uns zwei Patterns in Frage: Traditionelles \gls{MVC} und \hyperref[pd-architektur]{\brand{Flux}}\footnote{\url{https://facebook.github.io/flux/}}.\newline

\begin{table}[H]
\centering
\label{tb-evaluation-architektur}
\begin{tabular}{|p{7cm}|p{7cm}|}
\hline
\multicolumn{2}{|c|}{\textbf{Variante A: MVC}} \\
\hline
\textbf{Vorteile} & \textbf{Nachteile} \\
\hline
Unidirektionaler Datenfluss
& Grösserer Aufwand für erstmalige Einrichtung \\
\hline
Kaum Kopplung -- Anpassungen in späteren Arbeiten relativ problemlos möglich
& Viele Callbacks nötig \\
\hline
Abstraktion des Applikationszustands passend für unseren Anwendungsfall
&  \\
\hline
Nutzt wichtige Konzepte von \brand{React}
&  \\
\hline
\multicolumn{2}{|c|}{\textbf{Variante B: \brand{Flux}}} \\
\hline
\textbf{Vorteile} & \textbf{Nachteile} \\
\hline
Pattern ist bekannt, keine Einarbeitung nötig
 &  \\
\hline
\end{tabular}
\caption{Bewertung Komponente für die Navigation}
\end{table}

\subsection{Fazit}
Die Architektur-Entscheidung fiel zu Beginn sehr schwer, da vorgängig noch nicht genügend Erfahrungen im Umgang mit \brand{React} gesammelt werden konnten.
Entsprechend unserem \hyperref[pm-projektplan]{Projektplan} und der Grundidee, erst Erfahrungen mit für uns neuen Technologien (\brand{JavaScript}, \brand{React} und \brand{React Native}) zu sammeln, haben wir uns in einem ersten Ansatz dazu entschlossen, die ersten Schritte -- das Darstellen der Missionen auf der Karte -- mit dem \gls{MVC}-Pattern umzusetzen.
Welche Auswirkungen der Einsatz von \brand{Flux} hätte, war zu Beginn zu wenig voraussehbar.\newline
Als die Wahl der Architektur im Zusammenhang mit \hyperref[pm-ms5]{Meilenstein 5} neu evaluiert wurde, konnten wir leichte Vorteile in der \brand{Flux}- gegenüber einer MVC-Architektur sehen.
Einen guten Einblick in die Thematik verschafft der Artikel \emph{Why we are doing MVC and FLUX wrong}\cite{mvc-vs-flux}.\newline
Wir haben uns letztlich aufgrund der oben aufgelisteten Vorteile auf \brand{Flux} festgelegt.

\section{App Navigation}
Um die Navigation von \kort{} zu implementieren, gab es zwei Möglichkeiten. 
Die erste Variante war die Umsetzung mit der von \brand{React Native} zur Verfügung gestellten Navigator-Komponente. 
Für eine Tab-Ansicht müsste aber eine weitere Library evaluiert werden. 

Als weitere Variante für die Navigation gab es die React-Native-Router-Flux Library. 
Diese Komponente basiert auf dem Navigator von \brand{React Native} und unterstützt durch die bereits integrierte Tab-Library eine Tab-Ansicht.  

\begin{table}[H]
\centering
\label{tb-evaluation-app-navigation}
\begin{tabular}{|p{7cm}|p{7cm}|}
\hline
\multicolumn{2}{|c|}{\textbf{Variante A: \brand{React Native} Navigator-Komponente}} \\
\hline
\textbf{Vorteile} & \textbf{Nachteile} \\
\hline
Es gibt eine Unterstützung für \brand{Android} und \brand{iOS}.
& Die Dokumentation zur korrekten Verwendung ist nicht vorhanden. \\
\hline
Flexibel und für einfache Use Cases gedacht.
 & Die Tab-Ansicht wird nicht unterstützt. \\
\hline
\multicolumn{2}{|c|}{\textbf{Variante B: React-Native-Router-Flux\footnote{\url{https://github.com/aksonov/react-native-router-flux}}}} \\
\hline
\textbf{Vorteile} & \textbf{Nachteile} \\
\hline
Alle Views (Scenes) sind an einem Ort deklariert. 
Es müssen keine Navigator-Objekte herumgereicht werden. 
 & Stellt eine weitere Abhängigkeit an ein externes Projekt dar. \\
\hline
Leicht erweiterbar und wartbar.
Integrierte Tab-Navigation.
 & Der aktuelle Navigationszustand ist nicht genau definiert. \\
\hline
Ein Wechsel der View ist mit einem Funktionsaufruf von überall aus möglich. Daten lassen sich dabei einfach als Parameter weitergeben.
 & Die Hintergrundabläufe und der Lebenszyklus sind nicht erkennbar. \\
\hline
\end{tabular}
\caption{Bewertung Navigations-Komponente}
\end{table}

\subsection{Fazit}
Da \brand{React Native} standardmässig keine Tab-Navigation anbietet, wurde React-Native-Router-Flux als Navigations-Variante evaluiert.
Der erste Prototyp mit dieser Library erfüllte die Anforderungen.
Alle Views sind an einem Ort im Code festgelegt und es lassen sich bequem weitere hinzufügen.

Im Verlaufe des Projektes sind dann aber vermehrt Fehler aufgetreten.
Im Nachhinein wäre es sinnvoller gewesen, die Navigator-Komponente zu verwenden.


\section{Kartendarstellung}
\label{tb-evaluation-karte}
Für die Darstellung der Karte mit \brand{React Native} sind die Varianten A bis E ausfindig gemacht worden. 
Diese Punkte beinhalten Libraries oder Ideen zur Umsetzung einer Kartenansicht für die App. 


\subsubsection{Variante A: React Native Map Komponente}
Diese Variante wird standardmässig von \brand{React Native} als MapView-Komponente zur Verfügung gestellt. 
Die Möglichkeiten zur Verwendung sind eingeschränkt, denn es lassen sich auf \brand{Android} nur normale Pins als Marker einsetzen.\cite{react-native-mapview-pin}


\subsubsection{Variante B: Extended React Native Map Komponente}
Die Extended React Native Map Komponente\footnote{\url{https://github.com/lelandrichardson/react-native-maps}} wird von \brand{Facebook} anstelle der Standard-MapView-Komponente empfohlen.


\subsubsection{Variante C: Mapbox GL Library}
\brand{Mapbox} bietet mit dieser experimentellen \brand{React Native}-Komponente (Mapbox GL Library\footnote{\url{https://libraries.io/npm/react-native-mapbox-gl}}) eine weitere Lösung für \brand{iOS} und \brand{Android}.


\subsubsection{Variante D: Portierung von Leaflet nach React}
Für \brand{React} gibt es eine Map-Komponente namens React-Leaflet\footnote{\url{https://github.com/PaulLeCam/react-leaflet}}. 
Diese liesse sich für \brand{React Native} portieren.
Schon in der \kort{}-\gls{WebApp} wurde die \brand{Leaflet}\footnote{\url{http://leafletjs.com/t}}-Library verwendet.


\subsubsection{Variante E: Raster-Kacheln selbst darstellen}
Die letzte mögliche Variante war, dass wir die benötigten Raster-Kacheln, die der Benutzer braucht, mit einer eigenen Implementation entsprechend laden und anzeigen.

\begin{table}[H]
\centering
\label{tb-evaluation-map-komponente}
\begin{tabular}{|p{7cm}|p{7cm}|}
\hline
\multicolumn{2}{|c|}{\textbf{Variante A: \brand{React Native} Map Komponente}} \\
\hline
\textbf{Vorteile} & \textbf{Nachteile} \\
\hline
Komponente von \brand{Facebook} -- \brand{React Native}. & Pattern Fill ist nicht implementiert und auch nicht in Planung.\cite{react-native-mapview}
Dadurch lassen sich keine eigenen Marker auf der Kartenansicht darstellen. \\
\hline
 & Es lassen sich keine Map-Kacheln von einem beliebigen Service darstellen.  \\
\hline
\multicolumn{2}{|c|}{\textbf{Variante B: Extended React Native Map Komponente}} \\
\hline
\textbf{Vorteile} & \textbf{Nachteile} \\
\hline
Bietet alle benötigten Funktionen (eigene klickbare Marker platzieren und vieles mehr).
 & Nutzt die native Map API von \brand{Apple iOS} und \brand{Android} SDK. 
 Ist fest mit \brand{Apple} und \brand{Google Maps} verbunden. \\
\hline
Wird von \brand{Facebook} empfohlen.
 & Native Map APIs sind für ein \brand{OSM}-Projekt aus moralischen Gründen unpassend. \\
\hline
\multicolumn{2}{|c|}{\textbf{Variante C: Mapbox GL Library}} \\
\hline
\textbf{Vorteile} & \textbf{Nachteile} \\
\hline
Lässt sich mit Offline-Kacheln von \brand{OSM2VectorTiles}\footnote{\url{http://osm2vectortiles.org/downloads/}} füttern (Vektor Kacheln). & Diese Library ist eine experimentelle Komponente.\cite{react-native-mapbox} \\
\hline
Es können eigene Marker-Bilder eingesetzt werden. &  \\
\hline
\multicolumn{2}{|c|}{\textbf{Variante D: Portierung von Leaflet nach React}} \\
\hline
\textbf{Vorteile} & \textbf{Nachteile} \\
\hline
 & Die Darstellung ist nur in einer Webview möglich. \\
\hline
\multicolumn{2}{|c|}{\textbf{Variante E: Raster-Tiles selbst darstellen}} \\
\hline
\textbf{Vorteile} & \textbf{Nachteile} \\
\hline
 & Bringt einen zu hohen Aufwand mit sich. \\
\hline
\end{tabular}
\caption{Bewertung Map-Komponente}
\end{table}

\subsection{Fazit}
Varianten, die native Map APIs von \brand{Google} und \brand{Apple} verwenden, kamen für uns nicht in Frage.
Wir möchten mit unserer App \brand{OSM} Daten verbessern und möchten somit aus moralischen Aspekten auch auf diese Karte setzen. 

Bei der \brand{React Native} MapView-Komponente gab es keine Möglichkeit, Bilder auf der Karte darzustellen und die Raster-Tiles von Hand anzuzeigen wäre schlicht zu aufwendig. 
Es liesse sich auch nur sehr umständlich eine schöne Map designen.

Somit sprang uns als erstes die Portierung von Leaflet für \brand{React} ins Auge. 
Nach genauerem Betrachten fiel uns aber auf, dass diese Variante nur möglich ist, wenn die Karte in einer WebView-Komponente von \brand{React Native} dargestellt wird. 
Das heisst, die App müsste zur Browser-Ansicht wechseln.

Als letzte Möglichkeit blieb die Mapbox GL Library.
Diese hat beim Testen auf Anhieb funktioniert und uns überzeugt.
Die Kosten bei einer Anzahl von 50 000 Nutzern pro Monat sind für dieses Projekt nicht problematisch.\cite{mapbox-pricing}

Der Gedanke einer Offline-Unterstützung wurde auch besprochen.
Nur unterstützt dies das Backend nicht und es gäbe Probleme mit dem Speicherplatzbedarf der App. 


\section{OAuth Implementation}
Gebraucht wird ein Login-Dienst für \brand{Google}-, \brand{Facebook}- und \brand{OSM}-Konten. 
Für einen \brand{OSM}-Login muss eine eigene Lösung entwickelt werden. 
Für die Implementation der Authentifizierung wurden folgende zwei Möglichkeiten evaluiert:


\subsubsection{Variante A: Auth0}
\brand{Auth0}\footnote{\url{https://github.com/auth0/react-native-lock}, \url{https://auth0.com/}} bietet eine Implementation für beliebige \gls{OAuth} 2-Dienste. 


\subsubsection{Variante B: Open-Source-Projekte}
Ein geeignetes Open Source Projekt für die \brand{Google}-Authentifizierung wäre react-native-google-signin\footnote{\url{https://github.com/devfd/react-native-google-signin}}. 
Für \brand{Facebook} bot sich react-native-facebook-login\footnote{\url{https://github.com/magus/react-native-facebook-login}} an. 
Diese Variante ist zu generell und wurde nicht in einer Tabellenansicht dargestellt. 

\begin{table}[H]
\centering
\label{tb-evaluation-oauth-komponente}
\begin{tabular}{|p{7cm}|p{7cm}|}
\hline
\multicolumn{2}{|c|}{\textbf{Variante A: \brand{Auth0}}} \\
\hline
\textbf{Vorteile} & \textbf{Nachteile} \\
\hline
Sehr einfach Einbindung.
 & Nur \gls{OAuth} 2 Unterstützung.\cite{auth0-oauth} \\
\hline
 & Backend Anpassung nötig. \\
\hline
Kostenlos für Open-Source-Projekte.
 & Schlecht erweiterbar mit eigenem Login für \brand{OSM}. \\
\hline
\end{tabular}
\caption{Bewertung OAuth-Komponente}
\end{table}


\subsection{Fazit}
\brand{Auth0} kam definitiv nicht in Frage, da es in nächster Zukunft nicht mit einer \gls{OAuth} 1.0a-Authentifizierung, wie sie von \brand{OSM} unterstützt wird, erweiterbar ist.
Entschieden haben wir uns für das react-native-google-signin-Projekt.
Es funktioniert auf beiden Plattformen und liefert ein Token, das vom \kort{}-Backend überprüft werden kann.
Der Nachteil ist, dass es kein Open Source Projekt gab, welches alle gewünschten Social-Logins für \brand{iOS} und \brand{Android} anbietet.

\brand{Facebook} wird vom \kort{}-Backend derzeitig nicht unterstützt und das react-native-facebook-login-Projekt auf \brand{GitHub} liefert auch kein Token, wie es beim \brand{Google}-Login der Fall ist.
Um dies zu behandeln wären weitere Anpassungen am Backend nötig gewesen.
Wir hatten uns dazu entschieden, keine Backend-Anpassungen durchzuführen.


\section{Internationalisierung}
Die Übersetzungen, die für die Internationalisierung (I18n) der App zur Verfügung stehen, wurden in der \kort{}-Bachelorarbeit von 2012 mit \brand{Transifex}\footnote{\url{https://www.transifex.com/}} bereits erhoben.\cite{ba-kort-2012}
Es stehen bereits 13 vollständig und weitere 13 teilweise übersetzte Sprachen zur Verfügung.
\brand{Transifex} ist eine Plattform, bei der Projekte zur freiwilligen Übersetzung von dessen Benutzern aufgeschaltet werden können. 
Das heisst, dass sich ein Benutzer registriert und bei der Übersetzung von anderen Projekten beitragen kann.
Die Übersetzungen sind allerdings im Java Properties Format, was eine Schwierigkeit darstellen würde.

Bei der Umsetzung der I18n müssen folgende Punkte umgesetzt werden: Auslesen des Locales aus den Geräteinformationen, entsprechendes Einsetzen der Übersetzungen und Einbinden der Übersetzungen, welche auf \brand{Transifex} zur Verfügung stehen.
Eine Library, die eine \brand{React Native} Bridge zum Auslesen des Locales, die I18n-Funktionalität zur Verfügung stellt und \inlinecode{.properties} Dateien entgegennimmt, gibt es leider nicht.

Es boten sich folgende Alternativen an:
\begin{enumerate}
\item Auslesen des Locales mit einer Library für \brand{React Native}\footnote{\url{https://github.com/fixd/react-native-locale}} und Übersetzen anhand von Java Properties Files mit einer anderen Library\footnote{\url{https://github.com/jquery-i18n-properties/jquery-i18n-properties}}
\item Auslesen des Locales und Übersetzen mit einer Library für \brand{React Native}\footnote{\url{https://github.com/AlexanderZaytsev/react-native-i18n}} und manuelles Umwandeln der Java Properties Files in JSON
\end{enumerate}

Die 1. Variante kam für uns nicht in Frage, da JQuery nicht ins Projekt eingebunden werden soll.
So haben wir uns für die 2. Variante entschieden, vorerst aber nur Deutsch und Englisch ins richtige Format umgewandelt.
Die weiteren Sprachen werden noch nach der Abgabe übernommen.
