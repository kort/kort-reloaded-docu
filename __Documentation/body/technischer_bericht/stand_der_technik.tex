\chapter{Stand der Technik}
\label{tb-stand-der-technik}
Ähnliche Projekte, die \brand{OpenStreetMap} verwenden und mit \brand{React-Native} geschrieben sind, gibt es bis jetzt keine.
Parallel zu dieser Arbeit wird die Traildevils-Bachelorarbeit durchgeführt.
Diese Gruppe setzt auch auf \brand{React-Native} und hat ebenfalls die Karte als Hauptansicht.
Jedoch verwendet sie die native Map APIs von Android und iOS.

\section{React Native}


\section{Bestehende Lösungsansätze und Normen}
Da wir die \kort{}-\gls{WebApp} neu schreiben, übernehmen wir die dort evaluierten Konzepte.

%Hier Abgrenzung? -> Fortsetzungsarbeit

\section{OSM OAuth}


\section{Kort Schnittstelle}

