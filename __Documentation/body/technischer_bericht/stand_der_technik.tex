\chapter{Stand der Technik}
\label{tb-stand-der-technik}
Es gibt eine grosse Anzahl von Projekten, mit dem Ziel \brand{OpenStreetMap} durch \gls{Crowdsourcing} zu verbessern.
Es werden sowohl Editoren für erfahrene Benutzer, als auch Tools, die auf das finden und Beheben von Fehlern spezialisiert sind, angeboten.
Zum Beispiel gibt es JOSM\footnote{\url{https://josm.openstreetmap.de/}} (\brand{Java OpenStreetMap Editor}) als Desktop Client für \brand{Windows}, \brand{Mac OS X} und \brand{Linux}.
Sogar Geometrieobjekte sind editierbar.
Daneben gibt es auch Dienste, welche Fehlerdaten sammeln und öffentlich anbieten.
Ein bekanntes Beispiel ist \brand{KeepRight}\footnote{\url{http://www.keepright.at/}}.
Es werden über 50 Fehlertypen angeboten.
Mit diesen Werkzeugen können nun Daten korrigiert werden.
Weitere Editoren sind auf dem \brand{OpenStreetMap}-Wiki\footnote{\url{http://wiki.openstreetmap.org/wiki/Editors#Choice_of_editors}} zu finden.

Die Zielgruppe dieser Werkzeuge liegt bei Benutzern mit dem Interesse, die \brand{OpenStreetMap}-Daten zu verbessern.
Damit \gls{Crowdsourcing} genutzt werden kann sollte die Zielgruppe erweitert werden, um möglichst viele Benutzer anzusprechen.
Ein sehr gutes Beispiel dazu ist MapRoulette\footnote{\url{http://wiki.openstreetmap.org/wiki/MapRoulette}}.
Dem Benutzer werden Challenges präsentiert, die zum Beispiel ein Satellitenbild von einem angeblichen Fussballfeld zeigen. 
Nun entscheidet der Benutzer in dem er das Feld entsprechend markiert und so die Challenge löst.

\gls{Gamification} bietet sich also sehr gut dazu an.
Die Spiel-Elemente halten den Spieler motiviert und binden ihn an das Spiel.
Einige Projekte\footnote{\url{http://wiki.openstreetmap.org/wiki/Games#Gamification_of_map_contributions}}, die durch \gls{Gamification} beitragen \brand{OpenStreetMap} zu verbessern, sind bereits entstanden.

Ähnliche Projekte, die mit \brand{React-Native} geschrieben sind, gibt es bis jetzt keine.


\section{React Native}


\section{Bestehende Lösungsansätze und Normen}
Da wir die \kort{}-\gls{WebApp} neu schreiben, übernehmen wir die dort evaluierten Konzepte.

%Hier Abgrenzung? -> Fortsetzungsarbeit

\section{OSM OAuth}


\section{Kort Schnittstelle}

