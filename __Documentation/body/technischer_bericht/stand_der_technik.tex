\chapter{Stand der Technik}
\label{tb-stand-der-technik}
Es gibt eine grosse Anzahl von Projekten, mit dem Ziel \brand{OpenStreetMap} durch \gls{Crowdsourcing} zu verbessern.
Es werden sowohl Editoren für erfahrene Benutzer, als auch Tools, die auf das finden und Beheben von Fehlern spezialisiert sind, angeboten.
Zum Beispiel gibt es JOSM\footnote{\url{https://josm.openstreetmap.de/}} (\brand{Java OpenStreetMap Editor}) als Desktop Client für \brand{Windows}, \brand{Mac OS X} und \brand{Linux}.
Sogar Geometrieobjekte sind editierbar.
Daneben gibt es auch Dienste, welche Fehlerdaten sammeln und öffentlich anbieten.
Ein bekanntes Beispiel ist \brand{KeepRight}\footnote{\url{http://www.keepright.at/}}.
Es werden über 50 Fehlertypen angeboten.
Mit diesen Werkzeugen können nun Daten korrigiert werden.
Weitere Editoren sind auf dem \brand{OpenStreetMap}-Wiki\footnote{\url{http://wiki.openstreetmap.org/wiki/Editors\#Choice_of_editors/}} zu finden.

Die Zielgruppe dieser Werkzeuge liegt bei Benutzern mit dem Interesse, die \brand{OpenStreetMap}-Daten zu verbessern.
Damit \gls{Crowdsourcing} genutzt werden kann sollte die Zielgruppe erweitert werden, um möglichst viele Benutzer anzusprechen.
Ein sehr gutes Beispiel dazu ist MapRoulette\footnote{\url{http://wiki.openstreetmap.org/wiki/MapRoulette/}}.
Dem Benutzer werden Challenges präsentiert, die zum Beispiel ein Satellitenbild von einem angeblichen Fussballfeld zeigen. 
Nun entscheidet der Benutzer in dem er das Feld entsprechend markiert und so die Challenge löst.

\gls{Gamification} bietet sich also sehr gut dazu an.
Die Spiel-Elemente halten den Spieler motiviert und binden ihn an das Spiel.
Einige Projekte\footnote{\url{http://wiki.openstreetmap.org/wiki/Games\#Gamification_of_map_contributions}}, die durch \gls{Gamification} beitragen \brand{OpenStreetMap} zu verbessern, sind bereits entstanden.
Ähnliche Projekte, die mit \brand{React-Native} geschrieben sind, gibt es bis jetzt keine.


\section{Bestehende Lösungsansätze und Normen}
Da wir die \kort{}-\gls{WebApp} neu schreiben und es sich um eine Fortsetzungsarbeit handelt, übernehmen wir die bereits dort evaluierten Konzepte.


\section{React Native}
Aktuell befindet sich \brand{React Native} in der Version 0.27.
Die Version 0.28 ist bereits in Bearbeitung.
Am Anfang der Bachelorarbeit war 0.19 die aktuellste Version (29. Januar 2016).
\brand{React Native} ist ganz neu und noch in der Entwicklungsphase.
Momentan erscheint immer noch alle zwei Wochen ein neues Release, mit Änderungen, die teilweise sehr nützlich und wichtig sind.
Es ist aber auch schon vorgekommen, dass nach einem Update die App nicht mehr lauffähig war.
Meistens lag es an der Open Source Komponente für die Navigation, die erst mit Verzögerung aktualisiert werden konnte.
Build-Probleme sind also des öfteren aufgetreten.

Die offizielle Dokumentation\footnote{\url{https://facebook.github.io/react-native/docs/getting-started.html}} ist spärlich und Best Practices gibt es in vielen Bereichen gar keine.
Durch die vielen Änderungen können sich auch keine Best Practices etablieren.
 Die meisten Open Source Projekte verfolgen eigene Implementationsansätze.
\brand{React} gib es ebenfalls erst seit 2013.

Auch die Community wirkt zerstreut. 
Viele Informationen sind im GitHub-Repository\footnote{\url{https://github.com/facebook/react-native/issues}} (aktuell sind 694 offene Issues vorhanden), von \brand{React Native}, in den Issues versteckt.

\begin{itemize}
	\item Vorhanden ist eine öffentliche, aktive und hilfsbereite \brand{Facebook}-Gruppe\footnote{\url{https://www.facebook.com/groups/react.native.community/}}, mit derzeit ca. 3 500 Mitgliedern.
	\item Es gibt eine \brand{Stack-Overflow}-Kategorie\footnote{\url{http://stackoverflow.com/questions/tagged/react-native}} mit wenigen Antworten und Lösungen.
	\item JS.coach\footnote{\url{https://js.coach/}} - listet viele Open Source Projekte auf.
	\item Übersicht über aktuelle Artikel und Blogposts: 
	\begin{itemize}
		\item reactnative.com\footnote{\url{http://www.reactnative.com/}}
		\item \brand{React Native} Newsletter\footnote{\url{http://reactnative.cc/}}
	\end{itemize}
	\item Ebenfalls gibt es einen aktiven Subreddit\footnote{\url{https://www.reddit.com/r/reactnative}}
\end{itemize}


\section{OpenStreetMap OAuth}
Im Gegensatz zu den anderen Social-Login-Varianten (\brand{Google} und \brand{Facebook}), die OAuth 2.0 anbieten, unterstützt \brand{OpenStreetMap} noch OAuth 1.0a\footnote{\url{http://wiki.openstreetmap.org/wiki/OAuth}}.


\section{Kort Schnittstelle}
Die \kort{}-\gls{WebApp} nutzte eine Session-basierte (Cookie-based) Authentifizierung. 
Eine Session-ID wird auf der Seite des Clients in einem Cookie gespeichert. 
Bei jedem Request sendet der Client-Browser das Cookie an den Server und wird so wiedererkannt. 
Der Server geht dann davon aus, dass es sich beim Client um den Inhaber der Session-ID handelt. 
\kort{} als native App kann aber keine Session zu einem Server aufbauen.
Das \kort{}-Backend wurde von Stefan Oderbolz und Jürg Hunziker dann so angepasst, dass eine Token-basierte Authentifizierung unterstützt wird.
Vorerst wurde nur eine Unterstützung für eine Login-Variante über \brand{Google} umgesetzt.

