\chapter{Stand der Technik}
\label{tb-stand-der-technik}
Es gibt eine grosse Anzahl an Projekten mit dem Ziel \brand{OpenStreetMap} zu verbessern.
Sowohl Editoren für erfahrene Benutzer, als auch Tools, die auf das Finden und Beheben von Fehlern spezialisiert sind, werden angeboten.
Zum Beispiel gibt es JOSM\footnote{\url{https://josm.openstreetmap.de/}} (\brand{Java \brand{OSM} Editor}) als Desktop Client für \brand{Windows}, \brand{Mac OS X} und \brand{Linux}.
Sogar Geometrieobjekte sind damit editierbar.
Daneben gibt es auch Dienste, welche Fehlerdaten sammeln und öffentlich anbieten.
Ein bekanntes Beispiel ist \brand{KeepRight}\footnote{\url{http://www.keepright.at/}}.
Es werden über 50 Fehlertypen angeboten.
Mit diesen Werkzeugen können Daten korrigiert werden.
Weitere Editoren sind auf dem \brand{OSM}-Wiki\footnote{\url{http://wiki.openstreetmap.org/wiki/Editors\#Choice_of_editors/}} zu finden.

Die Zielgruppe dieser Werkzeuge sind Benutzer mit dem Interesse, die \brand{OSM}-Daten zu verbessern.
Um diesen Prozess effizienter zu gestalten, würde sich ein \gls{Crowdsourcing}-Ansatz anbieten.
Dafür muss die Zielgruppe aber erweitert werden.
Ein sehr gutes Beispiel dazu ist \brand{MapRoulette}\footnote{\url{http://wiki.openstreetmap.org/wiki/MapRoulette/}}.
Dem Benutzer werden Challenges präsentiert, die zum Beispiel ein Satellitenbild von einem Objekt anzeigen, das nach der Inspektion vom System einem Fussballfeld ähnlich sieht.
Nun muss der Benutzer entscheiden, ob es sich tatsächlich um ein Fussballfeld handelt, oder nicht.
\gls{Gamification} durch Challenges bietet sich also sehr gut an.
Die Spiel-Elemente halten den Spieler motiviert und binden ihn an das Spiel.
Einige Projekte\footnote{\url{http://wiki.openstreetmap.org/wiki/Games\#Gamification_of_map_contributions}}, die durch \gls{Gamification} an \brand{OSM}-Verbesserungen beitragen, sind bereits entstanden.\cite{ba-kort-2012}

\section{Bestehende Lösungsansätze und Normen}
Da wir die \kort{}-\gls{WebApp} neu schreiben und es sich um eine Fortsetzungsarbeit handelt, übernehmen wir die bereits dort evaluierten Konzepte.


\section{React Native}
Aktuell, am 17.06.2016, befindet sich \brand{React Native} in der Version 0.28.
Dieses Projekt startete mit \brand{React Native} 0.19 (veröffentlicht am 29. Januar 2016).
\brand{React Native} ist ganz neu und noch in der Entwicklungsphase.
Momentan erscheint immer noch alle zwei Wochen ein neues Release mit Änderungen, die teilweise sehr nützlich und wichtig sind.
Es ist aber auch schon vorgekommen, dass Updates ausgelassen werden mussten, da die App nicht mehr lauffähig war.
Build-Probleme sind des Öfteren aufgetreten.

Die offizielle Dokumentation\footnote{\url{https://facebook.github.io/react-native/docs/getting-started.html}} ist immer noch spärlich und Best Practices gibt es in vielen Bereichen keine.


\section{OpenStreetMap OAuth}
Im Gegensatz zu den anderen Social-Login-Varianten (\brand{Google} und \brand{Facebook}), die OAuth 2.0 anbieten, unterstützt \brand{OSM} noch OAuth 1.0a\footnote{\url{http://wiki.openstreetmap.org/wiki/OAuth}}.


\section{Kort Schnittstelle}
Die \kort{}-\gls{WebApp} nutzte eine Cookie-based Authentifizierung.\cite{ba-kort-2012} 
Eine Session-ID wird auf der Seite des Clients in einem Cookie gespeichert. 
Bei jedem Request sendet der Client-Browser das Cookie an den Server und wird so wiedererkannt. 
Der Server geht dann davon aus, dass es sich beim Client um den Inhaber der Session-ID handelt. 
\kort{} als native App kann aber keine Session zu einem Server aufbauen.
Das \kort{}-Backend wurde von den Projektpartnern, Stefan Oderbolz und Jürg Hunziker, dann so angepasst, dass eine Token-basierte Authentifizierung unterstützt wird.
Vorerst wurde nur eine Unterstützung für eine Login-Variante über \brand{Google} umgesetzt.

