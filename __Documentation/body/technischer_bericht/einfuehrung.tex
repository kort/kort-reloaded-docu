\chapter{Einführung}
\label{tb-einfuehrung}

\section{Problemstellung, Vision}
\kort{} ist eine \gls{WebApp}, entwickelt mit dem \brand{Sencha Touch 2} \gls{Framework}.
Diese Technologie ist jetzt aber veraltet und funktioniert auf neuen Browsern nicht mehr sinngemäss.
Zum Beispiel ist das Scrollen blockiert.
So ist es auf einem mobilen Gerät unmöglich, das Login-Feld überhaupt auszufüllen.

Aus diesem Grund entstand die Idee, die \kort{} \gls{WebApp} mit einer anderen Technologie, neu zu schreiben.
Dazu bot sich \brand{React-Native} an. 
Diese neue Technologie ermöglicht es uns, native \brand{iOS} und \brand{Android} Apps mit \brand{JavaScript} zu erstellen. 


\section{Ziele}
Unser Ziel ist es eine \brand{Android} App mit gleicher Funktionalität wie die derzeitige \gls{WebApp} zu erstellen.
Zusätzlich gibt es optionale Ziele, wie das Hinzufügen weiterer Mission-Types.
Zum Beispiel das Korrigieren von Hausnummern oder eine weitere Authentifizierung Option über Twitter (mit \gls{OAuth}).
Besprochen wurde auch die Integration von Social Media zum Austausch von Aktivitäten und das Erweitern vom Konzept für Gamification.

\begin{itemize}
	\item Erstellen einer \brand{Android} App mit \brand{React-Native}
	\item Einsatz der \brand{React-Native} Geolocation Komponente zur Ansteuerung vom GPS.
	\item Als Basis sollen Daten und Webdienste des \brand{OpenStreetMap}-Projekts verwendet werden.
	\item Es soll geprüft werden, wie viel Aufwand eine \brand{iOS} App mit dem bestehenden Code der \brand{Android} Version erstellen lässt.
	\item Es soll ein Erfahrungsbericht zu \brand{React-Native} erstellt werden.
	\item Es sollen Vorkehrungen getroffen werden, um die Internationalisierung einfach zu ermöglichen.
\end{itemize}


\section{Rahmenbedingungen, Umfeld, Definitionen, Abgrenzungen}
\begin{itemize}
\item Es gelten die Rahmenbedingungen, Vorgaben und Termine der HSR.
\item Die Projektabwicklung orientiert sich an einer iterativen, agilen Vorgehensweise. Als Vorgabe dient dabei Scrum.
\item Die Kommunikation in der Projektgruppe, in der Dokumentation und an den Präsentationen erfolgt auf Deutsch.
\end{itemize}


\section{Vorgehen, Aufbau der Arbeit}

