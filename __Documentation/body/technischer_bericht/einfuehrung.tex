\chapter{Einführung}
\label{tb-einfuehrung}

\section{Problemstellung}
\kort{} ist eine \gls{WebApp}, entwickelt mit dem \brand{Sencha Touch 2} \gls{Framework}.
Die Implementation funktioniert auf neuen Browsern nicht mehr sinngemäss.
Zum Beispiel ist das Scrollen blockiert.
So ist es auf einem mobilen Gerät unmöglich, das Login-Feld überhaupt auszufüllen.
Die Ortung und die HTTP Requests funktionieren nur noch mit dem \brand{Firefox}.
\brand{Google Chrome} erlaubt die Geolocation nur noch mit einer HTTPS-Verbindung und diese wird vom \kort{}-Backend nicht unterstützt.

Aus diesem Grund entstand die Idee, die \kort{}-\gls{WebApp} mit einer anderen Technologie, neu zu schreiben.
Dazu bot sich \brand{React-Native} an. 
Diese ganz neue Technologie ermöglicht es native \brand{iOS} und - seit Oktober 2015 - auch \brand{Android} Apps mit \brand{JavaScript} zu erstellen. 


\section{Ziele}
\label{tb-einfuehrung-ziele}
Ziel ist es eine \brand{Android} App mit gleicher Funktionalität wie die derzeitige \gls{WebApp} zu erstellen.
Zusätzlich gibt es optionale Ziele, wie das Erstellen einer \brand{iOS}-App.

\begin{itemize}
	\item Erstellen einer \brand{Android} App mit \brand{React-Native} - gleiche Funktionalität mit neuem Framework, damit die mobile App mit den neusten Technologien arbeitet und künftig besser wartbar ist.
	\item Neue Erkenntnisse zum aktuellen Framework (\brand{React-Native}), zur Realisierung von native mobile Apps, sammeln.
	\item Als Basis sollen Daten und Webdienste des \brand{OpenStreetMap}-Projekts verwendet werden.
	\item Es soll ein Erfahrungsbericht zu \brand{React-Native} erstellt werden.
	\item Die Internationalisierung soll einfach umgesetzt sein.
\end{itemize}


\section{Rahmenbedingungen}
\begin{itemize}
	\item Es gelten die Rahmenbedingungen, Vorgaben und Termine der HSR.
	\item Die Projektabwicklung orientiert sich an einer iterativen, agilen Vorgehensweise. Als Vorgabe dient dabei Scrum.
\end{itemize}


\section{Vorgehen}
\begin{itemize}
	\item Einarbeiten in JavaScript und React Native sowie damit verbundenen Technologien.
	\item Einarbeiten in den Code und die Infrastruktur von (play-)Kort.
	\item Iteratives Entwickeln des Prototyps.
	\item Dokumentation abschliessen.
\end{itemize}

Ziele und Resultate werden im technischen Bericht, Unterkapitel \nameref{tb-einfuehrung-ziele} und im Kapitel \nameref{tb-resultate}, erläutert.
Alle Risiken sind im Kapitel \nameref{pm-projektmanagement-risikomanagement} definiert und die Meilensteine im Kapitel \nameref{pm-ms}.
Das Team und die verschiedenen Rollen sind im Kapitel \nameref{pm-rollen} deklariert.
Der Source Code ist auf \brand{GitHub} frei zugänglich.
