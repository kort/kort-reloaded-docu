\chapter{Einführung}
\label{tb-einfuehrung}

\section{Problemstellung, Vision}
\kort{} ist eine \gls{WebApp}, entwickelt mit dem \brand{Sencha Touch 2} \gls{Framework}.
Diese Technologie ist jetzt aber veraltet und funktioniert auf neuen Browsern nicht mehr sinngemäss.
Zum Beispiel ist das Scrollen blockiert.
So ist es auf einem mobilen Gerät unmöglich, das Login-Feld überhaupt auszufüllen.

Aus diesem Grund entstand die Idee, die \kort{} \gls{WebApp} mit einer anderen Technologie, neu zu schreiben.
Dazu bot sich \brand{React-Native} an. 
Diese neue Technologie ermöglicht es uns, native \brand{iOS} und \brand{Android} Apps mit \brand{JavaScript} zu erstellen. 


\section{Ziele}
Unser Ziel ist es eine \brand{Android} App mit gleicher Funktionalität wie die derzeitige \gls{WebApp} zu erstellen.
Zusätzlich gibt es optionale Ziele, wie das Hinzufügen weiterer Mission-Types.
Zum Beispiel das Korrigieren von Hausnummern.


\section{Rahmenbedingungen, Umfeld, Definitionen, Abgrenzungen}
\begin{itemize}
\item Es gelten die Rahmenbedingungen, Vorgaben und Termine der HSR
\item Die Projektabwicklung orientiert sich an einer iterativen, agilen Vorgehensweise. Als Vorgabe dient dabei Scrum, wobei bedingt durch das kleine Projektteam gewisse Vereinfachungen vorgenommen werden.
\item Die Kommunikation in der Projektgruppe, in der Dokumentation und an den Präsentationen erfolgt auf Deutsch.
\end{itemize}


\section{Vorgehen, Aufbau der Arbeit}

