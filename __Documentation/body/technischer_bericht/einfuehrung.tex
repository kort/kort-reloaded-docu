\chapter{Einführung}
\label{tb-einfuehrung}

\section{Problemstellung}
\brand{OpenStreetMap} (\brand{OSM}) beinhaltet eine sehr grosse Menge an Geodaten, welche frei zugänglich sind.
Für die Pflege dieser Daten ist es daher naheliegend, auf unterstützende Software zurückzugreifen.
Zu diesem Zweck gibt es eine Reihe von Applikationen, welche sich grob in zwei Kategorien einteilen lassen:
Editoren und Tools zur Qualitätssicherung.

Mit den Editoren lässt sich die \brand{OSM}-Karte direkt oder indirekt verändern und ergänzen.
Die Qualitätssicherungstools zielen darauf ab, fehlende oder falsche Daten aufzuspüren.
Diese werden dann entweder automatisch korrigiert oder übersichtlich dargestellt, um eine manuelle Korrektur zu ermöglichen.

Einige Tools wie \brand{KeepRight}\footnote{\url{http://keepright.ipax.at/}} oder \brand{Osmose}\footnote{\url{http://osmose.openstreetmap.fr/map/}} berechnen aus den Karten-Rohdaten die vorhandenen Fehler.
Dazu werden einige Heuristiken verwendet oder einfache Plausibilitätsüberprüfung durchgeführt.
Typische Fehler aus diesen Quellen sind \glspl{POI} ohne Namen oder Strassen ohne definierte Geschwindigkeitslimiten.\cite{ba-kort-2012}

Zur Behebung dieser Fehler ist die \gls{WebApp} \kort{}\footnote{\url{http://play.kort.ch/}} in Form einer Bachelorarbeit von Jürg Hunziker und Stefan Oderbolz im Herbstsemester 2012/13 entwickelt  worden.\cite{ba-kort-2012}

\kort{} wurde mit dem \brand{Sencha Touch 2} \gls{Framework} entwickelt.
Da sich HTML5 weiterentwickelt hat, funktioniert die Implementation auf neuen Browsern nicht mehr sinngemäss.
Die Ortung und die HTTP Requests funktionieren nur noch mit \brand{Firefox}.
\brand{Google Chrome} erlaubt die Geolocation nur noch mit einer HTTPS-Verbindung und diese wird vom \kort{}-Backend nicht unterstützt.

Aus diesem Grund entstand die Idee, die \kort{} mit einem native Client zu ersetzen.
Dazu bot sich \brand{React-Native} an. 
Diese ganz neue Technologie ermöglicht es native \brand{iOS} und -- seit Oktober 2015 -- auch \brand{Android} Apps mit \brand{JavaScript} zu erstellen. 
Dadurch, dass \brand{React-Native} noch in den Kinderschuhen steckt und die Entwickler noch keine Erfahrung mit \brand{JavaScript} haben, stellt dies ein grosses \hyperref[pm-projektmanagement-risikomanagement]{Risiko} für diese Bachelorarbeit dar.

\section{Ziele}
\label{tb-einfuehrung-ziele}
Ziel ist es eine \brand{Android} App mit gleicher Funktionalität, wie die derzeitige \gls{WebApp} zu erstellen.
Zusätzlich gibt es optionale Ziele, wie das Erstellen der \brand{iOS}-App.

Es entstand die Idee, den Validationsmechanismus von \kort{} abzulösen.
Bis jetzt wurden viel zu viele Missionen gelöst, die nie von anderen Benutzern validiert worden sind.
Das führte dazu, dass nur sehr wenige Änderungen überhaupt in der \brand{OpenStreetMap}-Datenbank eingefügt wurden.
Die \brand{OSM}-Community forderte aber, dass die Antworten geprüft werden müssen.
Validationsaufträge werden dem Benutzer nun einfach als normale Aufträge angezeigt.
Sobald die abgeschickte Antwort mit der zu validierenden Antwort übereinstimmt, wird im Hintergrund eine positive Bewertung versendet.
Ab drei positiven Bewertungen gilt die Antwort als korrekt und sie schafft es in die \brand{OSM}-Datenbank.
Weitere Ziele sind:

\begin{itemize}
	\item Das Erstellen einer \brand{Android} App mit \brand{React-Native} -- gleiche Funktionalität mit neuem Framework, damit die mobile App mit den neusten Technologien arbeitet und künftig besser wartbar ist.
	\item Als Basis sollen Daten und Webdienste des \brand{OSM}-Projekts verwendet werden.
	\item Der Validationsmechanismus soll abgelöst werden.
	\item Es sollen neue Erkenntnisse zum aktuellen Framework (\brand{React-Native}), zur Realisierung von native mobile Apps, gesammelt werden.
	\item Ein Erfahrungsbericht zu \brand{React-Native} soll erstellt werden.
	\item Die Internationalisierung soll einfach umgesetzt sein.
\end{itemize}

\section{Rahmenbedingungen}
Die \brand{React-Native}-App baut auf das bestehende \kort{}-Backend auf.  
Folglich geht es bei der Entwicklung nur um das Frontend -- also um den native Client.
Anpassungen am Backend lagen nicht im \hyperref[pd-anforderungsspezifikation]{Aufgabenbereich} der Entwickler.
Einen Überblick über die Architektur und das Gesamtsystem von \kort{}, welches bereits bestand, gibt das Kapitel 8 der Bachelorarbeit von Jürg Hunziker und Stefan Oderbolz.\cite{ba-kort-2012}

\begin{itemize}
	\item Es gelten die Rahmenbedingungen, Vorgaben und Termine der HSR.
	\item Die Projektabwicklung orientiert sich an einer iterativen, agilen Vorgehensweise. 
	Als Vorgabe dient dabei Scrum.
\end{itemize}

\section{Vorgehen}
\begin{itemize}
	\item Einarbeiten in \brand{JavaScript} und \brand{React Native} sowie damit verbundenen Technologien
	\item Einarbeiten in den Code und die Infrastruktur von \kort{}
	\item Iteratives Entwickeln des Prototyps
	\item Dokumentation abschliessen
\end{itemize}

\nameref{tb-einfuehrung-ziele} und \nameref{tb-resultate} werden im technischen Bericht erläutert.
Die \nameref{pm-projektmanagement-risikoanalyse}, die \nameref{pm-meilensteine} und das \nameref{pm-team} befinden sich im Kapitel Projektmanagement. 
Der Source Code ist auf \brand{GitHub}\footnote{\url{https://github.com/kort/kort-reloaded}} frei zugänglich. 
