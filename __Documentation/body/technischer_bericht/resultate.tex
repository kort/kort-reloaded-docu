\chapter{Resultate}
\label{tb-resultate}

Wir konnten die wichtigsten Hauptziele erreichen.
\kort{} erfüllt nun die Grundlage aller wichtigen Anforderungen an eine moderne App.
Nach dem Login wird der Benutzer wie schon bei der \kort{}-\gls{WebApp} zur Kartenansicht weitergeleitet.
Er kann auf einen Marker klicken und gelangt direkt zur Ansicht, um die gewählte Mission zu lösen.
Der Zwischenschritt, dass der Benutzer zuerst gefragt wird, ob er die Lösung kennt, wurde weggelassen.
Somit entfällt ein weiterer Klick, was die Spielmechanik vereinfacht.
Nur noch das Marker-Icon auf der Karte verrät etwas über den Missionstyp und weckt dabei immer noch die Neugier beim Benutzer.

%ToDo: Screenshots

\section{Zielerreichung}

Erreichte Ziele:
\begin{itemize}
	\item \brand{Android}-App mit gleicher Grundfunktionalität, wie die \gls{WebApp}
	\item \brand{iOS}-App mit gleicher Grundfunktionalität, wie die \gls{WebApp}
	\begin{itemize}
		\item Nicht getestet
	\end{itemize}
	\item neuer Validationsmechanismus
	\item Erfahrungsbericht zu \brand{React-Native}
	\item Internationalisierung umgesetzt
\end{itemize}

Dadurch, dass das Backend nicht angepasst werden sollte, konnte kein Login für \brand{OpenStreetMap} und \brand{Facebook} realisiert werden.
Für \brand{Google} wurde eine Ausnahme gemacht.

Die Badges konnten nach dem Ablösen des Validationsmechanismus nicht mehr verwendet werden.
Das Backend macht nämlich immer noch die Unterscheidung zwischen Missionen und Validationen.

Um die zweite Highscore-Ansicht zu erstellen, gab es technische Schwierigkeiten mit der verwendeten Navigations-Komponente.
Die Highscore-Ansicht befindet sich in der Haupt-Tab-Ansicht der App. 
Im Highscore-Screen selber müsste eine weitere geschachtelte Tab-Komponente eingefügt werden.
Diese Ansicht würde die Ranglistenposition vom Benutzer, verglichen mit dem Rang der Benutzer direkt vor und nach ihm, in der globalen Highscore, anzeigen. 

Leider hat es zeitlich auch nicht mehr gereicht, um die Kartenansicht beim Lösen einer Mission anzuzeigen.

Für eine finale Version, die veröffentlicht werden kann, muss das Design aufgebessert werden. 
Wie diese zusätzlichen Arbeiten umgesetzt werden könnten wurde im Kapitel \hyperref[pd-weiterentwicklung-vorgehen]{Vorgehen} dokumentiert.

\section{Ausblick}
Offene Punkte und nächste geplante Arbeiten mit höherer Priorität:

\begin{itemize}
	\item \brand{OpenStreetMap}-Login
	\item Finale \brand{iOS}- und \brand{Android}-App
	\item Veröffentlichung im \brand{Apple App Store} und \brand{Google Play Store} 
	\item Promotions-Funktion
\end{itemize}

Das genaue Vorgehen, wie die offenen Punkte umgesetzt werden, wird im Kapitel \hyperref[pd-weiterentwicklung-vorgehen]{Vorgehen} beschrieben.
Für die Zukunft gibt es bereits viele weitere Ideen. 
Eine Liste wurde im Kapitel \hyperref[pd-weiterentwicklung-realistisch]{Weiterentwicklung} erstellt. 

\section{Persönliche Berichte}

\subsubsection{Dominic Mülhaupt}

\subsubsection{Marino Melchiori}
Beim Start von diesem Projekt hatte ich keine \brand{JavaScript}-Vorkenntnisse.
Im Verlauf der Arbeit gelang es mir aber, dank der Zusammenarbeit in unserem Team, mich gut  einzuarbeiten.
Das Highlight dieser Arbeit war für mich die Implementation der App mit \brand{React Native} und die Gestaltung der Benutzeroberfläche mit \brand{JSX}.
Trotzt all den neuen und teilweise unreifen Technologien, ist es uns gelungen, eine gute App-Idee neu zu entwickeln.
Ich bin froh, dass ich \brand{JavaScript}-Erfahrungen sammeln konnte.