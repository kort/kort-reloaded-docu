\chapter{Resultate}
\label{tb-resultate}



%ToDo: Screenshots

\section{Zielerreichung}
Erreichte Ziele:

\begin{itemize}
	\item finale \brand{Android}-App mit gleicher Funktionalität, wie die \gls{WebApp}
	\item \brand{iOS}-App mit gleicher Funktionalität, wie die \gls{WebApp}
	\item neuer Validationsmechanismus
	\item Erfahrungsbericht zu \brand{React-Native}
	\item Internationalisierung umgesetzt
\end{itemize}

\section{Ausblick}
Nächste geplante Arbeiten mit höherer Priorität:

\begin{itemize}
	\item \brand{OpenStreetMap}-Login
	\item Finale \brand{iOS}-App
	\item Veröffentlichung im \brand{Apple App Store} und \brand{Google Play Store} 
	\item Promotions-Funktion
\end{itemize}

Für die Zukunft gibt es bereits viele weitere Ideen. 
Eine Liste wurde im Kapitel \hyperref[pd-resultate-weiterentwicklung]{Weiterentwicklung} erstellt.

\section{Persönliche Berichte}

\subsubsection{Dominic Mülhaupt}

\subsubsection{Marino Melchiori}
Beim Start von diesem Projekt hatte ich keine \brand{JavaScript}-Vorkenntnisse.
Im Verlauf der Arbeit gelang es mir aber, dank der Zusammenarbeit in unserem Team, mich einzuarbeiten.
Das Highlight dieser Arbeit war für mich die Implementation der App mit \brand{React} und die Gestaltung der Benutzeroberfläche mit \brand{JSX}.
Trotzt all den neuen und teilweise unreifen Technologien, ist es uns gelungen, eine gute App-Idee neu zu entwickeln.
Ich bin froh, dass ich \brand{JavaScript}-Erfahrungen sammeln konnte.