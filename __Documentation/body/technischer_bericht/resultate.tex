\chapter{Resultate}
\label{tb-resultate}

Wir konnten die wichtigsten Hauptziele erreichen.
\kort{} erfüllt nun alle wichtigen Anforderungen an eine moderne App.
Nach dem Login wird der Benutzer wie schon bei der \kort{}-\gls{WebApp} zur Kartenansicht weitergeleitet.
Er kann auf einen Marker klicken und gelangt direkt zur Ansicht, um die Mission zu lösen.
Der Zwischenschritt, dass der Benutzer zuerst gefragt wird, ob er die Lösung kennt, wurde weggelassen.
Somit entfällt ein weiterer Klick, was die Spielmechanik vereinfacht.
Nur noch das Marker-Icon auf der Karte verrät etwas über den Missionstyp und weckt dabei immer noch die Neugier beim Benutzer.

%ToDo: Screenshots

\section{Zielerreichung}
Erreichte Ziele:

\begin{itemize}
	\item finale \brand{Android}-App mit gleicher Funktionalität, wie die \gls{WebApp}
	\item \brand{iOS}-App mit gleicher Funktionalität, wie die \gls{WebApp}
	\item neuer Validationsmechanismus
	\item Erfahrungsbericht zu \brand{React-Native}
	\item Internationalisierung umgesetzt
\end{itemize}

\section{Ausblick}
Offene Punkte und nächste geplante Arbeiten mit höherer Priorität:

\begin{itemize}
	\item \brand{OpenStreetMap}-Login
	\item Finale \brand{iOS}-App
	\item Veröffentlichung im \brand{Apple App Store} und \brand{Google Play Store} 
	\item Promotions-Funktion
\end{itemize}

Für die Zukunft gibt es bereits viele weitere Ideen. 
Eine Liste wurde im Kapitel \hyperref[pd-weiterentwicklung]{Weiterentwicklung} erstellt. 
Leider hat es zeitlich auch nicht mehr gereicht um die zweite Highscore-Ansicht zu erstellen. 
Diese Ansicht würde die Ranglistenposition vom Benutzer zentriert in der globalen Highscore anzeigen. 
Hier gab es technische Schwierigkeiten um mit dem verwendeten \gls{Framework} \brand{React Native} eine geschachtelte Tab-Ansicht zu erstellen.

\section{Persönliche Berichte}

\subsubsection{Dominic Mülhaupt}

\subsubsection{Marino Melchiori}
Beim Start von diesem Projekt hatte ich keine \brand{JavaScript}-Vorkenntnisse.
Im Verlauf der Arbeit gelang es mir aber, dank der Zusammenarbeit in unserem Team, mich einzuarbeiten.
Das Highlight dieser Arbeit war für mich die Implementation der App mit \brand{React} und die Gestaltung der Benutzeroberfläche mit \brand{JSX}.
Trotzt all den neuen und teilweise unreifen Technologien, ist es uns gelungen, eine gute App-Idee neu zu entwickeln.
Ich bin froh, dass ich \brand{JavaScript}-Erfahrungen sammeln konnte.