\chapter{Projektmanagement}
\label{pm-projektmanagement}
Hier finden Sie die Angaben zum Projekt, zu den eingesetzten Entwicklungswerkzeugen und zur eingesetzten Software.

\begin{itemize}
    \item Website: \url{http://www.kort.ch/} (\url{http://play.kort.ch/})
    \item Source Code: \url{https://github.com/kort/kort-reloaded/}
    \item Projektmanagement: Redmine (\url{sinv-56059.edu.hsr.ch/redmine/projects/ba-kort/ })
    \item Issues: (\url{http://sinv-56059.edu.hsr.ch/redmine/projects/ba-kort/issues}  (Backlog: im Wiki auf Redmine (mit SK für eigene Issues gekennzeichnet)) 
    \item Dokumentation: \url{https://github.com/kort/kort-reloaded-docu}
\end{itemize}

\section{Team}
\label{pm-team}
\begin{itemize}
	\item \textit{Betreuer}: Prof. Stefan F. Keller
	\item \textit{Projektpartner}: Liip AG, Limmatstrasse 183, CH-8005 Zürich
	\begin{itemize}
		\item Hr. Jürg Hunziker
		\item Hr. Stefan Oderbolz
	\end{itemize}
	\item \textit{Experte}: Hr. Claude Eisenhut
	\item \textit{Gegenleser}: Prof. Beat Stettler
\end{itemize}

\subsection*{Autoren}
Diese Arbeit wird als Bachelorarbeit an der Abteilung Informatik durchgeführt von
\begin{itemize}
	\item Hr. Marino Melchiori
	\item Hr. Dominic Mülhaupt
\end{itemize}


\section{Risikomanagement}
\label{pm-projektmanagement-risikomanagement}
Da \kort{} bereits in einer vorhergehenden Bachelorarbeit erfolgreich umgesetzt werden konnte und Anklang gefunden hat, wurden bereits viele Risiken abgedeckt.\newline
Dennoch wurde das Risikomanagement nicht vernachlässigt.
Alle bekannten Risiken sind gesammelt aufgelistet und wurden nach jedem Meilenstein neu evaluiert.

\subsection{Risikoanalyse}
\label{pm-projektmanagement-risikoanalyse}
Am Anfang der Bachelorarbeit wurden folgende Risiken identifiziert:

\begin{table}[H]
\centering
\label{pm-projektmanagement-risikomanagement-r01}
\begin{tabular}{|p{4.5cm}|p{11cm}|}
\hline
\multicolumn{2}{|l|}{\textbf{R01: Mangelnde Erfahrung mit \brand{JavaScript}, \brand{React} und \brand{React Native}}} \\
\hline
\textbf{Beschreibung} & Wirkt sich negativ auf Design und Programmcode aus. \\
\hline
\textbf{Schadenspotential} & 60h \\
\hline
\textbf{Eintrittswahrsch.} & 65\,\% \\
\hline
\textbf{Auswirkung} & Da die Entwickler sich mit den zugrundeliegenden Programmierkonzepten vertraut machen während sie bereits planen und Entscheidungen treffen, teilweise sogar schon programmieren müssen, kann und wird es vorkommen, dass gewisse Entscheidungen schlecht getroffen und gewisse Konzepte nicht sauber umgesetzt werden. 
Dies kann zu Verspätungen oder unsauberem Programmcode führen. \\
\hline
\textbf{Vorbeugung} & Die Entwickler haben mit dem Betreuer (Prof. Stefan F. Keller) besprochen, dass das \gls{Minimum Viable Product} der Bachelorarbeit eine Android App mit der gleichen Funktionalität, wie sie bereits im ursprünglichen Kort Game umgesetzt wurde, sein wird.
IM Rahmen dieser Arbeit blieb aber schlicht keine Zeit, um sich umfassend in die Technologien einzuarbeiten, bevor die restliche Arbeit angepackt ist.
Aus diesem Grund wird vor allem zu Beginn vermehrt auf \gls{Pair Programming} gesetzt.
Die Entwickler haben sich auch in regelmässigen Abständen die Zeit genommen, ein Code Review durchzuführen.
Ausserdem ist es wichtig, dass bei Schwierigkeiten nicht zu lange gezögert wird, den Kontakt zu Experten im jeweiligen Bereich zu suchen.  \\
\hline
\textbf{Massnahmen beim Eintreffen} & Komplexe Features vereinfachen und so gestalten, dass sie leicht erweiterbar sind. \\
\hline
\end{tabular}
\caption{Risiko R01}
\end{table}

\begin{table}[H]
\centering
\label{pm-projektmanagement-risikomanagement-r02}
\begin{tabular}{|p{4.5cm}|p{11cm}|}
\hline
\multicolumn{2}{|l|}{\textbf{R02: Es existiert keine passende Map Library für \brand{React Native}}} \\
\hline
\textbf{Schadenspotential} & 70h \\
\hline
\textbf{Eintrittswahrsch.} & 30\,\% \\
\hline
\textbf{Auswirkung} & Es müsste statt \brand{React Native} ein alternatives mobiles \gls{Framework} gefunden werden, welches die \brand{OSM}-Daten anzeigen kann. \\
\hline
\textbf{Vorbeugung} & Zu Beginn des Projektes muss eine \brand{React Native} Prototyp-Applikation implementiert werden, welche \brand{OSM}-Daten auf der Karte darstellt.  \\
\hline
\textbf{Massnahmen beim Eintreffen} & Auf eine aufwendigere Variante der Kartendarstellung, die im Kapitel \hyperref[tb-evaluation-karte]{Evaluation} evaluiert wurde, zurückgreifen. \\
\hline
\end{tabular}
\caption{Risiko R02}
\end{table}

\begin{table}[H]
\centering
\label{pm-projektmanagement-risikomanagement-r03}
\begin{tabular}{|p{4.5cm}|p{11cm}|}
\hline
\multicolumn{2}{|l|}{\textbf{R03: \brand{KeepRight} stellt den Dienst ein}} \\
\hline
\textbf{Schadenspotential} & 70h \\
\hline
\textbf{Eintrittswahrsch.} & 10\,\% \\
\hline
\textbf{Auswirkung} & Die Anzahl der zur Verfügung stehenden Missionen würde stark eingeschränkt werden.
Ausserdem wären diese auf die Schweiz beschränkt.
Ein anderer Dienst (z.\,B. \brand{Osmose}) müsste eingesetzt werden, was vor allem Änderungen im Backend erfordern würde. \\
\hline
\textbf{Vorbeugung} & Bereits früh im Projekt wird das Thema mit dem \hyperref[pm-team]{Projektpartner} besprochen, um festzustellen, ob dieser bereit wäre, sich dieser Problematik anzunehmen. \\
\hline
\textbf{Massnahmen beim Eintreffen} & Ausarbeitung einer neuen Aufgabenstellung mit dem Betreuer und den Projektpartnern (Jürg Hunziker und Stefan Oderbolz), die Änderungen am Backend beinhaltet. \\
\hline
\end{tabular}
\caption{Risiko R03}
\end{table}

\begin{table}[H]
\centering
\label{pm-projektmanagement-risikomanagement-r04}
\begin{tabular}{|p{4.5cm}|p{11cm}|}
\hline
\multicolumn{2}{|l|}{\textbf{R04: \brand{React Native} ist noch nicht ausgereift genug für \brand{Android}}} \\
\hline
\textbf{Schadenspotential} & 20h \\
\hline
\textbf{Eintrittswahrsch.} & 30\,\% \\
\hline
\textbf{Auswirkung} & \brand{Android} wird erst seit Oktober 2015 durch \brand{React Native} unterstützt. 
Einige Features von Komponenten stehen nur für \brand{iOS} zur Verfügung. \\
\hline
\textbf{Vorbeugung} & Bevor mit der Implementation begonnen wurde, sind die mangelnden Funktionalitäten (z.\,B. im Testing), so weit möglich, analysiert.
Grundsätzlich sollte es möglich sein, die App für \brand{Android} umzusetzen, da bereits komplexere Apps damit erstellt wurden.
Wahrscheinlich wird es vorkommen, dass Workarounds nötig sein werden. 
Im schlimmsten Fall müsste während der Entwicklung auf \brand{iOS} umgestellt werden, was grundsätzlich möglich ist. \\
\hline
\textbf{Massnahmen beim Eintreffen} & Ausarbeitung einer neuen Aufgabenstellung mit dem Betreuer, die eine \brand{iOS}-Version bevorzugt. \\
\hline
\end{tabular}
\caption{Risiko R04}
\end{table}

\begin{table}[H]
\centering
\label{pm-projektmanagement-risikomanagement-r05}
\begin{tabular}{|p{4.5cm}|p{11cm}|}
\hline
\multicolumn{2}{|l|}{\textbf{R05: Mangelnde Erfahrung mit der flux-Architektur}} \\
\hline
\textbf{Beschreibung} & Wirkt sich negativ auf Design und Programmcode aus. \\
\hline
\textbf{Schadenspotential} & 30h \\
\hline
\textbf{Eintrittswahrsch.} & 40\,\% \\
\hline
\textbf{Auswirkung} & Da die Konzepte der flux-Architektur neu erarbeitet werden müssen und flux in der Community als eher komplex eingestuft wird, kann die Entwicklung mehr Zeit beanspruchen. \\
\hline
\textbf{Vorbeugung} & Die korrekte Umsetzung wird bereits vor dem Start der Implementation so weit wie möglich analysiert.
Dabei wurden die Themengebiete zum Einlesen aufgeteilt und schlussendlich besprochen und erklärt. \\
\hline
\textbf{Massnahmen beim Eintreffen} & Rücksprache mit dem IFS-Team, das an einem \brand{React}-Projekt arbeitet. \\
\hline
\end{tabular}
\caption{Risiko R05}
\end{table}

\begin{table}[H]
\centering
\label{pm-projektmanagement-risikomanagement-r06}
\begin{tabular}{|p{4.5cm}|p{11cm}|}
\hline
\multicolumn{2}{|l|}{\textbf{R06: Mangelnde Erfahrung mit \gls{OAuth}}} \\
\hline
\textbf{Beschreibung} & Keiner der Entwickler ist mit \gls{OAuth} vertraut. Dadurch, dass sich die Implementation der Authentifizierung bei einem native Client zu einer Implementation einer \gls{WebApp} unterscheidet (Token- vs. Session-based), wird mehr Zeit beansprucht. \\
\hline
\textbf{Schadenspotential} & 40h \\
\hline
\textbf{Eintrittswahrsch.} & 60\,\% \\
\hline
\textbf{Auswirkung} & Die \brand{OSM}-Community würde nur sehr ungern auf einen entsprechenden Login verzichten. \\
\hline
\textbf{Vorbeugung} & Mehr Aufwand in die Evaluation stecken. \\
\hline
\textbf{Massnahmen beim Eintreffen} & Rücksprache mit dem Betreuer. \\
\hline
\end{tabular}
\caption{Risiko R06}
\end{table}

\subsubsection{Risikoanalyse MS2: Ende Elaboration}
Am 29. März 2016 (Fälligkeitsdatum des Meilensteins) wurden die Risiken erneut besprochen.

\textbf{R01: Mangelnde Erfahrung mit \brand{JavaScript}, \brand{React} und \brand{React Native}}: Die Eintrittswahrscheinlichkeit konnte auf 50\,\% gesenkt werden. 

\textbf{R03: \brand{KeepRight} stellt den Dienst ein}: Wurde nach Absprache mit Jürg Hunziker und Stefan Oderbolz ganz eliminiert.

\subsubsection{Risikoanalyse MS3: Evaluation der Komponenten}
\textbf{R02: Es existiert keine passende Map Library für \brand{React Native}}: Die Wahrscheinlichkeit des Eintretens konnte nach erfolgreicher Implementation auf 20\,\% gesenkt werden. Dadurch, dass noch keine Missionen gelöst wurden, galt das Risiko noch nicht als behoben.

\subsubsection{Risikoanalyse MS4: Zwischenpräsentation}
Das Eintreten von \textbf{R02} konnte weiterhin auf 10\,\% gesenkt werden. Die Marker waren klickbar und sie enthielten alle nötigen Informationen für das Lösen einer Mission.

\subsubsection{Risikoanalyse MS5: Basis-Komponenten umgesetzt}
\textbf{R02} wurde eliminiert, \textbf{R01} auf 20\,\% gesenkt und \textbf{R05} als tief eingeschätzt (10\,\%).

\subsubsection{Risikoanalyse MS6: Beta-Release mit Grundfunktionalität}
\textbf{R05} wurde ebenfalls eliminiert -- die flux-Architektur war stabil und eine Mission konnte erfolgreich gelöst werden.

\subsubsection{Risikoanalyse MS7: Schlussabgabe}
\textbf{R04: \brand{React Native} ist noch nicht ausgereift genug für \brand{Android}}: Wurde dank einer lauffähigen Version auf 10\,\% gesenkt.

\textbf{R06: Mangelnde Erfahrung mit \gls{OAuth}}: Dies ist ein weiterhin bestehendes Risiko -- die \brand{OSM}-Authentifizierung konnte leider nicht umgesetzt werden.


\section{Projektplan}

Die Planung wurde nach dem ersten \hyperref[pm-ms1]{Kickoff-Meeting} festgehalten. 
Der erste Prototyp, für die Demo an der Zwischenpräsentation, war am 11.04.2016 geplant. 
Am 16.05. war dann das erste Release, mit dem Missionen gelöst werden können, vorgesehen. 
Um bei der Abgabe vom Projekt die App zu veröffentlichen, entstand der folgende Projektplan.

\begin{figure}[H]
	\centering
	\includegraphics[width=\textwidth]{images/projektmanagement/zeitstrahl_v1.png}
	\caption{1. Projektplan als Zeitstrahl dargestellt}
	\label{image-project-plan-timeline1}
\end{figure}

Beim dritten Sprint wurde festgestellt, dass die Planung zu optimistisch war. 
Es gab Schwierigkeiten bei der Evaluation der Architektur und von den Libraries. 
Die Libraries und Komponenten wurden in einzelnen Projekten getestet. 
Das hat uns zu neuen Erkenntnissen für die \hyperref[tb-evaluation-architektur]{Architekturentscheidung} verholfen. 

Erst beim \hyperref[pm-ms5]{Meilenstein 5} konnte eine Version mit den definitiv evaluierten Komponenten entstehen. 
Beim achten Meeting, am 29.04.2016, wurde mit Herrn Stefan Keller besprochen, dass wir das Projekt nach der Abgabe gerne weiterführen würden.
Dann wäre es nämlich auch möglich, die App zu veröffentlichen.

Bis zum Ende vom \hyperref[pm-ms6]{Meilenstein 6} entstand ein erstes Release, von einem Prototypen mit der geplanten Grundfunktionalität. 
Bei der Abgabe ist das Projekt in einem Zustand, bei dem sich neue Features zügig implementieren lassen.
Dieser aktuelle Stand wäre am Anfang vom Projekt für den "Meilenstein 5" geplant gewesen.

Somit ist dieser neue Projektplan entstanden.

%ToDo: Bild-Pfad zu "..._v2" ändern
\begin{figure}[H]
	\centering
	\includegraphics[width=\textwidth]{images/projektmanagement/zeitstrahl_v1.png}
	\caption{2. Projektplan als Zeitstrahl dargestellt}
	\label{image-project-plan-timeline2}
\end{figure}

% Anwendung von Scrum in unserem Projekt
\subsection{Sprints}
Durch die Anwendung von Scurm-Ansätzen, wurden Sprints geplant. 
Eine Sprint-Periode dauerte immer bis zum Ende von einem Meilenstein. 
Am Anfang von einem Sprint wurde das konkrete Vorgehen geplant und am jeweiligen Ende sind die Schwerpunkte dokumentiert worden.

\subsubsection{Sprint 1}
Zum Schwerpunkt in diesem Sprint gehörte die Einarbeitung in die verwendeten Technologien und die Einarbeitung in das bestehende \kort{}-Projekt.


\subsubsection{Sprint 2}
Im zweiten Sprint wurden die benötigten Libraries evaluiert und getestet.
Die Details und die Begründungen zu den Entscheidungen wurden im \hyperref[tb-evaluation]{Kapitel Evaluation} festgehalten.


\subsubsection{Sprint 3}
Neben der Vorbereitung der Zwischenpräsentation wurden die Anforderungen aktualisiert. 
Das ersetzen der Validationen im Frontend kam neu dazu. 
Zukünftige Arbeiten und Ideen wurden ebenfalls gesammelt und im \hyperref[pd-weiterentwicklung-realistisch]{Kapitel Realistische Arbeiten} dokumentiert.


\subsubsection{Sprint 4}
Am Anfang vom Sprint vier wurde abgeklärt ob ein Webprototyp mit \brand{React} umsetzbar ist. 
Dies stellte sich dann aber als zu Aufwendig heraus --- Die Idee (für eine folgende Studienarbeit) einer \brand{React}-\gls{WebApp} wurde Abgewiesen.
Am Ende des Sprints wurde festgelegt, was getestet werden muss.

\subsubsection{Sprint 5}
Während diesem Sprint fand das Meeting mit Jürg Hunziker und Stefan Oderbolz, um die Authentifizierung abzuklären. 
Es ging darum, neu zu evaluieren, wie das Login in der App umgesetzt wird und was dies für Änderungen am Backend mit sich bringen würde. 

\subsubsection{Sprint 6}
Die Validationen konnten im Frontend, anhand von Anpassung der Businesslogik, abgeschafft werden.
Ausserdem konnte die wichtige Internationalisierung für die Sprachen Englisch und Deutsch umgesetzt werden. 
Am Ende von Sprint 6 wurde ein Code Freeze bis zum Abgabetermin eingeleitet.

\section{Meilensteine}
\subsection{MS1: Kickoff}
\label{pm-ms1}
\textbf{Fällig am 25.02.2016}
\subsubsection{Resultate}
\begin{itemize}
	\item Kickoff Meeting bei Liip mit Jürg Hunziker, Stefan Oderbolz und Stefan Keller
\end{itemize}

\subsection{MS2: Ende Elaboration}
\label{pm-ms2}
\textbf{Fällig am 29.03.2016}
\subsubsection{Resultate}
\begin{itemize}
	\item Ausgangslage
	\item Anforderungsspezifikation
	\item Risikomanagement
	\item Projektplan
	\item Map Komponente ausgewählt und eingesetzt
	\item Aufgabenstellung
	\item Testspezifikation
	\item Infrastruktur
	\begin{itemize}
		\item Datenbank
		\item CI
		\item uservoice
		\item Redmine
		\item Installationsskripte
	\end{itemize}
\end{itemize}

\subsubsection{Erledigte Arbeiten}
Vollständig dokumentiert wurden die Ausgangslage, die Anforderungsspezifikation, der Projektplan und das Risikomanagement.
Die Aufgabenstellung wurde soweit definiert, dass sie mit dem Projektplan übereinstimmt.
Die endgültige Aufgabenstellung ist erst zur Zwischenpräsentation, in der Mitte des Semesters, geplant.
Für die Map Komponente wurde die MapBox GL Library\footnote{\url{https://libraries.io/npm/react-native-mapbox-gl}} mit den Vektor Daten von osm2vectortiles\footnote{\url{http://osm2vectortiles.org/}}.
Nebenbei konnten wichtige Erfahrungen in den Technologien (\brand{Java Script}, \brand{React}, \brand{React Native}) gemacht werden.
Zusätzlich wurde ein für \brand{Android}-Geräte angepasstes Design entworfen und Grundkonzepte der Architektur erarbeitet.
Die Architektur ist noch nicht final, sie erleichtert uns aber den Einstig beim Programmieren.
Die Infrastruktur ist soweit aufgesetzt.
Es fehlten zu diesem Zeitpunkt nur noch die Zugangsdaten, welche aber noch nicht dringend gebraucht wurden.

\subsubsection{Probleme}
Für die detaillierte Erarbeitung der Testspezifikation fehlte noch die nötige Erfahrung in \brand{React Native}.


\subsection{MS3: 1. Prototyp}
\label{pm-ms3}
\textbf{Fällig am 08.04.2016}
\subsubsection{Resultate}
\begin{itemize}
	\item \brand{Android} Prototyp
	\begin{itemize}
		\item Darstellung der Karte
		\item Missionen auf Map anzeigen
		\item falls möglich:
		\begin{itemize}
			\item Highscore
			\item Profile
			\item About Kort
		\end{itemize}
	\end{itemize}
\end{itemize}

\subsection{MS4: Zwischenpräsentation}
\label{pm-ms4}
\textbf{Fällig am 22.04.2016}
\subsubsection{Resultate}
\begin{itemize}
	\item 1. Prototyp ausgebaut
	\item Zwischenpräsentation (Dauer ca. 1h)
	\begin{itemize}
		\item Präsentation der Ergebnisse
		\item Diskussion bezüglich weiterer Arbeit
	\end{itemize}
\end{itemize}

\subsection{MS5: 1. Release}
\label{pm-ms5}
\textbf{Fällig am 13.05.2016}
\subsubsection{Resultate}
\begin{itemize}
	\item die Funktionalität der ursprünglichen \kort{} Web-App ist als Native App implementiert
\end{itemize}

\subsection{MS6: 2. Release}
\label{pm-ms6}
\textbf{Fällig am 03.06.2016}
\subsubsection{Resultate}
\begin{itemize}
	\item die Native App enthält alle vorgegebenen Features
	\item sämtliche Tests wurden erfolgreich durchlaufen
\end{itemize}

\subsection{MS7: Schlussabgabe}
\label{pm-ms7}
\textbf{Fällig am 17.06.2016}
\subsubsection{Resultate}
\begin{itemize}
	\item Dokumentation vollständig
	\item gebundene Dokumentation eingereicht
	\item CD mit Abgabe eingereicht
	\item Abstract eingereicht
	\item Poster eingereicht
	\item evtl. App deployed
\end{itemize}