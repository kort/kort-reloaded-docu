\section{Sprints}
Durch die Anwendung von Scurm-Ansätzen, wurden Sprints geplant. 
Eine Sprint-Periode dauerte immer bis zum Ende von einem Meilenstein. 
Am Anfang von einem Sprint wurde das konkrete Vorgehen geplant und am jeweiligen Ende sind die Schwerpunkte dokumentiert worden.

% ToDo: Daten anpassen

\subsection{Sprint 1}
\textbf{14.03.2016 bis 30.03.2016}

Zum Schwerpunkt in diesem Sprint gehörte die Einarbeitung in die verwendeten Technologien und die Einarbeitung in das bestehende \kort{}-Projekt.

\subsection{Sprint 2}
\textbf{30.03.2016 bis 11.04.2016}

Im zweiten Sprint wurden die benötigten Libraries evaluiert und getestet.
Die Details und die Begründungen zu den Entscheidungen wurden im \hyperref[tb-evaluation]{Kapitel Evaluation} festgehalten.

\subsection{Sprint 3}
\textbf{11.04.2016 bis 25.04.2016}

Neben der Vorbereitung der Zwischenpräsentation wurden die Anforderungen aktualisiert. 
Das ersetzen der Validationen im Frontend kam neu dazu. 
Zukünftige Arbeiten und Ideen wurden ebenfalls gesammelt und im \hyperref[pd-weiterentwicklung-realistisch]{Kapitel Realistische Arbeiten} dokumentiert.

\subsection{Sprint 4}
\textbf{25.04.2016 bis 16.05.2016}

Am Anfang vom Sprint vier wurde abgeklärt ob ein Webprototyp mit \brand{React} umsetzbar ist. 
Dies stellte sich dann aber als zu Aufwendig heraus --- Die Idee (für eine folgende Studienarbeit) einer \brand{React}-\gls{WebApp} wurde Abgewiesen.
Am Ende des Sprints wurde festgelegt, was getestet werden muss.

\subsection{Sprint 5}
\textbf{16.05.2016 bis 06.06.2016}

Während diesem Sprint fand das Meeting mit Jürg Hunziker und Stefan Oderbolz, um die Authentifizierung abzuklären. 
Es ging darum, neu zu evaluieren, wie das Login in der App umgesetzt wird und was dies für Änderungen am Backend mit sich bringen würde. 

\subsection{Sprint 6}
\textbf{06.06.2016 bis 17.06.2016}

Die Validationen konnten im Frontend, anhand von Anpassung der Businesslogik, abgeschafft werden.
Ausserdem konnte die wichtige Internationalisierung für die Sprachen Englisch und Deutsch umgesetzt werden. 
Am Ende von Sprint 6 wurde ein Code Freeze bis zum Abgabetermin eingeleitet.
