\section{Meilensteine}
\label{pm-meilensteine}

In diesem Projekt wurde die Planung auf neun Meilensteine (MS) aufgeteilt. 
Die Meilensteine \hyperref[pm-ms8]{MS 8} und \hyperref[pm-ms9]{MS 9} finden nach der offiziellen Schlussabgabe statt und sind deswegen noch ohne Resultate dokumentiert.

\subsection{MS1: Kickoff}
\label{pm-ms1}
\textbf{Fällig am 25.02.2016}
\subsubsection{Resultate}
\begin{itemize}
	\item Kickoff Meeting bei Liip mit den Projektpartnern (Jürg Hunziker und Stefan Oderbolz) und dem Betreuer (Prof. Stefan F. Keller)
\end{itemize}

\subsection{MS2: Ende Elaboration}
\label{pm-ms2}
\textbf{Fällig am 29.03.2016}
\subsubsection{Resultate}
\begin{itemize}
	\item Infrastruktur aufgesetzt
	\begin{itemize}
		\item Datenbank
		\item \gls{CI}
		\item \brand{UserVoice}
		\item Redmine
		\item Installationsskripte
		\item Dokumentation aufgesetzt
	\end{itemize}
	\item Ausgangslage definiert und dokumentiert
	\item Anforderungsspezifikation erarbeitet
	\item Risikomanagement analysiert und dokumentiert
	\item Projektplan erarbeitet
	\item Map Komponente ausgewählt und eingesetzt
	\item Aufgabenstellung erarbeitet
	\item Testspezifikation erarbeitet
	\item Einarbeitung in die \kort{}-\gls{WebApp}
	\item \gls{GUI}-Mockups
\end{itemize}

\subsubsection{Erledigte Arbeiten}
Eine erste Version der Aufgabenstellung konnte erarbeitet werden. 
Die Dokumentation wurde eingeleitet.
Es gelang uns einen automatisierten Build der App aus dem \brand{GitHub}-Sourcecode mit \brand{Travis CI} aufzusetzen.
Für die Map-Komponente wurde die MapBox GL \gls{Library}\footnote{\url{https://libraries.io/npm/react-native-mapbox-gl}} mit den Vektor Daten von osm2vectortiles\footnote{\url{http://osm2vectortiles.org/}} evaluiert und getestet.
Nebenbei konnten wichtige Erfahrungen in den verwendeten Technologien (\brand{JavaScript}, \brand{React} und \brand{React Native}) gemacht werden.
Zusätzlich wurde ein für \brand{Android} angepasstes Design entworfen und Grundkonzepte der Architektur erarbeitet.
Die Architektur ist noch nicht final. 
Sie erleichtert uns aber den Einstieg beim Programmieren.
Die Infrastruktur ist soweit aufgesetzt.

\subsubsection{Probleme}
Für die detaillierte Erarbeitung der Testspezifikation fehlte noch die nötige Erfahrung in \brand{React Native}.


\subsection{MS3: Evaluation der Komponenten}
\label{pm-ms3}
\textbf{Fällig am 08.04.2016}
\subsubsection{Resultate}
\begin{itemize}
	\item Einarbeitung in \brand{JavaScript}, \brand{React} und \brand{React Native}
	\item \brand{Android} Prototyp
	\begin{itemize}
		\item Tab-Navigation
		\item Darstellung der Karte
		\item Evaluation der Architektur und Implementation des Grundgerüstes
		\item Missionen auf Map anzeigen
		\item Authentifizierung mit \gls{OAuth} evaluiert
	\end{itemize}
	\item \brand{Travis-CI} Konfiguration aktualisieren
\end{itemize}

\subsubsection{Erledigte Arbeiten}
Die Entwicklungsumgebung wurde optimiert. Das \brand{Travis}-Konfigurationsfile prüft den Code nun mit \brand{ESLint}\footnote{\url{http://eslint.org/}} (\brand{JavaScript} linter, prüft Styleguidelines) und \brand{flow}\footnote{\url{http://flowtype.org/}} (static type checker). 

Für die Tab-Navigation konnte eine Demo mit react-native-router-flux\footnote{\url{https://github.com/aksonov/react-native-router-flux}} umgesetzt werden. Diese muss noch in der \kort{} App implementiert werden. 

\gls{OAuth} (nur clientseitig, ohne Backend Kommunikation) wurde evaluiert und konnte dann anhand einer Demo getestet werden -- allerdings nur mit \brand{Facebook} und \brand{Google}.

Die Darstellung der Karte aus \nameref{pm-ms2} wurde leicht ausgebaut. Neu wird nun der Standort des Benutzers ermittelt.

Die Missionen konnten im Umkreis von fünf Kilometern geladen werden.

\subsubsection{Probleme}
Schwierigkeiten traten vor allem im Zusammenhang mit \brand{React Native} auf.
Oft gab es Build-Fehler bei gleicher Code-Basis.
Die Fehlerbehandlung hat uns enorm viel Zeit gekostet und es war schwer, im Internet Hilfestellungen zu erhalten.
Da \brand{React Native} alle zwei Wochen ein Update erhält, sind die Dokumentationen oder die Diskussionen im Internet teilweise schon wieder veraltet.\newline
Im Austausch mit anderen \brand{React Native} Entwicklern haben wir dasselbe Feedback erhalten.\newline
Der Prototyp konnte nicht wie geplant fertiggestellt werden und einige Arbeiten mussten auf den nächsten Meilenstein ausgelagert werden.

Die Authentifizierung mit dem Backend konnte noch nicht umgesetzt werden.

\subsection{MS4: Zwischenpräsentation}
\label{pm-ms4}
\textbf{Fällig am 22.04.2016}
\subsubsection{Resultate}
\begin{itemize}
	\item \brand{Mapbox} Prototyp fertig
	\begin{itemize}
		\item Missionen mit Marker auf Karte dargestellt
		\item Tab-Navigation implementiert
	\end{itemize}
	\item Zwischenpräsentation (Dauer ca. 30 Minuten)
	\begin{itemize}
		\item Aufgabenstellung, Problembesprechung
		\item IST-Situation
		\item geplantes Resultat
		\item Beschlussprotokoll für den \hyperref[pm-team]{Betreuer}
	\end{itemize}
\end{itemize}

\subsubsection{Erledigte Arbeiten}
Der Prototyp enthielt die Karte in einer Tab-Ansicht. 
An den jeweiligen Positionen der geladenen Missionen konnten Marker eingefügt werden.
Durch einen Klick auf einen Marker konnte der Titel der Mission erfolgreich in die Konsole geloggt werden.

Leider konnte noch immer keine Lösung für die Implementation eines Logins mit \gls{OAuth} 1.0a (\brand{OSM}) gefunden werden.
Dieses Feature noch zur Abgabe zu liefern wäre unrealistisch.
Deswegen wurde es auf den \hyperref[pm-ms9]{Meilenstein 9: Release für App Store} verschoben.

\subsubsection{Probleme}
Weiterhin traten Schwierigkeiten mit willkürlichen Fehlern der \brand{React Native} App auf. 
Aus diesen Gründen konnte die Navigation und die Struktur der App nicht abschliessend umgesetzt werden.
Dass der Benutzer nach dem Login automatisch zur Tab-Ansicht weitergeleitet wurde, hatte aufgrund eines Fehlers in der Navigations-\gls{Library} nicht geklappt.
Zwischenzeitlich wurde aus diesem Grund das Erhalten des Login-Tokens nach der Authentifizierung über \brand{Google} und \brand{Facebook} in einer separaten App getestet.


\subsection{MS5: Basis-Komponenten umgesetzt}
\label{pm-ms5}
\textbf{Fällig am 20.05.2016}
\subsubsection{Resultate}
\begin{itemize}
	\item Native Location Tracking implementiert
	\item View-Komponenten fertig entworfen
	\item Testing-\gls{Framework} aufgesetzt
	\item Architektur-Entscheid und -Implementation
	\item Token basierte \brand{Google}-Authentifizierung im Frontend umgesetzt, nachdem das Backend vom Projektpartner, Stefan Oderbolz und Jürg Hunziker, dafür angepasst wurde
	\item \kort{}-Datenbank für Tests lokal aufgesetzt
	\item \kort{} ist \brand{iOS}-fähig
	\item \brand{React}-\gls{WebApp} evaluiert
	\item Validationen vom Backend als Missionen laden
	\item Kurzvideo-Konzept besprochen
\end{itemize}

\subsubsection{Erledigte Arbeiten}

Wir haben die Architekturvarianten evaluiert und uns für die \brand{Flux}-Architektur entschieden.
Daraufhin wurde das bestehende -- noch sehr schlanke -- Grundgerüst, welches mit dem MVC-Pattern umgesetzt wurde, durch \brand{Flux} ersetzt.

Unterdessen konnte das Grundgerüst des \gls{GUI} grösstenteils fertiggestellt werden. 
Es fehlte noch die dynamische Behandlung von gewissen View-Komponenten.
Zum Beispiel die Unterscheidung, ob beim Lösen einer Mission ein Picker zum Auswählen der Antwort gerendert wird, oder ob ein Text-Input-Feld angeboten wird.

\kort{} konnte erfolgreich auf \brand{iOS} getestet werden. 

\subsubsection{Probleme}
Weiterhin war es nicht möglich mit der Router-Flux-Navigationskomponente eine Weiterleitung nach erfolgreichem Login zur Kartenansicht umzusetzen.
Hierbei handelte es sich um den bekannten Fehler dieser Komponente.
Das verzögerte leider das geplante Ziel: eine Mission zu lösen.

Die \kort{}-Datenbank konnte nur bei einem Entwickler richtig aufgesetzt werden. 
Es gab Probleme mit der Installation der Scripts auf \brand{OS X}.
Aus diesem Grund wurde für Tests nach Absprache mit dem Projektpartner (Jürg Hunziker), das Development-\kort{}-Backend genutzt.

\subsection{MS6: Beta-Release mit Grundfunktionalität}
\label{pm-ms6}
\textbf{Fällig am 06.06.2016}
\subsubsection{Resultate}
\begin{itemize}
	\item dynamische View-Komponenten fertiggestellt
	\item Login Weiterleitung
	\item Login-Logik mit Local Storage
	\item Testing abgeschlossen
	\item Missionen und Validationen sind lösbar
	\item Internationalisierung umgesetzt
\end{itemize}

\subsubsection{Erledigte Arbeiten}
Alle View-Komponenten sind nun dynamisch und zeigen je nach Zustand der Komponente die entsprechende Oberfläche.

Durch eine zusätzliche App-Loader-View konnte das Problem der Weiterleitung nach dem Login behoben werden.
Diese Komponente lädt nun alles im Voraus und erst danach wird je nach Zustand, ob der Benutzer eingeloggt ist oder nicht, die entsprechende Ansicht angezeigt.
Nach und nach wurden die Platzhalter der View-Komponenten mit dynamischen Daten der Businesslogik ersetzt.

Schlussendlich konnten wir erfolgreich Missionen und Validationen lösen.


\subsubsection{Probleme}
% ToDo



\subsection{MS7: Schlussabgabe}
\label{pm-ms7}

\textbf{Fällig am 17.06.2016}

\subsubsection{Resultate}

\begin{itemize}
	\item Gebundene, vollständige Dokumentation eingereicht
	\item CD mit Abgabe eingereicht
	\item Abstract eingereicht
	\item Poster eingereicht
\end{itemize}


\subsection{MS8: Schlusspräsentation}
\label{pm-ms8}

\textbf{Fällig am 29.06.2016}

\subsubsection{Ziele}

\begin{itemize}
	\item Badges anzeigen
	\item Aktualisierte Dokumentation
	\item Design Verbesserungen
	\item Präsentation fertiggestellt
	\item Handouts ausgedruckt
\end{itemize}

\subsection{MS9: Release für App Store}
\label{pm-ms9}

% ToDo: Datum

\textbf{Fällig am 03.07.2016}

\subsubsection{Ziele}

\begin{itemize}
	\item \brand{Android} App im \brand{Google Play} Store veröffentlicht
\end{itemize}
