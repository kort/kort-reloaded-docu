\section{Meilensteine}
\subsection{MS1: Kickoff}
\label{pm-ms1}
\textbf{Fällig am 25.02.2016}
\subsubsection{Resultate}
\begin{itemize}
	\item Kickoff Meeting bei Liip mit Jürg Hunziker, Stefan Oderbolz und Stefan Keller
\end{itemize}

\subsection{MS2: Ende Elaboration}
\label{pm-ms2}
\textbf{Fällig am 29.03.2016}
\subsubsection{Resultate}
\begin{itemize}
	\item Ausgangslage
	\item Anforderungsspezifikation
	\item Risikomanagement
	\item Projektplan
	\item Map Komponente ausgewählt und eingesetzt
	\item Aufgabenstellung
	\item Testspezifikation
	\item Infrastruktur
	\begin{itemize}
		\item Datenbank
		\item CI
		\item uservoice
		\item Redmine
		\item Installationsskripte
	\end{itemize}
\end{itemize}

\subsubsection{Erledigte Arbeiten}
Dokumentiert wurden die Ausgangslage, die Anforderungsspezifikation, der Projektplan und das Risikomanagement.
Die Aufgabenstellung wurde soweit definiert, dass sie mit dem Projektplan übereinstimmt.
Die endgültige Aufgabenstellung ist erst zur Zwischenpräsentation, in der Mitte des Semesters, geplant.
Für die Map Komponente wurde die MapBox GL Library\footnote{\url{https://libraries.io/npm/react-native-mapbox-gl}} mit den Vektor Daten von osm2vectortiles\footnote{\url{http://osm2vectortiles.org/}}.
Nebenbei konnten wichtige Erfahrungen in den Technologien (\brand{Java Script}, \brand{React}, \brand{React Native}) gemacht werden.
Zusätzlich wurde ein für \brand{Android} angepasstes Design entworfen und Grundkonzepte der Architektur erarbeitet.
Die Architektur ist noch nicht final, sie erleichtert uns aber den Einstig beim Programmieren.
Die Infrastruktur ist soweit aufgesetzt.
Es fehlten zu diesem Zeitpunkt nur noch die Zugangsdaten, welche aber noch nicht dringend gebraucht wurden.

\subsubsection{Probleme}
Für die detaillierte Erarbeitung der Testspezifikation fehlte noch die nötige Erfahrung in \brand{React Native}.


\subsection{MS3: 1. Prototyp}
\label{pm-ms3}
\textbf{Fällig am 08.04.2016}
\subsubsection{Resultate}
\begin{itemize}
	\item \brand{Android} Prototyp
	\begin{itemize}
		\item Tab Navigation
		\item Darstellung der Karte
		\item Missionen auf Map anzeigen
		\item Authentifizierung mit OAuth
	\end{itemize}
\end{itemize}

\subsubsection{Erledigte Arbeiten}
Die Entwicklungsumgebung wurde besser eingerichtet – das \brand{Travis} Konfigurationsfile verfeinert, \brand{ESLint}\footnote{\url{http://eslint.org/}} und \brand{flow}\footnote{\url{http://flowtype.org/}} aufgesetzt und konfiguriert.\newline
Für die Tab Navigation konnte eine Demo mit \brand{react-native-scrollable-tab-view}\footnote{\url{https://github.com/brentvatne/react-native-scrollable-tab-view}} umgesetzt werden. Diese muss noch in der \kort{} App eingebaut werden.\newline
Die Authentifizierung mit \brand{OAuth} konnte ebenfalls anhand einer Demo getestet werden – allerdings nur mit \brand{Facebook} und \brand{Google}, \brand{OpenStreetMap} erfordert noch zusätzliche Arbeiten.\newline
Die Darstellung der Karte aus \nameref{pm-ms2} wurde leicht ausgebaut. Neu wird nun der Standort des Benutzers ermittelt und die Missionen im Umkreis geladen, allerdings noch nicht dargestellt.

\subsubsection{Probleme}
Schwierigkeiten traten vor allem im Zusammenhang mit \brand{React Native} auf.
Oft konnte das Projekt aus scheinbar willkürlichen Gründen nicht kompiliert werden.
Die Fehlerbehandlung hat uns enorm viel Zeit gekostet.
Im Austausch mit anderen \brand{React Native} Entwicklern haben wir dasselbe Feedback erhalten.\newline
Des Weiteren ist es schwer, im Internet Hilfestellungen zu erhalten.
Dies liegt vor allem daran, dass \brand{React Native} noch immer in den Kinderschuhen steckt und ständig weiterentwickelt wird.
Deshalb ist die Dokumentation oder die Diskussionen im Internet teilweise schon wieder veraltet oder noch unvollständig.\newline
Der Prototyp konnte nicht wie geplant fertiggestellt werden und einige Arbeiten müssen auf den nächsten Meilenstein ausgelagert werden.

\subsection{MS4: Zwischenpräsentation}
\label{pm-ms4}
\textbf{Fällig am 22.04.2016}
\subsubsection{Resultate}
\begin{itemize}
	\item 1. Prototyp fertig (s. \nameref{pm-ms3})
	\begin{itemize}
		\item OAuth mit \brand{Google} und \brand{Facebook}
		\item Missionen auf Karte darstellen (Annotations)
		\item Tab Navigation
	\end{itemize}
	\item Zwischenpräsentation (Dauer ca. 30 Minuten)
	\begin{itemize}
		\item Aufgabenstellung, Problembesprechung
		\item IST-Situation
		\item geplantes Resultat
		\item Beschlussprotokoll für den \hyperref[pm-rollen]{Betreuer}
	\end{itemize}
\end{itemize}

\subsubsection{Erledigte Arbeiten}


\subsubsection{Probleme}
Weiterhin traten Schwierigkeiten mit willkürlichen Fehlern der \brand{React Native} App auf. 
Wir hatten beim gleichen Stand mit unterschiedlichen Fehlern zu kämpfen.
Aus diesen Gründen konnte die Navigation und die Struktur der App nicht abschliessend umgesetzt werden.
Das gilt auch für die Authentifizierung, mit \brand{Google} und \brand{Facebook}, welche in einer eigenständigen App getestet wurde.

\subsection{MS5: 1. Release}
\label{pm-ms5}
\textbf{Fällig am 13.05.2016}
\subsubsection{Resultate}
\begin{itemize}
	\item die Funktionalität der ursprünglichen \kort{} \gls{WebApp} ist als \gls{NativeApp} implementiert
\end{itemize}

\subsection{MS6: 2. Release}
\label{pm-ms6}
\textbf{Fällig am 03.06.2016}
\subsubsection{Resultate}
\begin{itemize}
	\item die Native App enthält alle vorgegebenen Features
\end{itemize}

\subsection{MS7: Schlussabgabe}
\label{pm-ms7}
\textbf{Fällig am 17.06.2016}
\subsubsection{Resultate}
\begin{itemize}
	\item Dokumentation vollständig
	\item gebundene Dokumentation eingereicht
	\item CD mit Abgabe eingereicht
	\item Abstract eingereicht
	\item Poster eingereicht
	\item evtl. App deployed
\end{itemize}