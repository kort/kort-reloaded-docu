\section{Meilensteine}
\label{pm-ms}
\subsection{MS1: Kickoff}
\label{pm-ms1}
\textbf{Fällig am 25.02.2016}
\subsubsection{Resultate}
\begin{itemize}
	\item Kickoff Meeting bei Liip mit Jürg Hunziker, Stefan Oderbolz und Stefan Keller
\end{itemize}

\subsection{MS2: Ende Elaboration}
\label{pm-ms2}
\textbf{Fällig am 29.03.2016}
\subsubsection{Resultate}
\begin{itemize}
	\item Ausgangslage
	\item Anforderungsspezifikation
	\item Risikomanagement
	\item Projektplan
	\item Map Komponente ausgewählt und eingesetzt
	\item Aufgabenstellung
	\item Testspezifikation
	\item Infrastruktur
	\begin{itemize}
		\item Datenbank
		\item CI
		\item uservoice
		\item Redmine
		\item Installationsskripte
	\end{itemize}
\end{itemize}

\subsubsection{Erledigte Arbeiten}
Eine erste Version der Aufgabenstellung konnte erarbeitet werden. 
Die Dokumentation wurde eingeleitet.
Es gelang uns einen automatisierten Build der App, aus dem \brand{GitHub}-Sourcecode, mit \brand{Travis CI} aufzusetzen.
Für die Map Komponente wurde die MapBox GL Library\footnote{\url{https://libraries.io/npm/react-native-mapbox-gl}} mit den Vektor Daten von osm2vectortiles\footnote{\url{http://osm2vectortiles.org/}} evaluiert und getestet.
Nebenbei konnten wichtige Erfahrungen in den verwendeten Technologien (\brand{Java Script}, \brand{React} und \brand{React Native}) gemacht werden.
Zusätzlich wurde ein für \brand{Android} angepasstes Design entworfen und Grundkonzepte der Architektur erarbeitet.
Die Architektur ist noch nicht final, sie erleichtert uns aber den Einstig beim Programmieren.
Die Infrastruktur ist soweit aufgesetzt.

\subsubsection{Probleme}
Für die detaillierte Erarbeitung der Testspezifikation fehlte noch die nötige Erfahrung in \brand{React Native}.


\subsection{MS3: 1. Prototyp}
\label{pm-ms3}
\textbf{Fällig am 08.04.2016}
\subsubsection{Resultate}
\begin{itemize}
	\item \brand{Android} Prototyp
	\begin{itemize}
		\item Tab Navigation
		\item Darstellung der Karte
		\item Evaluation der flux-Architektur und Implementation des Grundgerüstes
		\item Missionen auf Map anzeigen
		\item Authentifizierung mit OAuth
	\end{itemize}
\end{itemize}

\subsubsection{Erledigte Arbeiten}
Die Entwicklungsumgebung wurde optimiert. Das \brand{Travis}-Konfigurationsfile prüft den Code nun mit \brand{ESLint}\footnote{\url{http://eslint.org/}} (\brand{JavaScript linter, checkt Styleguides}) und \brand{flow}\footnote{\url{http://flowtype.org/}} (static type checker).\newline
Für die Tab Navigation konnte eine Demo mit \brand{react-native-scrollable-tab-view}\footnote{\url{https://github.com/brentvatne/react-native-scrollable-tab-view}} umgesetzt werden. Diese muss noch in der \kort{} App eingebaut werden.\newline
\brand{OAuth} konnte ebenfalls anhand einer Demo getestet werden – allerdings nur mit \brand{Facebook} und \brand{Google}.\newline
Die Darstellung der Karte aus \nameref{pm-ms2} wurde leicht ausgebaut. Neu wird nun der Standort des Benutzers ermittelt.
Wir haben die Architekturvarianten evaluiert und uns für die flux-Architektur entschieden.
Daraufhin wurde ein Grundgerüst für die Verwendung vom \kort{}-API mit der flux-Architektur erstellt. 
Somit konnten die Missionen im Umkreis von fünf Kilometern geladen werden.

\subsubsection{Probleme}
Schwierigkeiten traten vor allem im Zusammenhang mit \brand{React Native} auf.
Oft gab es Build-Fehler bei gleicher Code-Basis.
Die Fehlerbehandlung hat uns enorm viel Zeit gekostet und es war schwer im Internet Hilfestellungen zu erhalten.
Da \brand{React Native} jede zwei Wochen ein Update erhält, sind die Dokumentationen oder die Diskussionen im Internet teilweise schon wieder veraltet.\newline
Im Austausch mit anderen \brand{React Native} Entwicklern haben wir dasselbe Feedback erhalten.\newline
Der Prototyp konnte nicht wie geplant fertiggestellt werden und einige Arbeiten mussten auf den nächsten Meilenstein ausgelagert werden.
Die Authentifizierung mit dem Backend konnte noch nicht umgesetzt werden.

\subsection{MS4: Zwischenpräsentation}
\label{pm-ms4}
\textbf{Fällig am 22.04.2016}
\subsubsection{Resultate}
\begin{itemize}
	\item 1. Prototyp fertig (s. \nameref{pm-ms3})
	\begin{itemize}
		\item Missionen auf Karte darstellen (Annotation-Marker)
		\item Tab Navigation
	\end{itemize}
	\item Zwischenpräsentation (Dauer ca. 30 Minuten)
	\begin{itemize}
		\item Aufgabenstellung, Problembesprechung
		\item IST-Situation
		\item geplantes Resultat
		\item Beschlussprotokoll für den \hyperref[pm-rollen]{Betreuer}
	\end{itemize}
\end{itemize}

\subsubsection{Erledigte Arbeiten}
Der Prototyp enthielt nun die Kartenansicht in einer Tab-Ansicht. 
Marker an den jeweiligen Positionen der geladenen Missionen konnten eingefügt werden.
Durch einen Klick auf einen Marker konnte der Titel der Mission erfolgreich in die Konsole geloggt werden.
Leider konnte noch immer keine Lösung für die Implementation eines Logins mit OAuth 1.0a (\brand{OpenStreetMap}) gefunden werden.
Dieses Feature noch zur Abgabe zu liefern wäre unrealistisch.
Deswegen wurde es auf den \hyperref[pm-ms9]{Meilenstein 9: Salzburg Konferenz} verschoben.

\subsubsection{Probleme}
Weiterhin traten Schwierigkeiten mit willkürlichen Fehlern der \brand{React Native} App auf. 
Aus diesen Gründen konnte die Navigation und die Struktur der App nicht abschliessend umgesetzt werden.
Dass der Benutzer nach dem Login automatisch zur Tab-Ansicht weitergeleitet wurde, hatte noch nicht geklappt.
Zwischenzeitlich wurde aus diesem Grund das Erhalten des Login-Tokens, nach der Authentifizierung über \brand{Google} und \brand{Facebook}, in einer separaten App getestet.


\subsection{MS5: 1. Release}
\label{pm-ms5}
\textbf{Fällig am 20.05.2016}
\subsubsection{Resultate}
\begin{itemize}
	\item View-Komponenten fertig entworfen
	\item Backend API mit der flux-Architektur eingebunden
	\item Token basierte Authentifizierung im Frontend umgesetzt, nachdem das Backend von Stefan Oderbolz dafür angepasst wurde.
\end{itemize}

\subsubsection{Erledigte Arbeiten}
Unterdessen konnte die GUI grösstenteils fertiggestellt werden. 
Es fehlte noch die dynamische Behandlung von gewissen View-Komponenten.
Zum Beispiel die Unterscheidung ob beim Lösen einer Mission ein Picker, zum Auswählen der Antwort, gerendert wird, oder ob ein Text-Input-Feld angeboten wird.
Gleichzeitig wurden alle API-Aufrufe mit der flux-Architektur eingebunden - aber noch nicht vollständig getestet.

\subsubsection{Probleme}
Weiterhin war es nicht möglich mit der Router-Flux-Navigationskomponente eine Weiterleitung, nach Erfolgreichem Login, zur Kartenansicht umzusetzen.
Hierbei handelte es sich um einen bekannten Fehler dieser Komponente.
Das Verzögerte leider das geplante Ziel: eine Mission zu lösen.

\subsection{MS6: 2. Release}
\label{pm-ms6}
\textbf{Fällig am 03.06.2016}
\subsubsection{Resultate}
\begin{itemize}
	\item View-Komponenten fertiggestellt
	\item Login Weiterleitung
	\item Fertigstellung der Architektur
	\item Testing abgeschlossen
\end{itemize}

\subsubsection{Erledigte Arbeiten}
Alle View-Komponenten sind nun dynamisch und zeigen je nach Zustand der Komponente die entsprechende Oberfläche.
Durch eine zusätzliche App-Loader-View konnte das Problem der Weiterleitung nach dem Login behoben werden.
Diese Komponente lädt nun alles im Voraus und erst danach wird je nach Zustand, ob der Benutzer eingeloggt ist, oder nicht, entsprechend weitergeleitet.
Parallel wurde die Architektur und die Logik fertiggestellt und konnte nach und nach von den View-Komponenten verwendet werden.
Schlussendlich konnten wir erfolgreich eine Mission lösen.
% ToDo: Weitere erledigte Arbeiten

\subsubsection{Probleme}


\subsection{MS7: Schlussabgabe}
\label{pm-ms7}
\textbf{Fällig am 17.06.2016}
\subsubsection{Resultate}
\begin{itemize}
	\item Dokumentation vollständig
	\item gebundene Dokumentation eingereicht
	\item CD mit Abgabe eingereicht
	\item Abstract eingereicht
	\item Poster eingereicht
\end{itemize}

\subsubsection{Erledigte Arbeiten}


\subsubsection{Probleme}


\subsection{MS: Schlusspräsentation}
\label{pm-m8}
\textbf{Fällig am 29.06.2016}
\subsubsection{Ziele}
