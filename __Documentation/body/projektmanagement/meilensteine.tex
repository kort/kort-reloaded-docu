\section{Meilensteine}
\subsection{MS1: Kickoff}
\label{pm-ms1}
\textbf{Fällig am 25.02.2016}
\subsubsection{Resultate}
\begin{itemize}
	\item Kickoff Meeting bei Liip mit Jürg Hunziker, Stefan Oderbolz und Stefan Keller
\end{itemize}

\subsection{MS2: Ende Elaboration}
\label{pm-ms2}
\textbf{Fällig am 29.03.2016}
\subsubsection{Resultate}
\begin{itemize}
	\item Ausgangslage
	\item Anforderungsspezifikation
	\item Risikomanagement
	\item Projektplan
	\item Map Komponente ausgewählt und eingesetzt
	\item Aufgabenstellung
	\item Testspezifikation
	\item Infrastruktur
	\begin{itemize}
		\item Datenbank
		\item CI
		\item uservoice
		\item Redmine
		\item Installationsskripte
	\end{itemize}
\end{itemize}

\subsubsection{Erledigte Arbeiten}
Vollständig dokumentiert wurden die Ausgangslage, die Anforderungsspezifikation, der Projektplan und das Risikomanagement.
Die Aufgabenstellung wurde soweit definiert, dass sie mit dem Projektplan übereinstimmt.
Die endgültige Aufgabenstellung ist erst zur Zwischenpräsentation, in der Mitte des Semesters, geplant.
Für die Map Komponente wurde die MapBox GL Library\footnote{\url{https://libraries.io/npm/react-native-mapbox-gl}} mit den Vektor Daten von osm2vectortiles\footnote{\url{http://osm2vectortiles.org/}}.
Nebenbei konnten wichtige Erfahrungen in den Technologien (\brand{Java Script}, \brand{React}, \brand{React Native}) gemacht werden.
Zusätzlich wurde ein für \brand{Android}-Geräte angepasstes Design entworfen und Grundkonzepte der Architektur erarbeitet.
Die Architektur ist noch nicht final, sie erleichtert uns aber den Einstig beim Programmieren.
Die Infrastruktur ist soweit aufgesetzt.
Es fehlten zu diesem Zeitpunkt nur noch die Zugangsdaten, welche aber noch nicht dringend gebraucht wurden.

\subsubsection{Probleme}
Für die detaillierte Erarbeitung der Testspezifikation fehlte noch die nötige Erfahrung in \brand{React Native}.


\subsection{MS3: 1. Prototyp}
\label{pm-ms3}
\textbf{Fällig am 08.04.2016}
\subsubsection{Resultate}
\begin{itemize}
	\item \brand{Android} Prototyp
	\begin{itemize}
		\item Darstellung der Karte
		\item Missionen auf Map anzeigen
		\item falls möglich:
		\begin{itemize}
			\item Highscore
			\item Profile
			\item About Kort
		\end{itemize}
	\end{itemize}
\end{itemize}

\subsection{MS4: Zwischenpräsentation}
\label{pm-ms4}
\textbf{Fällig am 22.04.2016}
\subsubsection{Resultate}
\begin{itemize}
	\item 1. Prototyp ausgebaut
	\item Zwischenpräsentation (Dauer ca. 1h)
	\begin{itemize}
		\item Präsentation der Ergebnisse
		\item Diskussion bezüglich weiterer Arbeit
	\end{itemize}
\end{itemize}

\subsection{MS5: 1. Release}
\label{pm-ms5}
\textbf{Fällig am 13.05.2016}
\subsubsection{Resultate}
\begin{itemize}
	\item die Funktionalität der ursprünglichen \kort{} Web-App ist als Native App implementiert
\end{itemize}

\subsection{MS6: 2. Release}
\label{pm-ms6}
\textbf{Fällig am 03.06.2016}
\subsubsection{Resultate}
\begin{itemize}
	\item die Native App enthält alle vorgegebenen Features
	\item sämtliche Tests wurden erfolgreich durchlaufen
\end{itemize}

\subsection{MS7: Schlussabgabe}
\label{pm-ms7}
\textbf{Fällig am 17.06.2016}
\subsubsection{Resultate}
\begin{itemize}
	\item Dokumentation vollständig
	\item gebundene Dokumentation eingereicht
	\item CD mit Abgabe eingereicht
	\item Abstract eingereicht
	\item Poster eingereicht
	\item evtl. App deployed
\end{itemize}