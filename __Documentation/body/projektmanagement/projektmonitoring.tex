\chapter{Projektmonitoring}
\label{pm-projektmonitoring}

\section{Zeitanalyse}
Für eine Bachelorarbeit werden 12 ECTS-Punkte vergeben, wobei ein Punkt einem Aufwand von 30 Stunden entspricht.
In einem Team von zwei Entwicklern, entspricht dies 360 Stunden Aufwand pro Person. 
Leider haben wir diese Vorgabe (siehe Tabelle \ref{pm-arbeitsaufwand}) überschritten. 

\begin{table}[H]
\centering
\begin{tabular}{|l|r|}
\hline 
\multicolumn{1}{|c|}{\textbf{Person}} & \multicolumn{1}{|c|}{\textbf{Aufwand}} \\
\hline 
Marino Melchiori & 461.5 h \\
\hline 
Dominic Mülhaupt & 467 h \\  
\hline 
\end{tabular}
\caption{Arbeitsaufwand pro Person}
\label{pm-arbeitsaufwand}
\end{table}

Wenn die gleich folgenden Tabellen \ref{pm-arbeitsaufwand-aktivität-mm} und \ref{pm-arbeitsaufwand-aktivität-dm} verglichen werden, ist die Arbeitsaufteilung sehr gut erkennbar. 
Dies ist in den Unterschieden des Aufwandes, in den Aktivitäten Analyse, Implementation, Dokumentation und Testing, erkennbar. 
Bei neuen Erfahrungen haben sich die Entwickler ausgetauscht und waren so immer auf dem aktuellen Stand. 

\begin{table}[H]
\centering
\label{pm-arbeitsaufwand-aktivität-mm}
\begin{tabular}{|l|r|}
\hline
\multicolumn{2}{|l|}{\textbf{Marino Melchiori}} \\
\hline
\multicolumn{1}{|c|}{\textbf{Aktivität}} & \multicolumn{1}{|c|}{\textbf{Aufwand}} \\
\hline
Analyse \& Design & 106.00 h \\
\hline
Implementation & 125.00 h \\
\hline
Dokumentation & 141.00 h \\
\hline
Requirements & 30.75 h \\
\hline
Deployment & 16.25 h \\
\hline
Allgemein & 16.75 h \\
\hline
Projektmanagement & 25.75 h \\
\hline
\end{tabular}
\caption{Aufwand pro Aktivität --- Marino Melchiori}
\end{table}

\begin{table}[H]
\centering
\label{pm-arbeitsaufwand-aktivität-dm}
\begin{tabular}{|l|r|}
\hline
\multicolumn{2}{|l|}{\textbf{Dominic Mülhaupt}} \\
\hline
\multicolumn{1}{|c|}{\textbf{Aktivität}} & \multicolumn{1}{|c|}{\textbf{Aufwand}} \\
\hline
Analyse \& Design & 49.50 h \\
\hline
Implementation & 191.75 h \\
\hline
Dokumentation & 66.25 h \\
\hline
Requirements & 26.00 h \\
\hline
Deployment & 7.00 h \\
\hline
Allgemein & 52.50 h \\
\hline
Projektmanagement & 52.00 h \\
\hline
Testing & 22.00 h \\
\hline
\end{tabular}
\caption{Aufwand pro Aktivität --- Dominic Mülhaupt}
\end{table}


\subsection{Soll-Ist-Zeitvergleich}
Die Tabelle zum \ref{pm-arbeitsaufwand-kategorie-ges} zeigt, wie das Projekt kategorisiert wurde. 
In der \textit{Kort Projekt}-Kategorie ging es vor allem um die Implementation der Architektur und die Verwendung und Einrichtung von \brand{React Native}. 
Die Kategorien \textit{Login}, \textit{Missionen}, \textit{Profil} und \textit{Highscore} enthalten die Implementation der \gls{GUI}. 
\textit{Map} und \textit{Internationalisierung} sind Kategorien, bei denen es um die Installation der entsprechend verwendeten \glslink{Library}{Libraries} ging. 
In der Kategorie \textit{Technologien} wurde die Einarbeitungszeit gebucht. 
Die \textit{Qualitätssicherung} beinhaltet das Testing und die Kategorie \textit{Allgemein} die Evaluation von verwendeten Konzepten und \glslink{Library}{Libraries}.

Die Schätzung fand jeweils bei der Erstellung eines Tickets in einer Kategorie statt. 

Die Differenz der \textit{Kort Projekt}-Kategorie lässt sich durch die häufigen Build-Fehler erklären. 
Das Suchen nach Lösungen im Internet war sehr Zeitaufwendig. 

Bei der Kategorie \textit{Missionen} wurden 15 Stunden zu viel geschätzt. 
In diesem Fall setzten wir vermehrt auf \gls{Pair Programming}, da es sich dort um eine der ersten umgesetzten Funktionen handelte.
Somit konnte kostbare Zeit, die für das Refactoring geplant war, eingespart werden. 
Später folgende Features, der Kategorien \textit{Login}, \textit{Profil} und \textit{Highscore}, wurden viel besser geschätzt.

Schlussendlich wurden 21 Stunden zu viel geschätzt, was bei einem Arbeitstag von 8 Stunden etwa 2.5 Arbeitstagen entspricht. 


\begin{table}[H]
\centering
\label{pm-arbeitsaufwand-kategorie-ges}
\begin{tabular}{|l|r|r|r|}
\hline
\multicolumn{4}{|l|}{\textbf{Soll-Ist-Vergleich vom Gesamtaufwand der Kategorien}} \\
\hline
\multicolumn{1}{|c|}{\textbf{Kategorie}} & \multicolumn{1}{|c|}{\textbf{Soll-Aufwand}} & \multicolumn{1}{|c|}{\textbf{Ist-Aufwand}} & \multicolumn{1}{|c|}{\textbf{Differenz}}\\
\hline
Kort Projekt & 71.75 h & 88.25 h & +16.50 h \\
\hline
Login & 91.50 h & 92.50 h & +1.00 h \\
\hline
Map & 47.75	h & 45.50 h & -2.25 h \\
\hline
Missionen & 59.75 h & 44.75 h & -15.00 h \\
\hline
Highscore & 11.50 h & 12.70 h & +1,20 h \\
\hline
Profil & 18.00 h & 16.25 h & -1.75 h \\
\hline
Technologien & 182.50 h & 187.55 h & +0.05 h \\
\hline
Gamification und Design & 7.25 h & 4.75 h & -2.50 h \\
\hline
Qualitätssicherung & 36.50 h & 30.00 h & -6.50 h \\
\hline
Internationalisierung & 4.50 h & 4.25 h & -0.25 h \\
\hline
Dokumentation & 204.00 h & 198.25 h & -5.75 h \\
\hline
Allgemein & 152.75 h & 144.50 h & -8.25 h \\
\hline
Meeting & 61.75 h & 59.25 h & -2.50 h \\
\hline
\textbf{Total} & \textbf{975.00 h} & \textbf{926.50 h} & \textbf{-21.00 h} \\
\hline
\end{tabular}
\caption{Soll-Ist-Vergleich --- Gesamtaufwand pro Kategorie}
\end{table}

\section{Code-Statistik}

% Anzahl Zeilen pro Package (Actions, Stores, etc...)
Zur groben Einschätzung, der Grösse vom Projekt wurde in der Tabelle \ref{pm-cloc} die Gesamtanzahl der Dateien und der Code-Zeilen (ohne Code-Kommentare) gezählt.

\begin{table}[H]
\centering
\begin{tabular}{|l|l|l|}
\hline 
\textbf{Sprache} & \textbf{Dateien} & \textbf{Zeilen} \\ 
\hline 
\brand{JavaScript} & ... & ... \\
\hline 
\end{tabular}
\caption{Dateien und Codezeilen}
\label{pm-cloc}
\end{table}

Die Tabelle \ref{pm-package-cloc} zeigt die Anzahl an Code-Zeilen in einem Package. 
Die Packages \textit{Actions}, \textit{Data}, \textit{Dispatcher}, \textit{DTO} und \textit{Stores} sind von der verwendeten Architektur gegeben. 
\textit{Constants} beinhaltet Konstanten, IDs und sonstige fixe Parameter. 
Im Package \textit{Components} befinden sich die \gls{GUI}-Komponenten. 

\begin{table}[H]
\centering
\begin{tabular}{|l|l|}
\hline 
\textbf{Package} & \textbf{Zeilen} \\ 
\hline 
\brand{Actions} & ... \\
\hline 
\brand{Components} & ... \\
\hline 
\brand{Constants} & ... \\
\hline 
\brand{Data} & ... \\
\hline 
\brand{Dispatcher} & ... \\
\hline 
\brand{DTOs ohne Tests} & ... \\
\hline 
\brand{DTO Tests} & ... \\
\hline 
\brand{Stores ohne Tests} & ... \\
\hline 
\brand{Store Tests} & ... \\
\hline 
\end{tabular}
\caption{Codezeilen pro Package}
\label{pm-package-cloc}
\end{table}
