% Typ des Dokuments
\documentclass[abstracton, a4paper, 12pt]{scrreprt}

% Encoding (utf8)
\usepackage[utf8]{inputenc}
\usepackage[T1]{fontenc}

% Silbentrennung (Neu-Deutsch)
\usepackage[ngerman]{babel}

% Literaturverzeichnis (Deutsch)
\usepackage{bibgerm}
% Spezialseiten (Literarturverzeichnis, Abbildungsverzeichnis, usw.) in Inhaltsverzeichnis anzeigen
\usepackage[nottoc]{tocbibind}

% Farben
\usepackage{color}
\usepackage[table]{xcolor}
\definecolor{darkgreen}{rgb}{0,0.6,0}
\definecolor{darkgrey}{rgb}{0.5,0.5,0.5}
\definecolor{grey}{rgb}{0.8,0.8,0.8}
\definecolor{lightgrey}{rgb}{0.95,0.95,0.95}
\definecolor{mauve}{rgb}{0.58,0,0.82}

% Schriften
\usepackage{pifont}
\newcommand{\tick}{\ding{51}\hspace{0.2cm}}
\newcommand{\cross}{\ding{55}\hspace{0.2cm}}

% Grafiken
\usepackage[pdftex]{graphicx}
\usepackage{epsfig}
% Umfliessen von Text um Tabellen und Bilder
\usepackage{wrapfig}

% PDFs
\usepackage{pdfpages}

% Grafiken korrekt positionieren
\usepackage{float}
\restylefloat{figure}
\usepackage[section]{placeins}
\usepackage{subfigure}
% Zahlen verwenden für Subfigure counter
\renewcommand{\thesubfigure}{(\arabic{subfigure})}

% Hyperlinks und URLs
\usepackage[hyphens]{url}
\usepackage{hyperref}
\hypersetup{
   colorlinks,%
   citecolor=blue,%
   filecolor=blue,%
   linkcolor=blue,%
   urlcolor=blue
}
\urlstyle{same}

% Absatz
\setlength{\parindent}{0pt} % Absatzeinzug
\setlength{\parskip}{10pt} % Absatzabstand

% Glossar
\usepackage[toc]{glossaries}
\makeglossary

% Kontrolle über Listen-Eigenschaften
\usepackage{enumitem}

% Abstaende bei Ueberschriften
\usepackage{titlesec}
% \titlespacing*{command}{left}{before-sep}{after-sep}[right-sep]
\titlespacing{\section}{0em}{12pt}{3pt}
\titlespacing{\subsection}{0em}{10pt}{2pt}
\titlespacing{\subsubsection}{0em}{8pt}{0em}

% TODO Kommentare
\usepackage{todonotes}

%%%%%%%%%%%%%%%%%%%%%%%%%%%%%%%%%%%%%%%%%%%%%%%%%%%
% Kopf- und Fusszeile
%%%%%%%%%%%%%%%%%%%%%%%%%%%%%%%%%%%%%%%%%%%%%%%%%%%
% Seitenränder
\usepackage[inner=2.2cm,outer=2.2cm,top=1.7cm,bottom=1.7cm,includeheadfoot]{geometry}

% Kopf- und Fusszeile mit Linien
\usepackage[automark,headsepline,footsepline]{scrpage2}

\pagestyle{scrheadings}
% Kopf- und Fusszeile auch bei Kapitel- und Partsanfangsseiten
\renewcommand*{\chapterpagestyle}{scrheadings}
\renewcommand*{\partpagestyle}{scrheadings} 

% Kopf- und Fusszeile leeren
\clearscrheadfoot

% Inhalt der Kopfzeile
\ihead{\footnotesize{\leftmark}}

% Inhalt der Fusszeile
\ifoot{\footnotesize{Kort Reloaded – A Gamified App for Collecting OpenStreetMap Data}}
\ofoot{\footnotesize{\thepage}}

% MakeUppercase überschreiben, um Grosschreibung in Kopfzeile für Spezialseiten zu deaktivieren (Achtung böser Hack!)
\renewcommand*\MakeUppercase[1]{#1}

%%%%%%%%%%%%%%%%%%%%%%%%%%%%%%%%%%%%%%%%%%%%%%%%%%%
% Tabellen
%%%%%%%%%%%%%%%%%%%%%%%%%%%%%%%%%%%%%%%%%%%%%%%%%%%
% Für Tabellen, welche über mehrere Seiten gehen
\usepackage{longtable}

% Mehrere Spalten zusammenfassen
\usepackage{hhline}

% Mehrere Zeilen zusammenfassen
% \usepackage{multirow}
\usepackage{array}

% Tabularx
\usepackage{tabularx, booktabs}

% Padding links und rechts von Zelle
\setlength{\tabcolsep}{5px}
% Padding oben und unten (mittels arraystretch)
\renewcommand{\arraystretch}{1.3}

% Variabeln für Tabellenbreiten definieren
% 2-spaltige Tabelle
\newlength{\twocelltabwidth}
\setlength{\twocelltabwidth}{\textwidth}
\addtolength{\twocelltabwidth}{-4\tabcolsep - 3px} % subtrahiere 4x Padding (\tabcolsep) und 3 Rahmen

% 3-spaltige Tabelle
\newlength{\threecelltabwidth}
\setlength{\threecelltabwidth}{\textwidth}
\addtolength{\threecelltabwidth}{-6\tabcolsep - 4px} % subtrahiere Padding (\tabcolsep) und Rahmen

% 4-spaltige Tabelle
\newlength{\fourcelltabwidth}
\setlength{\fourcelltabwidth}{\textwidth}
\addtolength{\fourcelltabwidth}{-8\tabcolsep - 5px} % subtrahiere Padding (\tabcolsep) und Rahmen

%%%%%%%%%%%%%%%%%%%%%%%%%%%%%%%%%%%%%%%%%%%%%%%%%%%
% Syntaxhighlighter
%%%%%%%%%%%%%%%%%%%%%%%%%%%%%%%%%%%%%%%%%%%%%%%%%%%
% Syntaxhighlighter (benoetigt color und xcolor package)
\usepackage{listings}
\renewcommand{\lstlistingname}{Code-Ausschnitt}

\lstset{ %
  language=HTML,                % the language of the code
  basicstyle=\footnotesize,           % the size of the fonts that are used for the code
  numbers=left,                   % where to put the line-numbers
  numberstyle=\tiny\color{darkgrey},  % the style that is used for the line-numbers
  stepnumber=1,                   % the step between two line-numbers. If it's 1, each line will be numbered
  numbersep=5pt,                  % how far the line-numbers are from the code
  backgroundcolor=\color{white},  % choose the background color. You must add \usepackage{color}
  showspaces=false,               % show spaces adding particular underscores
  showstringspaces=false,         % underline spaces within strings
  showtabs=false,                 % show tabs within strings adding particular underscores
  frame=single,                   % adds a frame around the code
  rulecolor=\color{darkgrey},        % if not set, the frame-color may be changed on line-breaks within not-black text (e.g. commens (green here))
  tabsize=2,                      % sets default tabsize to 2 spaces
  captionpos=b,                   % sets the caption-position to bottom
  breaklines=true,                % sets automatic line breaking
  breakatwhitespace=false,        % sets if automatic breaks should only happen at whitespace
  title=\lstname,                   % show the filename of files included with \lstinputlisting;
                                  % also try caption instead of title
  keywordstyle=\color{blue},          % keyword style
  commentstyle=\color{darkgreen},       % comment style
  stringstyle=\color{mauve},         % string literal style
  escapeinside={\%*}{*)},            % if you want to add a comment within your code
  morekeywords={*,...}               % if you want to add more keywords to the set
}

% Javascript Syntaxhighliting
\lstdefinelanguage{JavaScript} {
	morekeywords={
		break,const,continue,delete,do,while,export,for,in,function,
		if,else,import,in,instanceOf,label,let,new,return,switch,this,
		throw,try,catch,typeof,var,void,with,yield
	},
	sensitive=false,
	morecomment=[l]{//},
	morecomment=[s]{/*}{*/},
	morestring=[b]",
	morestring=[d]'
}
\lstset{
	frame=tb,
	framesep=5pt,
	basicstyle=\footnotesize\ttfamily,
	showstringspaces=false,
	keywordstyle=\ttfamily\bfseries\color{blue},
	identifierstyle=\ttfamily,
	stringstyle=\ttfamily\color{mauve},
	commentstyle=\color{darkgreen},
	rulecolor=\color{darkgrey},
	xleftmargin=5pt,
	xrightmargin=5pt,
	aboveskip=\bigskipamount,
	belowskip=\bigskipamount
}

\lstdefinestyle{examples}
{numbers=none, frame=none}

% Inline Code-Formatierung
\newcommand{\inlinecode}{\texttt}

% Applikationsname Kort
\newcommand\kort{\textsc{Kort}}

% Marken- oder Produktnamen
\newcommand\brand{\emph}

% Silbentrennung
\hyphenation{Web-app-li-ka-tion}
\hyphenation{Ja-va-Scr-ipt}

% Glossar
\newglossaryentry{NativeApp} {
	name = Native App,
	description = {Native Apps im engeren Sinn zeichnen sich dadurch aus, dass sie speziell an die Zielplattform angepasst und sehr leicht über ein herstellerspezifisches Online-Portal bezogen und installiert werden können\cite{mobileapp}}
}

\newglossaryentry{API} {
	name = API,
	description = {Application Programming Interface, Schnittstelle für die Programmierung\cite{ba-kort-2012}}
}

\newglossaryentry{OAuth} {
	name = OAuth,
	description = {OAuth ist ein offenes Protokoll, das eine standardisierte, sichere API-Autorisierung erlaubt\cite{oauth}}
}

\newglossaryentry{CRUD} {
	name = CRUD,
	description = {Das Akronym CRUD beschreibt die 4 Standardoperationen einer Datenbank: \textbf{C}reate, \textbf{R}ead, \textbf{U}pdate, \textbf{D}elete\cite{crud}}
}

\newglossaryentry{WebApp} {
	name = Web-App,
	description = {Der Begriff Web-App (von der englischen Kurzform für web application) bezeichnet im allgemeinen Sprachgebrauch Apps für mobile Endgeräte wie Smartphones und Tablet-Computer. 
	Diese Apps laufen auf dem im Betriebssystem integrierten Browser, werden aus dem Internet geladen und können so ohne Installation auf dem mobilen Endgerät genutzt werden\cite{webapp}}
}

\newglossaryentry{REST} {
	name = REST,
	description = {Representational State Transfer\cite{rest} ist ein Programmierparadigma, welches besagt, dass sich der Zustand einer Webapplikation als Ressource in Form einer URL beschreiben lässt. Auf solche Ressourcen können folgende Befehle angewendet werden: \inlinecode{GET}, \inlinecode{POST}, \inlinecode{PUT}, \inlinecode{PATCH}, \inlinecode{DELETE}, \inlinecode{HEAD} und \inlinecode{OPTIONS}.
	HTTP ist ein Protokoll welches REST implementiert\cite{ba-kort-2012}}
}

\newglossaryentry{Git} {
	name = Git,
	description = {Git ist ein verteiltes Versionsverwaltungssystem für Dateien. Ursprünglich wurde es für die Entwicklung des Linux Kernels kreiert\cite{ba-kort-2012}}
}

\newglossaryentry{CI} {
	name = Continuous Integration,
	description = {Der Begriff \emph{Continuous Integration}\cite{cont-integration} beschreibt die Idee, dass Änderungen an einer Software schnell eingebracht werden sollen. Dazu zählt, dass diese in einem Tool zur Versionsverwaltung eingetragen und durch automatisierte Tests geprüft werden\cite{ba-kort-2012}}
}

\newglossaryentry{POI} {
	name = POI,
	description = {Abkürzung für Point of Interest. Dies ist ein allgemeiner Begriff für einen Ort mit irgendeiner Bedeutung, sei es eine Schule, Kirche, Bushaltestelle oder sonst etwas von besonderem Interesse\cite{ba-kort-2012}}
}

\newglossaryentry{Mapper} {
	name = Mapper,
	description = {Personen welche auf OpenStreetMap die Karten ergänzen und pflegen, nennen sich selbst \emph{Mapper}\cite{ba-kort-2012}}
}

\newglossaryentry{App Store} {
	name = App Store,
	description = {Ein App Store ist eine Verkaufsplattform eines Betriebssystemherstellers für Smartphones, beispielsweise Google Play für Android, Apple App Store für iOS\cite{ba-kort-2012}}
}

\newglossaryentry{Node} {
	name = Node,
	description = {Ein Node ist das grundlegendste Objekt in OpenStreetMap, es wird durch seine Attribute (genannt \glspl{Tag}) genauer bestimmt. Nodes können Teil eines \gls{Way} sein\cite{ba-kort-2012}}
}

\newglossaryentry{Way} {
	name = Way,
	description = {Ein Weg oder eine Strasse wird in OpenStreetMap als \emph{Way} bezeichnet. Es handelt sich dabei um eine Serie von miteinander verbundenen Nodes\cite{ba-kort-2012}}
}

\newglossaryentry{Relation} {
	name = Relation,
	description = {Eine Relation stellt eine Verbindung zwischen verschiedenen OSM-Objekten dar. Relationen werden meist dazu gebraucht, um übergeordnete Beziehungen darzustellen. Beispielsweise können alle Bushaltestellen einer Buslinie über eine Relation miteinander verknüpft sein\cite{ba-kort-2012}}
}

\newglossaryentry{Tag} {
	name = Tag,
	description = {Ein Tag bezeichnet ein Attribut eines \glslink{Node}{Nodes}. Es besteht aus einem Namen und einem Wert. Ein \gls{Node} kann beliebig viele Tags beinhalten\cite{ba-kort-2012}}
}

\newglossaryentry{OpenStreetMap} {
	name = OpenStreetMap,
	description = {OpenStreetMap ist ein freies Projekt, das für jeden frei nutzbare Geodaten sammelt (Open Data). Mit Hilfe dieser Daten können Weltkarten errechnet oder Spezialkarten abgeleitet werden sowie Navigation betrieben werden\cite{ba-kort-2012}}
}

\newglossaryentry{Gamification} {
	name = Gamification,
	description = {Unter Gamification versteht man das Hinzufügen von Spielelementen in einen nicht spieltypischen Kontext.\cite{for-the-win}}
}

\newglossaryentry{MVC} {
	name = MVC,
	description = {Der englischsprachige Begriff \emph{model view controller} (MVC) ist ein Muster zur Strukturierung von Software-Entwicklung in die drei Einheiten Datenmodell (model), Präsentation (view) und Programmsteuerung (controller)\cite{patterns}}
}

\newglossaryentry{Crowdsourcing} {
	name = Crowdsourcing,
	description = {Unter Crowdsourcing versteht man die Auslagerung von traditionell internen Teilaufgaben an eine Menge von freiwilligen Usern im Internet\cite{crowdsourcing}}
}

\newglossaryentry{Routing} {
	name = Routing,
	description = {Routing beschreibt den Vorgang einen optimalen Weg zwischen zwei oder mehr Punkten zu finden. Eine so gefundene Strecke wird oft als \emph{Route} bezeichnet\cite{ba-kort-2012}}
}

\newglossaryentry{Pair Programming}{
	name = Pair Programming,
	description = {Bei Pair Programming handelt es sich um eine Arbeitsweise, bei der zwei Programmierer an einem Rechner arbeiten. Dabei ist ein Programmierer damit beschäftigt, den Code zu schreiben, während der andere über die Problemstellungen nachdenkt, den geschriebenen Code kontrolliert und Probleme, die ihm dabei auffallen, sofort anspricht\cite{pairprogramming}}
}

\newglossaryentry{Minimum Viable Product}{
	name = Minimum Viable Product,
	description = {Das Minimum Viable Product ist ein Produkt, welches gerade die Kernfunktionalität aufweist, die nötig ist um es zu veröffentlichen\cite{mvp}}
}

\newglossaryentry{Framework}{
	name = Framework,
	description = {Ein Framework ist noch kein fertiges Programm. 
	Es ist ein Rahmen (ein Gerüst) welches der Programmierer verwenden kann um sein eigenes, persönliches Programm zu erstellen\cite{framework}}
}

\newglossaryentry{GUI}{
	name = GUI,
	description = {GUI ist eine Abkürzung für graphical user interface und bedeutet grafische Benutzer-oberfläche\cite{gui}}
}

\newglossaryentry{Library}{
	name = Library,
	description = {Eine Library (Programmbibliothek) bezeichnet in der Programmierung eine Sammlung von Unterprogrammen, die Lösungswege anbieten. 
	Bibliotheken laufen im Unterschied zu Programmen nicht eigenständig, sondern sie enthalten Hilfsmodule, die angefordert werden können\cite{library}},
	plural= Libraries
}

\newglossaryentry{DOM}{
	name = DOM,
	description = {DOM bedeutet Document Object Model und ist eine Spezifikation für den Zugriff auf HTML-Dokumente. 
	Anhand des DOMs kann ein Computerprogramm den Inhalt der HTML-Elemente dynamisch verändern. 
	Für einen Webentwickler wäre das der HTML-Code.\cite{dom}}
}

\newglossaryentry{Virtual DOM}{
	name = Virtual DOM,
	description = {Das virtual DOM ist eine Abstraktion des DOM. 
	Es ist eine lokale Kopie des DOM\cite{virtual-dom}}
}

\newglossaryentry{JSX}{
	name = JSX,
	description = {JSX ist eine \brand{JavaScript}-Syntax-Erweiterung die ähnlich wie XML aussieht\cite{jsx}}
}

\newglossaryentry{Backend}{
	name = Backend,
	description = {Als Backend wird das auf dem Server laufenden Programm bezeichnet}
}

\newglossaryentry{Frontend}{
	name = Frontend,
	description = {Als Frontend wird das beim Client laufende Programm bezeichnet}
}

\newglossaryentry{DTO}{
	name = DTO,
	description = {Ein Datentransferobjekt ist ein Objekt, das nur zur Übertragung von Daten oder Informationen dient und keine logischen Komponenten enthält}
}

\newglossaryentry{Singleton}{
	name = Singleton,
	description = {Das Singleton beschreibt ein Entwurfsmuster, wonach sichergestellt wird, dass von einer Klasse immer höchstens ein Objekt existiert, welches üblicherweise global verfügbar ist\cite{singleton}}
}

\newglossaryentry{IDE}{
	name = IDE,
	description = {IDE bedeutet Integrierte Entwicklungsumgebung -- auf Englisch integrated development environment\cite{ide}}
}

\newglossaryentry{Unit Test}{
	name = Unit Test,
	description = {Unit Tests werden geschrieben um kleine Teile des Codes -- üblicherweise Module -- zu testen.
	Da wirklich nur das Verhalten des entsprechenden Moduls getestet werden soll, müssen Abhängigkeiten dieses Moduls durch \glslink{Mock}{Mocks} ersetzt werden}
}

\newglossaryentry{Integration Test}{
	name = Integration Test,
	description = {In Integration Tests wird die Zusammenarbeit von Systemteilen getestet}
}

\newglossaryentry{Funktionaler Test}{
	name = Funktionaler Test,
	description = {Funktionale Tests überprüfen das Gesamtverhalten des Systems.
	Mit einem automatisierten UI Testing Tool werden verschiedene Szenarien durchlaufen, welche aus den Use Cases abgeleitet werden können.
	Funktionale Tests werden im Vergleich zu \glslink{Unit Test}{Unit Tests} und \glslink{Integration Test}{Integration Tests} eher spärlich eingesetzt}
}
	
\newglossaryentry{Mock}{
	name = Mock,
	description = {Mock-Objekte werden in diversen Tests verwendet um das Verhalten eines anderen Objekts für spezifische Fälle zu simulieren ohne es tatsächlich zu implementieren}
} 

\newglossaryentry{Packager}{
	name = Packager,
	description = {Der Packager kompiliert und bundelt den JavaScript Code für das Smartphone während der Entwicklungsphase und dem Testen auf dem Gerät}
} 